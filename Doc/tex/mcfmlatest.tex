\section{New features in this release}
\label{sec:latest}

Version 10.2 of the code introduces the ability to compute diboson processes
to NNLO.  It also allows all NNLO calculations to be performed using two
variants of slicing: using 0-jettiness (as in previous versions) or $q_T$ (new).
We have included a suite of input files that have been set up
to showcase all processes that may be computed at NNLO in this version, located
in {\tt Bin/2202.07738\_nonlocal/nnlo} (and similarly for {\tt lo} and {\tt nlo}).
These files also reproduce the results presented in Ref.~\cite{Campbell:2022gdq}.
A list of the processes, input files and results (NNLO coefficient and running time)
is shown in Table~\ref{table:nnlo10}.

\renewcommand{\arraystretch}{1.05}
\begin{table}
\begin{tabular}{llllr}
\toprule
Final state & input file & precision & $\delta_{NNLO}$ & Time (CPU days) \\
\midrule
$H$
& in\texttt{put\_H\_qt.int}   & 0.01 & 10.13(0.09)pb & 2.18 \\ % 187978 \\
& in\texttt{put\_H\_scet.int} &      & 10.09(0.09)pb & 4.07 \\[2pt] % 351574 \\
$Z$
& in\texttt{put\_Z\_qt.int}   & 0.05 & -388(20)pb & 313 \\ % 27007796. \\
& in\texttt{put\_Z\_scet.int} &      & -371(18)pb & 261 \\[2pt] % 22543465. \\
$W^-$
& in\texttt{put\_W-\_qt.int}   & 0.03 & -667(20)pb & 268 \\ % 23154067. \\
& in\texttt{put\_W-\_scet.int} &      & -624(19)pb & 213 \\[2pt] % 18434954. \\
$W^+$
& in\texttt{put\_W+\_qt.int}   & 0.06 & -358(21)pb & 556 \\ % 48061977. \\
& in\texttt{put\_W+\_scet.int} &      & -275(16)pb & 555 \\[2pt] % 47926621. \\
$ZH$
& in\texttt{put\_ZH\_qt.int}   & 0.01 & 65.9(0.6)fb & 19.4 \\ % 1673353.0 \\
& in\texttt{put\_ZH\_scet.int} &      & 67.4(0.6)fb & 19.1 \\[2pt] % 1653967.0 \\
$W^-H+W^+H$
& in\texttt{put\_WH\_qt.int}   & 0.01 & 29.7(0.3)fb & 271 \\ % 23379691 \\
& in\texttt{put\_WH\_scet.int} &      & 29.4(0.3)fb & 281 \\[2pt] % 24241018.\\
$e^-e^+\gamma$
& in\texttt{put\_eexa\_qt.int}   & 0.03 & 133.8(4.0)fb & 183 \\ % 15818271. \\
& in\texttt{put\_eexa\_scet.int} &      & 128.7(3.8)fb & 203 \\[2pt] % 17528784.\\
$e^-\bar\nu_e\gamma$
& in\texttt{put\_enexa\_qt.int}   & 0.01 & 389.8(3.9)fb & 26.1 \\ % 2254963.1 \\
& in\texttt{put\_enexa\_scet.int} &      & 371.7(3.6)fb & 50.0 \\[2pt] % 4320244.9 \\
$\nu_e e^+\gamma$
& in\texttt{put\_neexa\_qt.int}   & 0.01 & 488.0(4.8)fb & 29.1 \\ % 2510408.8 \\
& in\texttt{put\_neexa\_scet.int} &      & 461.7(4.6)fb & 47.1 \\[2pt] % 4071605.7 \\
$\gamma\gamma$
& in\texttt{put\_aa\_qt.int}   & 0.01 & 13.98(0.13)pb & 1.08 \\ % 92901.640 \\
& in\texttt{put\_aa\_scet.int} &      & 14.17(0.14)pb & 0.92 \\[2pt] % 79255.726 \\
$e^-\mu^+\bar\nu_e\nu_\mu$
& in\texttt{put\_emxnexnm\_qt.int}   & 0.015 & 21.2(0.3)fb  & 325 \\ % 28043518 \\
& in\texttt{put\_emxnexnm\_scet.int} &       & 21.6(0.3)fb  & 319 \\[2pt] %  \\
$e^- \bar\nu_e \mu^-\mu^+$
& in\texttt{put\_enexmmx\_qt.int}   & 0.01   & 2.23(0.02)fb & 82.7 \\ % 7142746.4 \\
& in\texttt{put\_enexmmx\_scet.int} &        & 2.25(0.02)fb & 76.6 \\[2pt] %  \\
$\bar\nu e^+ \mu^- \mu^+$
& in\texttt{put\_neexmmx\_qt.int}   & 0.01   & 3.13(0.03)fb & 124 \\ % 10685538 \\
& in\texttt{put\_neexmmx\_scet.int} &        & 3.14(0.03)fb & 121 \\[2pt] % 10487091 \\
$e^- e^+ \mu^- \mu^+$
& in\texttt{put\_eexmmx\_qt.int}   & 0.01    & 2.99(0.03)fb & 54.2 \\ % 4685505.7 \\
& in\texttt{put\_eexmmx\_scet.int} &         & 3.03(0.03)fb & 76.1 \\ % 6572676.0 \\
\bottomrule
\end{tabular}
\caption{Processes that may be computed at NNLO in this version.  Results have been obtained
when requiring the given precision (\texttt{precisiongoal} in the input file)
on the calculation of the NNLO coefficient ($\delta_{NNLO}$).
\label{table:nnlo10}}
\end{table}
\renewcommand{\arraystretch}{1.0}

This version also extends the capabilities of the interface to allow a calculation
of one-loop amplitudes representing diboson+jet production with
a variety of $W$ and $Z$ boson decays, including all appropriate interferences.
The calculation of diboson amplitudes (without the presence of an additional
jet) has also been extended to include additional processes that include
interference contributions.
The new scattering amplitudes available are:
\begin{lstlisting}
d u~ e- ve~ e+ e-
u d~ e+ ve e+ e-
u u~ e- e+ ve ve~

d u~ e- ve~ a g
u d~ e+ ve a g
u u~ e- e+ a g
u u~ e- ve~ mu+ vmu g
d u~ e- ve~ mu+ mu- g
u d~ e+ ve mu+ mu- g
u u~ e- e+ mu+ mu- g
u u~ e- e+ e- e+ g
u u~ e- e+ vmu vmu~ g
u u~ e- e+ ve ve~ g
\end{lstlisting}
where, in addition, all relevant combinations of quark flavors are included.

A description of the new features added in recent releases (v9.0 onwards) is 
given in Appendix~\ref{mcfm9plus}.
