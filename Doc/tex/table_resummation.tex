\begin{longtable}{p{1.5cm}p{12cm}}
		\toprule
		\multicolumn{1}{c}{{\textbf{Section} \texttt{resummation}}} & \multicolumn{1}{c}{{\textbf{Description}}} \\ 
		\midrule
\begin{minipage}[t]{0.24\columnwidth}\raggedright
\texttt{usegrid}\strut
\end{minipage} & \begin{minipage}[t]{0.71\columnwidth}\raggedright
\texttt{.true.} or \texttt{.false.} determines whether pregenerated
LHAPDF interpolation grids should be used for the resummation beam
functions.\strut
\end{minipage}\tabularnewline
\begin{minipage}[t]{0.24\columnwidth}\raggedright
\texttt{makegrid}\strut
\end{minipage} & \begin{minipage}[t]{0.71\columnwidth}\raggedright
If \texttt{.true.}, then MCFM runs in grid generation mode. This
generates LHAPDF grid files in the directory \texttt{gridoutpath} from
LHAPDF grids in the directory \texttt{gridinpath}. After the grid
generation MCFM stops and should be run subsequently with
\texttt{makegrid = .false.} and \texttt{usegrid = .true.}. When
\texttt{lhapdf\%dopdferrors=.true.} then also grids for the error sets
are generated.\strut
\end{minipage}\tabularnewline
\begin{minipage}[t]{0.24\columnwidth}\raggedright
\texttt{gridoutpath}\strut
\end{minipage} & \begin{minipage}[t]{0.71\columnwidth}\raggedright
Output directory for LHAPDF grid files, for example
\texttt{/home/tobias/local/share/LHAPDF/}\strut
\end{minipage}\tabularnewline
\begin{minipage}[t]{0.24\columnwidth}\raggedright
\texttt{gridinpath}\strut
\end{minipage} & \begin{minipage}[t]{0.71\columnwidth}\raggedright
Input directory for LHAPDF grid files, for example
\texttt{/home/tobias/local/share/LHAPDF/}\strut
\end{minipage}\tabularnewline
\begin{minipage}[t]{0.24\columnwidth}\raggedright
\texttt{res\_range}\strut
\end{minipage} & \begin{minipage}[t]{0.71\columnwidth}\raggedright
Integration range of purely resummed part, for example \texttt{0.0 80.0}
for \(q_T\) integration between 0 and 80 GeV.\strut
\end{minipage}\tabularnewline
\begin{minipage}[t]{0.24\columnwidth}\raggedright
\texttt{resexp\_range}\strut
\end{minipage} & \begin{minipage}[t]{0.71\columnwidth}\raggedright
Integration range of fixed-order expanded resummed part, for example
\texttt{1.0 80.0} for \(q_T\) integration between 1 and 80 GeV.\strut
\end{minipage}\tabularnewline
\begin{minipage}[t]{0.24\columnwidth}\raggedright
\texttt{fo\_cutoff}\strut
\end{minipage} & \begin{minipage}[t]{0.71\columnwidth}\raggedright
Lower \(q_T\) cutoff $q_0$ for the fixed-order part. % see eq.~\eqref{eq:matchingmod} below.
Typically the value should agree with the lower range of \texttt{resexp\_range}.\strut
\end{minipage}\tabularnewline
\begin{minipage}[t]{0.24\columnwidth}\raggedright
\texttt{transitionswitch}\strut
\end{minipage} & \begin{minipage}[t]{0.71\columnwidth}\raggedright
Parameter passed to the plotting routine to modify the transition
function, see text.\strut
\end{minipage}\tabularnewline
\bottomrule
\end{longtable}
