	\begin{longtable}{p{1.5cm}p{12cm}}
		\toprule
		\multicolumn{1}{c}{{\textbf{Section} \texttt{scales}}} & \multicolumn{1}{c}{{\textbf{Description}}} \\ 
		\midrule
		\texttt{renscale} &
		This parameter may be used to adjust the value
		of the {\it renormalization} scale. This is the scale
		at which $\alpha_S$ is evaluated and will typically be set to
		a mass scale appropriate to the process ($M_W$, $M_Z$, $M_t$ for
		instance). \\
		\texttt{facscale} &
		This parameter may be used to adjust the value
		of the {\it factorization} scale and will typically be set to
		a mass scale appropriate to the process ($M_W$, $M_Z$, $M_t$ for
		instance). \\
		\texttt{dynamicscale} &
		This character string is used to specify whether
		the renormalization, factorization and fragmentation scales are dynamic, i.e. recalculated
		on an event-by-event basis. If this string is set to `{\tt none}' then the scales
		are fixed for all events at the values	specified by {\tt renscale}, {\tt facscale}
		as well as \texttt{fragmentation\_scale} as defined further below.
		
		The type of dynamic scale to be used is selected by using a particular string
		for the variable {\tt dynamicscale}, as indicated in \cref{tab:dynamicscales} on \cpageref{tab:dynamicscales}.
		Not all scales are defined for each process, with program execution halted if
		an invalid selection is made in the input file.
		The selection chooses a reference scale, $\mu_0$. The actual scales used in
		the code are then,
		\begin{equation}
		\mu_{\mathrm{ren}} = {\tt scale} \times \mu_0 \;, \qquad
		\mu_{\mathrm{fac}} = {\tt facscale} \times \mu_0
		\label{eq:dynscale}
		\end{equation}
		Note that, for simplicity, the fragmentation scale (relevant only for processes
		involving photons) is set equal to the renormalization scale.
		In some cases it is possible for the dynamic scale to become very large. This can cause problems 
		with the interpolation of data tables for the PDFs and fragmentation functions. As a result if a dynamic scale 
		exceeds a maximum of $60$ TeV (PDF) or $990$ GeV (fragmentation) this value is set by default to the maximum. 	
		\\
		\texttt{doscalevar} &
		
		This additional option can be set to \texttt{.true.} to enable scale variation.
		It performs a variation of the scales used in \cref{eq:dynscale} by a factor of 
		two so that it surveys the 
		additional possibilities,
		\begin{eqnarray}
		&&
		(2\mu_{\mathrm{ren}},2\mu_{\mathrm{fac}}),
		(\mu_{\mathrm{ren}}/2,\mu_{\mathrm{fac}}/2), \nonumber \\ &&
		(2\mu_{\mathrm{ren}},\mu_{\mathrm{fac}}),
		(\mu_{\mathrm{ren}}/2,\mu_{\mathrm{fac}}),
		(\mu_{\mathrm{ren}},2\mu_{\mathrm{fac}}),
		(\mu_{\mathrm{ren}},\mu_{\mathrm{fac}}/2) \,.
		\label{eq:scalevar}
		\end{eqnarray}
		The histograms corresponding to these different choices are included in the output file, from which an
		envelope of theoretical uncertainty may be constructed by the user. \\
		\texttt{maxscalevar} &
		Number of additional scale variation points to choose, can be set to two or six. For two
		it just samples the first two variations as in eq.~\ref{eq:scalevar}. \\
		\bottomrule
	\end{longtable}
