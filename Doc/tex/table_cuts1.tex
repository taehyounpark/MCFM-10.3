	\begin{longtable}{p{1.5cm}p{12cm}}
		\toprule
		\multicolumn{1}{c}{{\textbf{Section} \texttt{cuts}}} & \multicolumn{1}{c}{{\textbf{Description}}} \\ 
		\midrule
		\texttt{makecuts} &
		If this parameter is set to {\tt .false.} then
		no additional cuts are applied to the events and the remaining
		parameters in this section are ignored. Otherwise, events will
		be rejected according to a set of cuts that is specified below.
		Further options may be implemented by editing {\tt src/User/gencuts\_user.f90}. \\
		
		{\tt ptleptmin, etaleptmax} & These specify the values
		of $p_{T,{\mathrm{min}}}$ and $|\eta|_{\mathrm{max}}$ for one of the leptons produced
		in the process. One can also introduce optional settings \texttt{ptleptmax}
		and \texttt{etaleptmin}. \\
		
		{\tt etaleptveto} & This should be specified as a pair of double
		precision numbers that indicate a rapidity range that should be excluded
		for the lepton that passes the above cuts. \\
		
		{\tt ptminmiss} & Specifies the minimum missing transverse
		momentum (coming from neutrinos). \\
		
		{\tt ptlept2min}, \texttt{etalept2max} & These specify
		the values of $p_{T,{\mathrm{min}}}$ and $|\eta|_{\mathrm{max}}$ for the remaining
		leptons in the process. This allows for staggered cuts where, for
		instance, only one lepton is required to be hard and central.
		One can also introduce optional settings \texttt{ptlept2max} and
		\texttt{etalept2min}. \\
		
		{\tt etalept2veto} & This should be specified as a pair of double
		precision numbers that indicate a rapidity range that should be excluded
		for the remaining leptons. \\
		
		\bottomrule
	\end{longtable}
%\end{table}
\clearpage
