	\begin{longtable}{p{1.5cm}p{12cm}}
		\toprule
		\multicolumn{1}{c}{{\textbf{Section} \texttt{nnlo}}} & \multicolumn{1}{c}{{\textbf{Description}}} \\ 
		\midrule
		\texttt{taucut} & 
		Optional. This sets the value of the jettiness variable
		$\tau_\text{cut}$, as multiplied by the invariant mass of the Born system,
		that separates the resolved and unresolved regions in \NNLO{}
		calculations that use zero-jettiness. The default value results
		in total inclusive cross sections with less than $1\%$ residual cutoff effects. \\
		\texttt{tcutarray} &
		Optional. Array that specifies multiple taucut values that should be sampled
		on the fly in addition to the nominal taucut value. Both larger and smaller
		values than the nominal one can be specified, although uncertainties for
		smaller values will be large. We generally do not recommend smaller values
		than the nominal one chosen with \texttt{taucut}. Default values are chosen
		to be $2,4,8,20,40$ times the nominal choice of \texttt{taucut}.  \\
		\texttt{dynamictau} &
		Optional. If \texttt{.false.}, the \texttt{taucut} value specified
		is not multiplied by the invariant mass of the Born system. Default is \texttt{.true.}. \\
                \texttt{useqt} & Flag to use $q_T$ slicing, rather than
		0-jettiness, in the calculation of NNLO contributions.
		Default is \texttt{.false.} \\
                \texttt{useGLY} & If \texttt{.true.}, implement non-local $q_T$ subtraction using formulas from \cite{Gehrmann:2014yya}.
		Default is \texttt{.true.}  when \texttt{useqt} is enabled.  If \texttt{.false},
		implement non-local $q_T$ subtraction using formulas from \cite{Billis:2019vxg}. \\
                \texttt{qtcut} & If \texttt{useqt} is enabled, the value of the slicing parameter, defined
		in the same way as \texttt{taucut} described above.  \\
                \texttt{tauboost} & When using 0-jettiness, perform the slicing cut in the
		centre-of-mass of the color singlet system.  Default is \texttt{.true.} \\
                \texttt{incpowcorr} & When using 0-jettiness, include leading power corrections
		in the below-cut calculation. Default is \texttt{.false.} \\
                \texttt{onlypowcorr} & When using 0-jettiness, only compute the power corrections
		to the below-cut calculation. Default is \texttt{.false.} \\
		\bottomrule
	\end{longtable}


taucut
tcutarray
dynamictau
useqt
useGLY
oldqt
qtcut
   
