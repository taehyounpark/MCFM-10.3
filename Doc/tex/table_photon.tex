	\begin{longtable}{p{3.5cm}p{12cm}}
		\toprule
		\multicolumn{1}{c}{{\textbf{Section} \texttt{photon}}} & \multicolumn{1}{c}{{\textbf{Description}}} \\ 
		\midrule
		{\tt fragmentation} &  This parameter is a logical variable that determines whether the production of photons 
		by a parton 
		fragmentation process is included. If {\tt fragmentation} is set to {\tt .true.} the code uses a a standard 
		cone isolation
		procedure (that includes LO fragmentation contributions in the NLO calculation).
		If {\tt fragmentation} is set to {\tt .false.} the code implements
		a Frixione-style photon cut~\cite{Frixione:1998jh},
		\begin{eqnarray}
		\sum_{i \in R_0} E_{T,i}^j  < \epsilon_h E_{T}^{\gamma} \bigg(\frac{1-\cos{R_{i\gamma}}}{1-\cos{R_0}}\bigg)^{n} 
		\;.
		\label{frixeq}
		\end{eqnarray}
		In this equation, $R_0$, $\epsilon_h$ and $n$ are defined by {\tt cone\_ang}, {\tt epsilon\_h} 
		and {\tt n\_pow}  respectively (see below).
		$E_{T,i}^{j}$ is the transverse energy of a parton, $E_{T}^\gamma$ is the
		transverse energy of the photon and $R_{i\gamma}$ is the separation between the photon and the parton using the 
		usual definition 
		$R=\sqrt{\Delta\phi^2+\Delta\eta^2}$. $n$ is an integer parameter which by default is set to~1. \\
		
		{\tt fragmentation\_set} & A length eight character variable that is used to choose the particular photon 
		fragmentation set.
		Currently implemented fragmentation functions can be called with `{\tt BFGSet\_I}', `{\tt 
			BFGSetII}'~\cite{Bourhis:1997yu} or `{\tt GdRG\_\_LO}'~\cite{GehrmannDeRidder:1998ba}. \\
		
		{\tt fragmentation\_scale} & A double precision variable that will be used to choose the scale 
		at which the photon fragmentation is evaluated. \\
		
		{\tt gammptmin} & This specifies the value
		of $p_T^{\mathrm{min}}$ for the photon with the largest transverse momentum.
		Note that this cut, together with all the photon cuts specified in this section
		of the input file, are applied even if {\tt makecuts} is set to {\tt .false.}.
		One can also add an entry for \texttt{gammptmax} to cut on a range. \\
		
		{\tt gammrapmax} & This specifies the value
		of $|y|^{\mathrm{max}}$ for any photons produced in the process. One can also add an entry
		for \texttt{gammrapmin} to cut on a range. \\
		
		{\tt gammpt2} and {\tt gammpt3} & These specify the values
		of $p_T^{\mathrm{min}}$ for the second and third photons, ordered by $p_T$. \\
		
		{\tt Rgalmin} & Using the usual definition of $\Delta R$,
		this requires that all photon-lepton pairs are separated by
		$\Delta R >$~{\tt Rgalmin}. This parameter must be non-zero
		for processes in which photon radiation from leptons is included. \\
		
		{\tt Rgagamin} & Using the usual definition of $\Delta R$,
		this requires that all photon pairs are separated by
		$\Delta R >$~{\tt Rgagamin}. \\
		
		{\tt Rgajetmin} & Using the usual definition of $\Delta R$,
		this requires that all photon-jet pairs are separated by
		$\Delta R >$~{\tt Rgajetmin}. \\
		
		{\tt cone\_ang} & Fixes the cone size ($R_0$) for photon isolation.
		This cone is used in both forms of isolation. \\
		
		{\tt epsilon\_h} & This cut controls the amount of radiation allowed in cone when  {\tt fragmentation} is set 
		to 
		{\tt .true.}. If  {\tt epsilon\_h} $ < 1$ then the photon is isolated using
		$\sum_{\in R_0} E_T{\mathrm{(had)}} < \epsilon_h \, p^{\gamma}_{T}.$ Otherwise {\tt epsilon\_h}  $ > 1$ sets 
		$E_T(max)$ in  $\sum_{\in R_0} E_T{\mathrm{(had)}} < E_T(max)$. \\
		
		{\tt n\_pow} & When using the Frixione isolation prescription, the exponent $n$ in Eq.~(\ref{frixeq}). \\
		
                {\tt fixed\_coneenergy} & This is only operational when using the Frixione isolation prescription.
		If {\tt fixed\_coneenergy} is .false. then $\epsilon_h$ controls the amount of hadronic energy allowed 
		inside the cone using the
		Frixione isolation prescription (see above, Eq.~(\ref{frixeq}))
		If {\tt fixed\_coneenergy} is .true. then this formula
		is replaced by one where $\epsilon_h E_T^\gamma \rightarrow \epsilon_h$. \\		

                {\tt hybrid}, {\tt  R\_inner} & If {\tt hybrid} is set to .true. use a hybrid isolation scheme
		with Frixione isolation on an inner cone of radius {\tt  R\_inner}. \\
    \bottomrule
	\end{longtable}
