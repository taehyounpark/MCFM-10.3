	\begin{longtable}{p{1.5cm}p{12cm}}
		\toprule
		\multicolumn{1}{c}{{\textbf{Section} \texttt{extra}}} & \multicolumn{1}{c}{{\textbf{Description}}} \\ 
		\midrule
		{\tt debug} &
		A logical variable which can be used during a 
		debugging phase to mandate special behaviours. 
		Passed by common block {\tt common/debug/debug}. \\
		
		{\tt verbose} &
		A logical variable which can be used during a debugging phase to write 
		special information. Passed in common block {\tt common/verbose/verbose}. \\
		
		{\tt new\_pspace} &
		A logical variable which can be used during a debugging phase to test alternative versions of the phase space.
		Passed in common block {\tt common/new\_pspace/new\_pspace}. \\
		
		{\tt spira} & 
		A  variable. If {\tt spira} is true, we calculate the 
		width of the Higgs boson by interpolating from a table
		calculated using the NLO code of M. Spira.	Otherwise the LO value valid for low Higgs masses only is used. \\
		
		{\tt noglue} &
		A logical variable. 
		The default value is false. If set to true, no processes
		involving initial gluons are included. \\
		{\tt ggonly} &
		A logical variable. 
		The default value is false. If set to true, 
		only the processes
		involving initial gluons in both hadrons are included.\\
		{\tt gqonly} &
		The default value is false. If set to true, 
		only the processes
		involving an initial gluon in one hadron and an initial quark
		or antiquark in the other hadron (or vice versa) are included.\\
		{\tt omitgg} &
		A logical variable. 
		The default value is false. If set to true, the gluon-gluon
		initial state is not included.\\
		
		{\tt clustering} &
		This logical parameter determines whether clustering is performed to yield
		jets. Only during a debugging phase should this variable be set to false. \\
		
		{\tt colourchoice} &
		If colourchoice=0, all colour structure are included ($W,Z+2$~jets).
		If colourchoice=1, only the leading 
		colour structure is included ($W,Z+2$~jets). \\
		
		{\tt rtsmin} &
		A minimum value of $\sqrt{s_{12}}$, which ensures that the invariant mass
		of the incoming partons can never be less than {\tt rtsmin}. \\
		
		
		{\tt cutoff} & Cutoff according to \texttt{src/Need/smallnew.f} and
		\texttt{src/Need/smalltau.f} \\

                \texttt{reweight} & Flag to set the use of
		the user-implemented reweighting procedure \texttt{reweight\_user}
		in the routine \texttt{src/User/gencuts\_user.f90}.\\
    		\bottomrule
	\end{longtable}
