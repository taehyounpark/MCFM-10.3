	\begin{longtable}{p{1.5cm}p{12cm}}
		\toprule
		\multicolumn{1}{c}{{\textbf{Section} \texttt{basicjets}}} & \multicolumn{1}{c}{{\textbf{Description}}} \\ 
		\midrule
		\texttt{inclusive} &
		This logical parameter chooses whether the
		calculated cross-section should be inclusive in the number of jets
		found at \NLO{}. An {\em exclusive}
		cross-section contains the same number of jets at next-to-leading
		order as at leading order. An {\em inclusive} cross-section may
		instead contain an extra jet at \NLO{}. \\
		\texttt{algorithm} &
		This specifies the jet-finding algorithm that
		is used, and can take the values
		{\tt ktal} (for the Run II $k_T$-algorithm), {\tt ankt} (for the
		``anti-$k_T$'' algorithm~\cite{Cacciari:2008gp}), {\tt cone} (for
		a midpoint cone algorithm), {\tt hqrk} (for a simplified cone
		algorithm designed for heavy quark processes) and {\tt none} (to
		specify no jet clustering at all). The latter option is only a
		sensible choice when the leading order cross-section is well-defined
		without any jet definition: e.g. the single top process,
		$q{\bar q^\prime} \to t{\bar b}$, which is finite as
		$p_T({\bar b}) \to 0$. \\
		\texttt{ptjetmin}, \texttt{etajetmax} &
		These specify the values
		of $p_{T,{\mathrm{min}}}$ and $|\eta|_{\mathrm{max}}$ for the
		jets that are found by the algorithm.  \\
		\texttt{etajetmin} &
		Optional parameter for setting a minimum jet rapidity $|\eta|_{\mathrm{min}}$. \\
		\texttt{ptjetmax} &
		Optional parameter for setting maximum jet $p_{T,{\mathrm{min}}}$\\
		\texttt{Rcutjet} &
		If the final state of the chosen process contains
		either quarks or gluons then for each event an attempt will be made
		to form them into jets. For this it is necessary to define the
		jet separation $\Delta R=\sqrt{{\Delta \eta}^2 + {\Delta \phi}^2}$
		so that after jet combination, all jet pairs are separated by
		$\Delta R >$~{\tt Rcutjet}.\\
		\bottomrule
	\end{longtable}
