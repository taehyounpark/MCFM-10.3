% !TeX encoding = UTF-8
% !TeX spellcheck = en_US
%\pdfoutput=1
\documentclass[letterpaper,parskip=half]{scrartcl}

\makeatletter
\DeclareOldFontCommand{\rm}{\normalfont\rmfamily}{\mathrm}
\DeclareOldFontCommand{\sf}{\normalfont\sffamily}{\mathsf}
\DeclareOldFontCommand{\tt}{\normalfont\ttfamily}{\texttt}
\DeclareOldFontCommand{\bf}{\normalfont\bfseries}{\mathbf}
\DeclareOldFontCommand{\it}{\normalfont\itshape}{\mathit}
\DeclareOldFontCommand{\sl}{\normalfont\slshape}{\@nomath\sl}
\DeclareOldFontCommand{\sc}{\normalfont\scshape}{\@nomath\sc}
\makeatother

\usepackage[top=2.5cm, bottom=2.5cm, left=3.2cm, right=3cm]{geometry}

\usepackage[T1]{fontenc}
\usepackage[utf8]{inputenc}

%\usepackage[numbers,sort&compress]{natbib}
\usepackage[sorting=none,style=numeric-comp,backend=biber,eprint=true,url=true]{biblatex}
%\DeclareFieldFormat[article]{title}{}
\renewbibmacro{in:}{}
\addbibresource{manual.bib}


\usepackage{listings}
\usepackage{color}

\definecolor{dkgreen}{rgb}{0,0.6,0}
\definecolor{gray}{rgb}{0.5,0.5,0.5}
\definecolor{mauve}{rgb}{0.58,0,0.82}

\usepackage[colorlinks=true,urlcolor=blue,linkcolor=blue,citecolor=red]{hyperref}
%\usepackage[separate-uncertainty = true,multi-part-units=single]{siunitx}
%\usepackage{microtype}
%\usepackage{csquotes}
\usepackage{graphicx}
\usepackage{amsmath}
\usepackage{amssymb}
\usepackage{braket}
\usepackage{slashed}
\usepackage{longtable}
\usepackage{subcaption}
\usepackage{multirow}

% !TeX encoding = UTF-8
% !TeX spellcheck = en_US
%\pdfoutput=1
\input macro.tex

%Needed to allow pdf article and html book
\def\topheading{\section}
\def\midheading{\subsection}
\def\bottomheading{\subsubsection}

% Attempt to add consistent spacing for section names in TOC
\RedeclareSectionCommand[tocnumwidth=1.6em]{section}
\RedeclareSectionCommand[tocindent=2.1em,tocnumwidth=2.7em]{subsection}
\RedeclareSectionCommand[tocindent=4.1em,tocnumwidth=3.7em]{subsubsection}

\begin{document}
\begin{titlepage}
	\centering
	{\huge\bfseries MCFM-10.3 manual\par\vspace{0.3em}}
	{\large\bfseries A Monte Carlo for FeMtobarn processes at Hadron Colliders}
	
	John M. Campbell ({\tt johnmc@fnal.gov}) \\
	R. Keith Ellis ({\tt ellis@fnal.gov}) \\
	Tobias Neumann ({\tt tneumann@fnal.gov}) \\
	Ciaran Williams ({\tt ciaranwi@buffalo.edu}) \\	
	
	{\par\vspace{1em} Updated January 2023}
        \abstract{
\noindent
MCFM is a parton-level Monte Carlo program that gives predictions for a
wide range of processes at hadron colliders. Almost all processes are
available at NLO, but some processes are also available at NNLO
or N$^3$LO in QCD. The calculation of some processes can also account for
NLO electroweak effects. Transverse momentum and jet veto resummation
is available for the production of color singlet final states.
Please look at the {\href{\mcfmprocs/proclist.html}{list of available processes}}.
This document is available as a {\href{\mcfmweb}{series of webpages}} and as a {\href{\mcfmpdf}{pdf file}}.
Download and installation instructions are in Section~\ref{overview}.
}
\tableofcontents
\end{titlepage}
\tableofcontents
\input body.tex
\printbibliography
\end{document}
