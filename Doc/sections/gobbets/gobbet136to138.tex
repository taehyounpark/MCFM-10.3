\midheading{$H+b$ production, processes 136--138}
\label{subsec:Hb}

These processes represent the production of a Standard Model Higgs
boson that decays into a pair of bottom quarks,
in association with a further bottom quark. The initial state at lowest order
is a bottom quark and a gluon.
The calculation may be performed at NLO, although radiation from the
bottom quarks in the Higgs decay is not included.
For more details on this calculation, please see Ref.~\cite{Campbell:2002zm}.

For this process, the matrix elements are divided up into a number of
different sub-processes, so the user must sum over these after performing
more runs than usual. At lowest order one can proceed as normal, using
{\tt nproc=136}. For a NLO calculation, the sequence of runs is as follows:
\begin{itemize}
\item Run {\tt nproc=136} with {\tt part=virt} and {\tt part=real} (or, both
at the same time using {\tt part=tota});
\item Run {\tt nproc=137} with {\tt part=real}.
\end{itemize}
The sum of these yields the cross-section with one identified $b$-quark in
the final state. To calculate the contribution with two $b$-quarks in the
final state, one should use {\tt nproc=138} with {\tt part=real}.

When {\tt removebr} is {\tt .true.}, the Higgs boson does not decay.
