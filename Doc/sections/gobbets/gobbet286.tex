\midheading{$\gamma\gamma+$jet production, process 286}

Process {\tt 286}, corresponding to the production of a pair of real photons in association with a jet, can be computed at NLO.   
Since this process includes two real photons, the cross section diverges
when one of the photons is very soft or in the direction of the beam.
Thus in order to produce sensible results, the input file must supply values for both
{\tt gammptmin} and {\tt gammrapmax}. This will ensure that
the cross section is well-defined.

The calculation of process {\tt 286} may be performed at NLO using either the
Frixione algorithm~\cite{Frixione:1998jh} or standard cone isolation.  This process also includes
the one-loop gluon-gluon contribution as given in
ref.~\cite{Bern:2002jx}.  The production of a photon via parton fragmentation is included at NLO and
can be run separately by using the {\tt frag} option in {\tt part}. This option includes the contributions from the
integrated
photon dipole subtraction terms and the LO QCD matrix element multiplied by the fragmentation function.

The phase space cuts for the final state photons are defined in {\tt{input.ini}}, for multiple photon processes such
as {\tt 285 - 287} the $p_T$'s of the individual photons (hardest, second hardest and third hardest or softer) can be
controlled independently.
The remaining cuts on $R_{\gamma j}$, $\eta_{\gamma}$ etc. are applied universally to all photons. Users wishing to
alter
this feature should edit the file {\tt{photon\_cuts.f}} in the directory {\tt{src/User}}.


The calculation is described in Ref.~\cite{Campbell:2014yka}.
