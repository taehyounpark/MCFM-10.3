\midheading{$t\bar{t}$ production with 2 semi-leptonic decays, processes 141--145}
\label{subsec:ttbar}

These processes describe $t \bar{t}$ production including semi-leptonic
decays for both the top and the anti-top.
Since version 6.2 we have updated this to use the one-loop amplitudes of
ref.~\cite{Badger:2011yu}. The code for the virtual amplitudes now runs
about three times faster than earlier versions where the virtual
amplitudes of ref.~\cite{Korner:2002hy} were used.
Switching {\tt zerowidth} from {\tt .true.} to {\tt .false.} only affects
the $W$ bosons from the top quark decay, because our method of including spin
correlations requires the top quark to be on shell.

Process {\tt 141} includes all corrections, i.e.\ both radiative corrections
to the decay and to the production. This process is therefore the
basic process for the description of top production where both quarks
decay semi-leptonically.  When {\tt removebr} is true in process {\tt 141},
the top quarks do not decay.
When one wishes to calculate observables related to the decay of the top
quark, {\tt removebr} should be false in process {\tt 141}.
The LO calculation proceeds as normal.
At NLO, there are two options:
\begin{itemize}
\item {\tt part=virt, real} or {\tt tota} : final state radiation is included
in the production stage only
\item {\tt part = todk} : radiation is included in the decay of the top
quark also and the final result corresponds to the sum of real and virtual
diagrams.
Note that these runs automatically perform an extra integration, so
will take a little longer.
\end{itemize}

Process {\tt 142} includes only the corrections in
the semileptonic decay of the top quark. Thus it is of primary
interest for theoretical studies rather than for physics applications.
Because of the method that we have used to include the radiation in the decay,
the inclusion of the corrections in the decay does not change the
total cross section. This feature is explained in section 6 of ref.~\cite{Campbell:2012uf}.

In the case of process {\tt 145}, there are no spin correlations in
the decay of the top quarks. The calculation is performed by
multiplying the spin summed top production cross section, by the decay
matrix element for the decay of the $t$ and the $\bar{t}$. These
processes may be used as a diagnostic test for the importance of the
spin correlation.
