\midheading{$\gamma\gamma\gamma\gamma$ production, process 289}
\label{subsec:fourgam}

Process {\tt 289} represents the production of four real photons.
The cross section diverges
when one of the photons is very soft or in the direction of the beam.
Thus in order to produce sensible results, the input file must supply values for both
{\tt gammptmin} and {\tt gammrapmax}. This will ensure that
the cross section is well-defined.

The calculation of process {\tt 289} may be performed at NLO using either the
Frixione algorithm~\cite{Frixione:1998jh} or standard cone isolation.  The production of a photon via parton
fragmentation is included at NLO and
can be run separately by using the {\tt frag} option in {\tt part}. This option includes the contributions from the
integrated
photon dipole subtraction terms and the LO QCD matrix element multiplied by the fragmentation function.
The phase space cuts for the final state photons are defined in {\tt{input.ini}}, for multiple photon processes such
as {\tt 285 - 289} the $p_T$'s of the individual photons (hardest, next-to hardest and softest) can be controlled
independently.
The remaining cut on $R_{\gamma j}$, $\eta_{\gamma}$ etc. are applied universally to all photons. Users wishing to alter
this feature should edit the file {\tt{photon\_cuts.f}} in the directory {\tt{src/User}}.

Note that for this process the second softest and softest photons are forced to have equal minimum $p_T$, defined
by the {\tt{[gammptmin(3rd)]}} variable in the input file.
