\midheading{Higgs production, ($m_t=\infty$), processes 111--121}
\label{subsec:h}

This process is calculable at leading LO, 
at next-to-leading order NLO, and at next-to-next-to-leading order NNLO. 

These processes represent the production of a Standard Model Higgs
boson that decays either into a bottom quark
pair ({\tt nproc=111}), a pair of tau's ({\tt nproc=112}), 
a $W^+W^-$ 
pair that further decays leptonically ({\tt nproc=113}) 
a $W^+W^-$ pair where the $W^-$ decays hadronically ({\tt nproc=114,115}) 
or a $ZZ$ pair ({\tt nproc=116-118}) . In addition, the loop-level decays of the Higgs 
into a pair of photons ({\tt nproc=119}) and the $Z\gamma$ decay are included
({\tt nproc=120,121}).

For the case of $W^+W^-$ process {\tt nproc=115} gives the contribution 
of radiation from the hadronically decaying $W^-$.
Process {\tt 114} may be run at NLO with the option {\tt todk},
including radiation in the decay of the hadronically decaying $W^-$.~\footnote{
We have not included the case of a hadronically decaying $W^+$; it can
be obtained from processes {\tt nproc=114,115} by performing the
substitutions $\nu \to e^-$ and $e^+ \to \bar{\nu}$.}
For the case of a $ZZ$ decay,
the subsequent decays can either be into a pair of muons and a pair of electrons
({\tt nproc=116)}, a pair of muons and neutrinos ({\tt nproc=117}) or
a pair of muons and a pair of bottom quarks ({\tt nproc=118}).

At LO the relevant diagram
is the coupling of two gluons to the Higgs via a top quark loop.
This calculation is performed in the limit of infinite top quark mass, so that 
the top quark loop is replaced by an effective operator. This corresponds
to the effective Lagrangian,
\begin{equation}
\mathcal{L} = \frac{1}{12\pi v} \, G^a_{\mu\nu} G^{\mu\nu}_a H \;,
\label{eq:HeffL}
\end{equation}
where $v$ is the Higgs vacuum expectation value and $G^a_{\mu\nu}$ the
gluon field strength tensor.
The calculation may be performed at NLO, although radiation from the
bottom quarks in the decay of processes {\tt 111} and {\tt 118} is not yet included.

%At the end of the output the program will also display the cross section rescaled
%by the constant factor,
%\begin{equation}
%\frac{\sigma_{\rm LO}(gg \to H, \mbox{finite}~m_t)}{\sigma_{\rm LO}(gg \to H, m_t \to \infty)} \;.
%\label{eqn:hrescale}
%\end{equation}
%For the LO calculation this gives the exact result when retaining a finite value for $m_t$,
%but this is only an approximation at NLO. The output histograms are not rescaled in this way.

When {\tt removebr} is true in processes {\tt 111,112,113,118},
the Higgs boson does not decay.

Process {\tt 119} implements the decay of the Higgs boson into two photons
via loops of top quarks and $W$-bosons.
The decay is implemented using the formula Eq.(11.12) from ref.~\cite{Ellis:1991qj}.
When {\tt removebr} is true in process {\tt 119} the Higgs boson does not decay.

Processes {\tt 120} and {\tt 121} implement the decay of the Higgs boson into an lepton-antilepton
pair and a photon. As usual the production of a charged lepton-antilepton pair is mediated by a 
$Z/\gamma^*$ (process {\tt 120}) and the production of three types of neutrinos 
$\sum  \nu \bar{\nu}$ by a $Z$-boson (process {\tt 121}). These processes are implemented 
using a generalization of the formula of \cite{Djouadi:1996yq}. (Generalization to take into
account off-shell $Z$-boson and adjustment of the sign of $C_2$ in their Eq.(4)).
