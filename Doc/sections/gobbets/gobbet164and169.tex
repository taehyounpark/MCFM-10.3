\midheading{Off-shell single top production in SM and SMEFT, processes 164,169}
\label{subsec:offstop}

The processes 164 and 169 represent off-shell single-top-quark and anti-top-quark production, respectively.
The calculations are performed in the complex-mass scheme.
Both the SM and contributions from the SMEFT can be calculated.
For more details on this calculation, please refer to ref.~\cite{Neumann:2019kvk}.

Dynamical double deep inelastic scattering scales can be
consistently used at NLO by setting \texttt{dynamicscale} to `DDIS'
and \texttt{scale}$=$\texttt{facscale} to 1d0. In this way the
momentum transfer along the $W$-boson $Q^2$ is used as the scale for
the light-quark-line corrections $\mu^2=Q^2$, and $\mu^2=Q^2+m_t^2$ for
the heavy-quark-line corrections. These scales are also consistently
used for the non-resonant contributions, with QCD corrections on the
$ud$-quark line, and separate QCD corrections on the bottom-quark
line.

The new block `Single top SMEFT, nproc=164,169' in the input
file governs the inclusion of SMEFT operators and corresponding
orders.  The scale of new physics $\Lambda$ can be separately set, and
has a default value of $1000$~GeV.  The flag \texttt{enable
	1/lambda4} enables the $1/\Lambda^4$ contributions, where operators
$\Qtwo, \Qfour, \Qseven$ and $\Qnine$ can contribute for the first
time.  For the non-Hermitian operators we allow complex Wilson
coefficients.  We also have a flag to disable the pure SM
contribution, leaving only contributions from SMEFT operators
either interfered with the SM amplitudes or as squared
contributions at $1/\Lambda^4$.  This can be used to directly and
quickly extract kinematical distributions and the magnitudes of
pure SMEFT contributions.

To allow for easier comparisons with previous anomalous couplings
results, and possibly estimate further higher order effects, we allow
for an anomalous couplings mode at LO by enabling the corresponding
flag.  The relations between our operators and the anomalous couplings
are

\begin{align*}
	 \delta V_L &= \Cone \frac{m_t^2}{\Lambda^2} ,\,\text{where } V_L = 1 + \delta V_L\,,\\
	 V_R &= \Ctwo{}^* \frac{m_t^2}{\Lambda^2}\,, \\
	 g_L &= -4\frac{m_W m_t}{\Lambda^2} \cdot \Cfour\,, \\
	 g_R &= -4 \frac{m_W m_t}{\Lambda^2} \cdot \Cthree{}^*\,,
\end{align*}

where $m_W$ is the $W$-boson mass, and $m_W = \frac{1}{2} g_W v$ has
been used to derive this equivalence.  Note that the minus sign for
$g_L$ and $g_R$ is different from the literature. See also the publication~~\cite{Neumann:2019kvk} for more information.

For comparisons with on-shell results one needs to add up the contributions
from processes 161 at NLO and from the virt and real contributions from 162, see above.

The analysis/plotting routine is contained in the file
`\texttt{src/User/nplotter\_ktopanom.f}', where all observables
presented in this study are implemented, and the $W$-boson/neutrino
reconstruction is implemented and can be switched on or off.

