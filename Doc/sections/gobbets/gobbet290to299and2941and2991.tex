\midheading{$W\gamma$ production, processes 290-299, 2941, 2991}
\label{subsec:wgamma}

These processes represent the production of a $W$ boson which subsequently
decays leptonically, in association with a real photon.
Since this process includes a real photon, the cross section diverges
when the photon is very soft or in the direction of the beam.
Thus in order to produce sensible results, the input file must supply values for both
{\tt gammptmin} and {\tt gammrapmax}. Moreover, when the parameters {\tt zerowidth}
and {\tt removebr} are set to {\tt .false.} the decay $W \to \ell \nu$ will include
photon radiation from the lepton, so that a non-zero $R(\gamma,\ell)_{min}$
({\tt Rgalmin}) should
also be supplied. This will ensure that the cross section is well-defined.
Virtual amplitudes are taken from ref.~\cite{Dixon:1998py}.

The calculation of processes {\tt 290} and {\tt 295} may be performed
at NLO and NNLO using the Frixione algorithm~\cite{Frixione:1998jh} or standard isolation. 
The phenomenology of these processes at NNLO has been treated in ref.~\cite{Campbell:2021mlr}.
The one-loop virtual diagrams are taken from \cite{Dixon:1998py} and the two-loop virtual diagrams
are taken from \cite{Gehrmann:2011ab}.
For processes {\tt 290} and {\tt 295} the role of {\tt mtrans34cut} changes to become a cut 
on the transverse mass on the $M_{345}$ system, i.e. the photon is included with the leptons in the cut. 

Processes {\tt 292} and {\tt 297} represent the $W\gamma$+jet production
processes.  They may be computed to NLO. 

Processes {\tt 294} and {\tt 299} represent the photon-induced reactions,
$p + \gamma \to e \nu \gamma j$ and should be computed at NLO. 
Processes {\tt 2941} and {\tt 2991} represent the photon-induced reactions,
$p + \gamma \to e \nu \gamma j j$ and should be computed at NLO. 

\bottomheading{Anomalous $WW\gamma$ couplings}
Processes {\tt 290}-{\tt 297} may also be computed including the effect of anomalous $WW\gamma$ couplings, making
use of the amplitudes calculated in Ref.~\cite{DeFlorian:2000sg}. Including only dimension 6 operators
or less and demanding gauge, $C$ and $CP$ invariance gives the general form of the anomalous 
vertex~\cite{DeFlorian:2000sg},
\begin{eqnarray}
 \Gamma^{\alpha \beta \mu}_{W W \gamma}(q, \bar q, p) &=& 
  {\bar q}^\alpha g^{\beta \mu} 
    \biggl( 2 + \Delta\kappa^\gamma + \lambda^\gamma {q^2\over M_W^2} \biggr) 
 - q^\beta g^{\alpha \mu}
    \biggl( 2 + \Delta\kappa^\gamma + \lambda^\gamma {{\bar q}^2\over M_W^2}
\biggr) \nonumber \\  
&& \hskip 1 cm
 + \bigl( {\bar q}^\mu - q^\mu \bigr) 
 \Biggl[ - g^{\alpha \beta} \biggl( 
   1 + {1\over2} p^2 \frac{\lambda^\gamma}{M_W^2} \biggr) 
 +\frac{\lambda^\gamma}{M_W^2} p^\alpha p^\beta \Biggr] \,,
\end{eqnarray}
where the overall coupling has been chosen to be $-|e|$. The parameters that
specify the anomalous couplings, $\Delta\kappa^\gamma$ and $\lambda^\gamma$, are
specified in the input file as already discussed in Section~\ref{subsec:diboson}.
If the input file contains a negative value for the form-factor scale $\Lambda$
then no suppression factors are applied to the anomalous couplings.
Otherwise, the couplings are included
in MCFM only after suppression by dipole form factors,
\begin{equation}
\Delta \kappa^{\gamma} \rightarrow
 \frac{\Delta \kappa_1^{\gamma}}{(1+\hat{s}/\Lambda^2)^2}, \qquad
\lambda^{\gamma} \rightarrow
 \frac{\Delta \lambda^{\gamma}}{(1+\hat{s}/\Lambda^2)^2} \;,
\end{equation}
where $\hat{s}$ is the $W\gamma$ pair invariant mass.

The Standard Model cross section is obtained by setting $\Delta\kappa^\gamma = \lambda^\gamma = 0$.

