\midheading{$Z+Q$ production, processes 261--267}
\label{subsec:ZQ}

These processes represent the production of a $Z$
boson that decays into a pair of electrons,
in association with a heavy quark, $Q$.
For more details on this calculation, please see Ref.~\cite{Campbell:2003dd}.

For processes {\tt 261}, {\tt 262}, {\tt 266} and {\tt 267} the initial
state at lowest order is the heavy quark and a gluon and
the calculation may be performed at NLO.
As for $H+b$ production, the matrix elements are divided into two
sub-processes at NLO. Thus the user must sum over these after performing
more runs than usual. At lowest order one can proceed as normal, using
{\tt nproc=261} (for $Z+b$) or {\tt nproc=262} (for $Z+c$).
For a NLO calculation, the sequence of runs is as follows:
\begin{itemize}
\item Run {\tt nproc=261} (or {\tt 262}) with {\tt part=virt} and
{\tt part=real} (or, both at the same time using {\tt part=tota});
\item Run {\tt nproc=266} (or {\tt 267}) with {\tt part=real}.
\end{itemize}
The sum of these yields the cross-section with one identified heavy quark in
the final state when {\tt inclusive} is set to {\tt .false.} . To calculate the
rate for at least one heavy quark, {\tt inclusive} should be {\tt .true.}.

For processes {\tt 263} and
{\tt 264}, the calculation uses the matrix elements for the production
of a $Z$ and a heavy quark pair and demands that one of the heavy quarks
is not observed. It may either lie outside the range of $p_T$ and $\eta$
required for a jet, or both quarks may be contained in the same jet.
Due to the extra complexity (the calculation must retain the full
dependence on the heavy quark mass), this can only be computed at LO.

When {\tt removebr} is true, the $Z$ boson does not decay.
