\midheading{Dark Matter Processes  Mono-jet and Mono-photon 800-848} 

\textbf{This process is currently only officially supported with version 8.0 and earlier, use at your own risk!}

This section provides an overview of the Dark Matter (DM) processes
available in MCFM. For more details on this calculation, please see Ref.~\cite{Fox:2012ru}.
Since these processes are quite different in the
range of possible input parameters (reflecting the range of potential
BSM operators) the majority of the new features are controlled by the
file {\tt dm\_parameters.DAT} located in the {\tt Bin} directory.  We
begin this section by describing the inputs in this file.  Note that
these processes are still controlled, as usual by {\tt input.ini}
which is responsible for selecting the process, order in perturbation
theory, PDFs and phase space cuts etc. The new file controls only the
new BSM parameters in the code.

\begin{itemize} 
\item 
{\tt [dm mass]} This parameter sets the mass of the dark matter particle $m_{\chi}$. 
\item 
{\tt [Lambda]} Controls the mass scale associated with the suppression of the higher dimensional operator in the 
effective theory approach. Note that each 
operator has a well defined scaling with respect to Lambda, so cross sections and distributions obtained with one 
particular value can be readily extended to 
determine those with different $\Lambda$. 
\item
{\tt [effective theory] } Is a logical variable which controls whether or not the effective field theory is used in the 
calculation of the DM process. If this value is set to 
{\tt .false.} then one must specify the mass of the light mediator and its width (see below for more details).
\item
{\tt [Yukawa Scalar couplings]} Is a logical variable which determines if the scalar and pseudo scalar operators scale 
with the factor $m_{q}/\Lambda$ ( {\tt. .true.}) 
or 1  ({\tt .false.}).  
\item
{ \tt [Left handed DM couplings] } and { \tt [Right handed DM couplings] } 
These variables determine the coupling of the
various flavours of quarks to the DM operators.  The default value is 1. 
Note that the code uses the fact that vector operators scale as
$(L+R)$ and axial operators scale as $(L-R)$ in constructing cross
sections. Therefore you should be careful if modifying these
parameters. For the axial and pseudo scalar operators the code will
set the right-handed couplings to be the negative of the left handed
input couplings (if this is not already the case from the setup) and
inform the user it has done so. The most likely reason to want to
change these values is to inspect individual flavour operators
separately, i.e.\ to investigate an operator which only couples to up
quarks one would set all couplings to 0d0 apart from {\tt [up type]}
which would be left as 1d0.
\item 
{\tt [mediator mass]} If {\tt [effective theory]} is set to {\tt .false.} this variable controls the mass of the 
mediating particle.
\item 
{\tt [mediator width]} If {\tt [effective theory]} is set to {\tt .false.} this variable controls the width of the 
mediating particle 
\item 
{\tt [g\_x]} If {\tt [effective theory]} is set to {\tt .false.} this variable controls the coupling of the mediating 
particle to the DM.
\item 
{\tt [g\_q]} If {\tt [effective theory]} is set to {\tt .false.} this variable controls the coupling of the mediating 
particle to the quarks.
\end{itemize}

