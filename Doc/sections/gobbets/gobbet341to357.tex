\midheading{$Z+Q+$jet production, processes 341--357}
\label{subsec:ZQj}

These processes represent the production of a $Z$
boson that decays into a pair of electrons,
in association with a heavy quark, $Q$ and an untagged jet.
For more details on this calculation, please see Ref.~\cite{Campbell:2005zv}.

For processes {\tt 341} and {\tt 351} the initial state at lowest
order is the heavy quark and a gluon and the calculation may be
performed at NLO.  Thus in these processes the quark $Q$ occurs as
parton PDF in the initial state.  As for $H+b$ and $Z+Q$ production,
the matrix elements are divided into two sub-processes at NLO. Thus
the user must sum over these after performing more runs than usual. At
lowest order one can proceed as normal, using {\tt nproc=341} (for
$Zbj$) or {\tt nproc=351} (for $Zcj$).  For a NLO calculation, the
sequence of runs is as follows:
\begin{itemize}
\item Run {\tt nproc=341} (or {\tt 351}) with {\tt part=virt} and
{\tt part=real} (or, both at the same time using {\tt part=tota});
\item Run {\tt nproc=342} (or {\tt 352}) with {\tt part=real}.
\end{itemize}
The sum of these yields the cross-section with one identified heavy
quark and one untagged jet in the final state when {\tt inclusive} is
set to {\tt .false.} . To calculate the rate for at least one heavy
quark and one jet (the remaining jet may be a heavy quark, or
untagged), {\tt inclusive} should be {\tt .true.}.

Processes {\tt 346,347} and {\tt 356,357} are the lowest order processes that enter
the above calculation in the real contribution. They can be computed only at LO.

When {\tt removebr} is true, the $Z$ boson does not decay.
