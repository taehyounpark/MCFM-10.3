\midheading{$H \to ZZ \to e^- e^+ \mu^- \mu^+$ production, ($m_t=$~finite), processes 128-133}
These processes represent the production of a Higgs boson that decays to $Z Z$,
with subsequent decay into charged leptons. For process {\tt 128}, the exact form of the triangle
loop coupling a Higgs boson to two gluons is included, with both top and bottom quarks
circulating in the loop. This is to be contrasted with process {\tt 116} in which only the
top quark contribution is included in the effective coupling approach.

Process {\tt 129} includes only the effect of the interference of the
Higgs and $gg \to ZZ$ amplitudes.
The calculation is available at LO only. LO corresponds to $O(\alpha_s^2)$ in this case.
The calculation of loops containing the third quark generation
includes the effect of both the top quark mass and the bottom quark, while the first two
generations are considered massless. For numerical stability, a small cut on the
transverse momentum of the $Z$ bosons is applied: $p_T(Z)>0.05$~GeV.
This typically removes less than $0.1$\% of the cross section. The
values of these cutoffs can be changed by editing \verb|src/ZZ/getggZZamps.f| and recompiling.

Process {\tt 130} includes all $gg$-initiated diagrams that have a Higgs boson in the $s$-channel,
namely the square of the $s$-channel Higgs boson production and the interference with the diagrams
that do not contain a Higgs boson, (i.e. $gg \to Z/\gamma^*+Z/\gamma^* \to e^- e^+ \mu^- \mu^+$),
i.e.~$|M_H|^2+2 |M_H^* M_{ZZ}|$.

Process {\tt 131} calculates the full result for this process from  $gg$-intitiated diagrams.
This includes diagrams that have a Higgs boson in the $s$-channel, the continuum $ Z/\gamma^*+Z/\gamma^*$
diagrams described above and their interference, i.e.~$|M_H+M_{ZZ}|^2$.


Process {\tt 132}  gives the result for the square of the box diagrams alone, i.e. the process
$gg \to Z/\gamma^*+Z/\gamma^* \to e^- e^+ \mu^- \mu^+$, i.e.~$|M_{ZZ}|^2$.

Process {\tt 133} calculates the interference for the $qg$ initiated process.

For those processes that include contributions from the Higgs boson, the form
of the Higgs propagator may be changed by editing the file
{\tt src/Need/sethparams.f}.  If the logical variable {\tt CPscheme} is
changed from the default value {\tt .false.} to {\tt .true.} then the
Higgs propagator is computed using the ``bar-scheme'' that is
implemented in the HTO code of G. Passarino~\cite{Goria:2011wa,Passarino:2010qk}.
The value of the Higgs boson width has been computed with v1.1 of the
HTO code, for Higgs masses in the interval $50 < m_H< 1500$~GeV.  These
values are tabulated, in $0.5$~GeV increments, in the file
{\tt Bin/hto\_output.dat}.  The widths for other masses in this range
are obtained by linear interpolation.

% \bottomheading{Specifying other final states}
% \label{specifyingZdecays}
% As described above, these processes refer to a final state 
% $e^- e^+ \mu^- \mu^+$.  It is however possible to specify a final state
% that corresponds to a different set of $Z$ boson decays.  This is achieved
% by altering the value of {\tt NPROC} in the input file by appending a
% period, followed by two 2-character strings that identify each of the decays.
% Possible values for the strings, and the corresponding decays, are
% shown in the table below.
% \begin{center}
% \begin{tabular}{ll}
% string & $Z$ decay \\
% \hline
% {\tt el,EL} & $(e^-,e^+)$ \\
% {\tt mu,MU,ml,ML} & $(\mu^-,\mu^+)$ \\
% {\tt tl,TL} & $(\tau^-,\tau^+)$ \\
% {\tt nu,NU,nl,NL} & $(\nu,\bar\nu) \times 3$ \\
% {\tt bq,BQ} & $(b,\bar b)$ \\
% \end{tabular}
% \end{center}
% Note that, for the case of neutrino decays, the sum over three flavours of
% neutrino is performed.  The labelling of the particles in the output is best
% understood by example.  Setting {\tt nproc=132.ELNU} corresponds to the
% process $gg \to Z/\gamma^*+Z/\gamma^* \to e^-(p_3) e^+(p_4) \nu(p_5) \bar\nu(p_6)$.
% Note that the default process corresponds to the string {\tt ELMU} so that,
% for instance {\tt nproc=132.ELMU} is entirely equivalent to
% {\tt nproc=132}.
% The effect of changing the lepton flavour is only seen in the output
% of LHE events, where the correct mass is then used when producing the
% event record.

