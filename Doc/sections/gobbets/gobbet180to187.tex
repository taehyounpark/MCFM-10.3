\midheading{$Wt$ production, processes 180--187}
\label{subsec:wt}

These processes represent the production of a $W$ boson that decays leptonically
in association with a top quark. The lowest order diagram involves a gluon and
a bottom quark from the PDF, with the $b$-quark radiating a $W$ boson and
becoming a top quark. The calculation can be performed up to NLO.
For more details on this calculation, please see Ref.~\cite{Campbell:2005bb}.

Processes {\tt 180} and {\tt 185} produce a top quark that does not decay,
whilst in processes {\tt 181} and {\tt 186} the top quark decays leptonically.
Consistency with
the simpler processes ({\tt 180,185}) can be demonstrated by running process
{\tt 181,186} with {\tt removebr} set to true.

At next-to-leading order, the calculation includes contributions from diagrams
with two gluons in the initial state, $gg \rightarrow Wtb$. The $p_T$ of the
additional $b$ quark is vetoed according to the value of the parameter
{\tt ptmin\_bjet} which is specified in the input file. The contribution from
these diagrams when the $p_T$ of the $b$ quark is above {\tt ptmin\_bjet}
is zero. The values of this parameter and the factorization scale ({\tt facscale})
set in the input file should be chosen carefully. Appropriate values for both
(in the range $30$-$100$~GeV) are discussed in the associated paper~\cite{Campbell:2005bb}.

When one wishes to calculate observables related to the decay of the top
quark, {\tt removebr} should be false.
The LO calculation proceeds as normal. At NLO, there are two options:
\begin{itemize}
\item {\tt part=virt, real} or {\tt tota} : final state radiation is included
in the production stage only
\item {\tt part = todk} : radiation is included in the decay of the top
quark also and the final result corresponds to the sum of real and virtual
diagrams. This process can only be performed at NLO with
{\tt zerowidth = .true}. This should be set automatically.
Note that these runs automatically perform an extra integration, so
will take a little longer.
\end{itemize}

The contribution from radiation in the decay may be calculated separately using
processes {\tt 182,187}. These process numbers can be used with {\tt part=virt,real}
only. To ensure consistency, it is far simpler to use {\tt 181,186}
and this is the recommended approach.
