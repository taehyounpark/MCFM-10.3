\topheading{Input file configuration}
\midheading{Run-time input file parameters}

MCFM execution is performed in the {\tt Bin/} directory,
with syntax:
\begin{center}
	{\tt mcfm }{\it input.ini}
\end{center}
If no command line options are given, then MCFM will default
to using the file {\tt input.ini} in the current directory for
choosing options. The \texttt{input.ini} file can be in any directory and
then the first argument to \texttt{mcfm} should be the location
of the file. Furthermore, one can overwrite or append single
configuration options with additional parameters like:
\begin{center} 
\texttt{./mcfm benchmark/input.ini -general\%part=nlo -lhapdf\%dopdferrors=.true.}
\end{center}
Here specifying a parameter uses a single dash, then the section name as in the input file (see below), followed
by a percent sign, followed by the option name, followed by an equal sign and the actual value of the setting.

All default settings in the input file are explained below, as well as further optional parameters.
The top level setting \texttt{mcfm\_version} specifies the input file version number and it must  match the version of 
the code being used.

The general structure of a fixed-order calculation up to NNLO is as follows:
\begin{equation}
\sigma = \sigma_0 + \Delta\sigma_1 + \Delta\sigma_2 \,,
\end{equation}
where $\Delta\sigma_k$ is of order $\alpha_s^k$ with respect to the leading
order cross section $\sigma_0$, thus representing the
N$^k$LO contribution to the cross section.
When performing the NLO calculation using dipole subtraction
its contribution to the cross section can be decomposed as,
\begin{equation}
\Delta\sigma_1 = \Delta\sigma_1^v + \Delta\sigma_1^r \,.
\end{equation}
$\Delta\sigma_1^v$ includes virtual (loop) contributions, as well
as counterterms that render them finite.
$\Delta\sigma_1^r$ includes contributions from diagrams
involving real parton emission, again with counterterms to make them finite.
Only the sum of $\Delta\sigma_1^v$ and $\Delta\sigma_1^v$ is physical.

This contribution can also be computed using a slicing method with the
corresponding decomposition,
\begin{equation}
\Delta\sigma_1^a = \Delta\sigma_1^{a,<} + \Delta\sigma_1^{a, >} \,.
\end{equation}
$a$ labels the slicing resolution variable, which in MCFM can be either
0-jettiness, $q_T$ (of a color-singlet system) or $p_T^{j_1}$ (lead jet $p_T$)
(thus corresponding to a jet veto).
$\Delta\sigma_1^{a,<}$ is termed the below-cut slicing contribution which is
computed by the means of a factorization theorem and includes loop contributions.
$\Delta\sigma_1^{a,>}$ is the above-cut contribution containing radiation of
an additional parton.
Only the sum $\Delta\sigma_1^a$ is physical and contains a dependence on
the slicing resolution variable $a_{\text{cut}}$ that tends to zero as
$a_{\text{cut}} \to 0$

At NNLO only slicing calculations are available.  The decomposition is,
\begin{equation}
\Delta\sigma_2^a = \Delta\sigma_2^{a,<} + \Delta\sigma_2^{a, v>}  + \Delta\sigma_2^{a, r>} \,.
\end{equation}
$\Delta\sigma_2^{a,<}$ is the below-cut slicing contribution containing 2-loop
contributions.
$\Delta\sigma_1^{a, v>}$ is the above-cut contribution containing loop corrections
to radiation of an additional parton.
$\Delta\sigma_1^{a, r>}$ is the above-cut contribution representing
radiation of up to two additional partons.
Only the sum $\Delta\sigma_2^a$ is physical and contains a dependence on
the slicing resolution variable $a_{\text{cut}}$ that tends to zero as
$a_{\text{cut}} \to 0$

The type of computation that is performed depends on the parameter
\texttt{part} in the \texttt{general} section.
The list of possible values,
and the associated meaning, is shown in Tables~\ref{tab:partchoicesfo}
and~\ref{tab:partchoicesresum}.  They can also be listed by
setting \texttt{part} equal to \texttt{help in the input file}.


\begin{longtable}{p{4.5cm}p{9.0cm}}
\caption{Possible values for the parameter \texttt{part} that correspond to
performing a fixed-order calculation. \label{tab:partchoicesfo}} \\
		\hline
		\texttt{part} & description\\
		\hline
			 {\tt lo}/{\tt lord} &
			$\sigma_0$
			\\
			 {\tt virt} &
			$\Delta\sigma_1^v$
			\\
			 {\tt real} &
			$\Delta\sigma_1^r$
			\\
			 {\tt nlocoeff}/{\tt totacoeff} &
			$\Delta\sigma_1$
			\\
			 {\tt nlo}/{\tt tota} &
			$\sigma_0+\Delta\sigma_1$. For photon processes that include fragmentation,
			{\tt nlo} also includes the calculation of the fragmentation ({\tt frag})
			contributions.
			\\
			 {\tt frag} &
			Processes 280, 285, 290, 295, 300-302, 305-307,  820-823 only, see sections~\ref{subsec:gamgam},
			\ref{subsec:wgamma} and
			\ref{subsec:zgamma} below.
			\\
			 {\tt nlodk}/{\tt todk} &
			Processes 114, 161, 166, 171, 176, 181, 186, 141, 146, 149, 233, 238, 501, 511 only, see
			sections~\ref{subsec:stop} and
			\ref{subsec:wt} below.
			\\
                         {\tt snloR} & $\Delta\sigma_1^{a,>}$
			 \\
                         {\tt snloV} & $\Delta\sigma_1^{a,<}$
			 \\
                         {\tt snlocoeff}/{\tt scetnlocoeff} & $\Delta\sigma_1^a$
			 \\
                         {\tt snlo}/{\tt scetnlo} & $\sigma_0 + \Delta\sigma_1^a$
			 \\
                         {\tt nnloVVcoeff} & $\Delta\sigma_2^{a,<}$
			 \\
                         {\tt nnloRVcoeff} & $\Delta\sigma_2^{a,v>}$
			 \\
                         {\tt nnloRRcoeff} & $\Delta\sigma_2^{a,r>}$
			 \\
                         {\tt nnloVV} & $\Delta\sigma_1^{a,<} + \Delta\sigma_2^{a,<}$
			 \\
                         {\tt nnloRV} & $\Delta\sigma_1^{a,>} + \Delta\sigma_2^{a,v>}$
			 \\
                         {\tt nnloRR} & $\Delta\sigma_2^{a,r>}$
			 \\
                         {\tt nnlocoeff} & $\Delta\sigma_2^{a}$
			 \\
			 {\tt nnlo} & $\sigma_0 + \Delta\sigma_1 + \Delta\sigma_2^{a}$
\end{longtable}

\begin{longtable}{p{4.5cm}p{9.0cm}}
\caption{Possible values for the parameter \texttt{part} that correspond to
performing a calculation including large-log resummation. \label{tab:partchoicesresum}} \\
		\hline
		\texttt{part} & description\\
		\hline
                         {\tt resLO} & NLL resummed and matched
			 \\
                         {\tt resonlyLO} & NLL resummed only
			 \\
                         {\tt resonlyLOp} & NLLp resummed only
			 \\
                         {\tt resexpNLO} & NNLL resummed expanded to NLO
			 \\
                         {\tt resonlyNLO} & NNLL resummed
			 \\
                         {\tt resaboveNLO} & fixed-order matching to NLO
			 \\
                         {\tt resmatchcorrNLO} & matching corrections at NLO
			 \\
                         {\tt resonlyNLOp} & NNLLp resummed
			 \\
                         {\tt resexpNNLO} & N$^3$LL resummed expanded to NNLO
			 \\
                         {\tt resonlyNNLO} & N$^3$LL resummed
			 \\
                         {\tt resaboveNNLO} & fixed-order matching to NLO
			 \\
                         {\tt resmatchcorrNNLO} & matching corrections at NLO
			 \\
                         {\tt resLOp} & NLLp resummed and matched
			 \\
                         {\tt resNLO} & NNLL resummed, matched to NLO
			 \\
                         {\tt resNLOp} & N$^3$LL resummed, matched to NLO
			 \\
                         {\tt resNNLO} & N$^3$LL resummed, matched to NNLO
			 \\
                         {\tt resNNLOp} & N$^3$LLp resummed, matched to NNLO
			 \\
                         {\tt resonlyNNLOp} & N$^3$LLp resummed
\end{longtable}

\bottomheading{General}
\begin{longtable}{p{1.5cm}p{12cm}}
		\hline
		\multicolumn{1}{c}{{\textbf{Section} \texttt{general}}} & \multicolumn{1}{c}{{\textbf{Description}}} \\ 
		\hline
		\texttt{nproc} & 
		The process to be studied is given by
		choosing a process number, according to Tables
		in Section~\ref{MCFMprocs}.
		$f(p_i)$ denotes a generic partonic jet. Processes denoted as
		``LO'' may only be calculated in the Born approximation. For photon
		processes, ``NLO+F'' signifies that the calculation may be performed
		both at NLO and also including the effects of photon fragmentation
		and experimental isolation. In contrast, ``NLO'' for a process involving
		photons means that no fragmentation contributions are included and isolation
		is performed according to the procedure of Frixione~\cite{Frixione:1998jh}.	\\
 		\texttt{part} &
		The type of calculation to be performed.  Possible values are given in 
		Tables~\ref{tab:partchoicesfo} and~\ref{tab:partchoicesresum}. \\
		\texttt{runstring} &
		When MCFM is run, it will write output to several files. The
		label {\tt runstring} will be included in the names of these files.
		\\
		\texttt{rundir} &
		Directory for output and snapshot files
		\\
		\texttt{sqrts} & Center of mass energy in GeV. \\
		\texttt{ih1}, \texttt{ih2} &
		The identities of the incoming hadrons
		may be set with these parameters, allowing simulations for both
		$p{\bar p}$ (such as the Tevatron) and $pp$ (such as the LHC). 
		Setting {\tt ih1} equal to ${\tt +1}$ corresponds to
		a proton, whilst ${\tt -1}$ corresponds to an anti-proton. \\
%		Values greater than {\tt 1000d0} represent a nuclear collision,
%		as described in Section~\ref{sec:nucleus}. \\
		\texttt{zerowidth} &
		When set to {\tt .true.} then all 
		bosons are produced on-shell. This is appropriate for calculations
		of {\it total} cross-sections (such as when using {\tt removebr} equal
		to {\tt .true.}, below). When interested in decay products of the
		bosons this should be set to {\tt .false.}. \\
		\texttt{removebr} &
		When set to {\tt .true.} the branching ratios are 
		removed for unstable particles such as vector bosons or top quarks. See the
		process notes in Section~\ref{sec:specific}, or the process
                web-pages accessed via the \href{\mcfmprocs/proclist.html}{list of processes}
                for further details. \\
		\texttt{ewcorr} & 
		Specifies whether or not to compute EW corrections
		for the process.  Default is {\tt none}.  May be set to {\tt exact}
		or {\tt sudakov} for processes {\tt 31} (neutral-current DY),
		{\tt 157} (top-pair production) and {\tt 190} (di-jet production).
		For more details see section~\ref{subsec:EW}.		\\
%                {\texttt{vdecayid}}, {\texttt{v34id}}, {\texttt{v56id}} &
%		Flags to manually set the decays of vector bosons (34) and
%		(56) (experimental, not for general use). \\
		\hline
	\end{longtable}

\bottomheading{Resummation}
\begin{longtable}{p{1.5cm}p{12cm}}
		\toprule
		\multicolumn{1}{c}{{\textbf{Section} \texttt{resummation}}} & \multicolumn{1}{c}{{\textbf{Description}}} \\ 
		\midrule
\begin{minipage}[t]{0.24\columnwidth}\raggedright
\texttt{usegrid}\strut
\end{minipage} & \begin{minipage}[t]{0.71\columnwidth}\raggedright
\texttt{.true.} or \texttt{.false.} determines whether pregenerated
LHAPDF interpolation grids should be used for the resummation beam
functions.\strut
\end{minipage}\tabularnewline
\begin{minipage}[t]{0.24\columnwidth}\raggedright
\texttt{makegrid}\strut
\end{minipage} & \begin{minipage}[t]{0.71\columnwidth}\raggedright
If \texttt{.true.}, then MCFM runs in grid generation mode. This
generates LHAPDF grid files in the directory \texttt{gridoutpath} from
LHAPDF grids in the directory \texttt{gridinpath}. After the grid
generation MCFM stops and should be run subsequently with
\texttt{makegrid = .false.} and \texttt{usegrid = .true.}. When
\texttt{lhapdf\%dopdferrors=.true.} then also grids for the error sets
are generated.\strut
\end{minipage}\tabularnewline
\begin{minipage}[t]{0.24\columnwidth}\raggedright
\texttt{gridoutpath}\strut
\end{minipage} & \begin{minipage}[t]{0.71\columnwidth}\raggedright
Output directory for LHAPDF grid files, for example
\texttt{/home/tobias/local/share/LHAPDF/}\strut
\end{minipage}\tabularnewline
\begin{minipage}[t]{0.24\columnwidth}\raggedright
\texttt{gridinpath}\strut
\end{minipage} & \begin{minipage}[t]{0.71\columnwidth}\raggedright
Input directory for LHAPDF grid files, for example
\texttt{/home/tobias/local/share/LHAPDF/}\strut
\end{minipage}\tabularnewline
\begin{minipage}[t]{0.24\columnwidth}\raggedright
\texttt{res\_range}\strut
\end{minipage} & \begin{minipage}[t]{0.71\columnwidth}\raggedright
Integration range of purely resummed part, for example \texttt{0.0 80.0}
for \(q_T\) integration between 0 and 80 GeV.\strut
\end{minipage}\tabularnewline
\begin{minipage}[t]{0.24\columnwidth}\raggedright
\texttt{resexp\_range}\strut
\end{minipage} & \begin{minipage}[t]{0.71\columnwidth}\raggedright
Integration range of fixed-order expanded resummed part, for example
\texttt{1.0 80.0} for \(q_T\) integration between 1 and 80 GeV.\strut
\end{minipage}\tabularnewline
\begin{minipage}[t]{0.24\columnwidth}\raggedright
\texttt{fo\_cutoff}\strut
\end{minipage} & \begin{minipage}[t]{0.71\columnwidth}\raggedright
Lower \(q_T\) cutoff $q_0$ for the fixed-order part. % see eq.~\eqref{eq:matchingmod} below.
Typically the value should agree with the lower range of \texttt{resexp\_range}.\strut
\end{minipage}\tabularnewline
\begin{minipage}[t]{0.24\columnwidth}\raggedright
\texttt{transitionswitch}\strut
\end{minipage} & \begin{minipage}[t]{0.71\columnwidth}\raggedright
Parameter passed to the plotting routine to modify the transition
function, see text.\strut
\end{minipage}\tabularnewline
\bottomrule
\end{longtable}

\bottomheading{NNLO}
	\begin{longtable}{p{1.5cm}p{12cm}}
		\hline
		\multicolumn{1}{c}{{\textbf{Section} \texttt{nnlo}}} & \multicolumn{1}{c}{{\textbf{Description}}} \\ 
		\hline
		\texttt{taucut} & 
		Optional. This sets the value of the jettiness variable
		$\tau_{\text{cut}}$, as a multiple of the invariant mass of the Born system,
		i.e.
		\begin{equation}
		\tau_{\text{cut}} = \texttt{taucut} \times Q
		\end{equation}
		This variable separates the resolved and unresolved regions in NNLO
		calculations that use zero-jettiness. The default value results
		in total inclusive cross sections with less than $1\%$ residual cutoff effects. \\
		\texttt{tcutarray} &
		Optional. Array that specifies multiple taucut values that should be sampled
		on the fly in addition to the nominal taucut value. Both larger and smaller
		values than the nominal one can be specified, although uncertainties for
		smaller values will be large. We generally do not recommend smaller values
		than the nominal one chosen with \texttt{taucut}. Default values are chosen
		to be $2,4,8,20,40$ times the nominal choice of \texttt{taucut}.  \\
		\texttt{dynamictau} &
		Optional. If \texttt{.false.}, the \texttt{taucut} value specified
		is not multiplied by the invariant mass of the Born system. Default is \texttt{.true.}. \\
                \texttt{useqt} & Flag to use $q_T$ slicing, rather than
		0-jettiness, in the calculation of NNLO contributions.
		Default is \texttt{.false.} \\
                \texttt{useGLY} & If \texttt{.true.}, implement non-local $q_T$ subtraction using formulas from Gehrmann et al.(GLY)~\cite{Gehrmann:2014yya}.
		Default is \texttt{.true.}  when \texttt{useqt} is enabled.  If \texttt{.false},
		implement non-local $q_T$ subtraction using formulas from Billis et al.(BEMT)~\cite{Billis:2019vxg}. \\
                \texttt{qtcut} & If \texttt{useqt} is enabled, the value of the slicing parameter, defined
		in the same way as \texttt{taucut} described above.  \\
                \texttt{tauboost} & When using 0-jettiness, perform the slicing cut in the
		centre-of-mass of the color singlet system.  Default is \texttt{.true.} \\
                \texttt{incpowcorr} & When using 0-jettiness, include leading power corrections
		in the below-cut calculation. Default is \texttt{.false.} \\
                \texttt{onlypowcorr} & When using 0-jettiness, only compute the power corrections
		to the below-cut calculation. Default is \texttt{.false.} \\
                \texttt{usept} & This flag has two separate uses.  In a fixed-order sliciing calculation,
		e.g. {\tt part} is equal to {\tt snlo} or {\tt nnlo}, the code uses $p_T^{veto}$ slicing, rather than
		0-jettiness, in the calculation of higher-order contributions.
		In a resummed calculation, e.g. {\tt part} is equal to {\tt resNLO} or {\tt resNNLO},
		it enables the use of $p_T^{veto}$ resummation rather that $q_T$ resummation.
		In this case the value of the jet veto (\texttt{ptveto}) is set separately in the
		\texttt{resummation} block.
		Default is \texttt{.false.} \\
                \texttt{useBNR} & Implements $p_T^{veto}$ formalism using the refactorized approach of
		Ref.~\cite{Becher:2013xia}.  Otherwise uses original `$B \times B \times S$' factorization
		of below-cut cross-section into beam and soft functions (that gives identical results).
		Default is \texttt{.true.} \\
		\hline
	\end{longtable}

\bottomheading{PDFs}
	\begin{longtable}{p{1.5cm}p{12cm}}
		\toprule
		\multicolumn{1}{c}{{\textbf{Section} \texttt{pdf}}} & \multicolumn{1}{c}{{\textbf{Description}}} \\ 
		\midrule
		\texttt{pdlabel} &
		This specifies the parton distributions used in case the code has been built with
		\texttt{PDFROUTINES = NATIVE}. The choice of parton distribution is made by
		inserting the appropriate 7-character code from the table in \cref{subsec:pdfsets}
		or in \cref{olderPDFs} for historical \PDF{} sets.
		As mentioned above, this also sets the value of $\alpha_S(M_Z)$.\\
		\bottomrule
	\end{longtable}

\bottomheading{LHAPDF}
	\begin{longtable}{p{1.5cm}p{12cm}}
		\toprule
		\multicolumn{1}{c}{{\textbf{Section} \texttt{lhapdf}}} & \multicolumn{1}{c}{{\textbf{Description}}} \\ 
		\midrule
		\texttt{lhapdfset} &
		Specifies the parton distributions used in case the code has been built with
		\texttt{PDFROUTINES = LHAPDF}. For a default global installation the PDFs reside
		in \texttt{/usr/share/LHAPDF/} or \texttt{/usr/local/share/LHAPDF}, and the name
		equals the set name from \url{https://lhapdf.hepforge.org/pdfsets.html}, which is
		also the directory name of the sets. Multiple PDF sets separated by a space can be specified. \\
		\texttt{lhapdfmember} & Specifies the individual members of the parton distribution sets.
		A value of zero corresponds to the central value for Hessian sets. In case multiple sets
		have been specifies above, each one needs a member number separated by space. \\
		\texttt{dopdferrors} & When this is set to \texttt{.true.} PDF uncertainties are calculated
		for every specified \PDF{} set according to the routines provided by \LHAPDF{}.
		The \texttt{lhapdfmember} numbers are ignored but must still be set for each member. \\
		\bottomrule
	\end{longtable}

\bottomheading{Scales}
	\begin{longtable}{p{1.5cm}p{12cm}}
		\toprule
		\multicolumn{1}{c}{{\textbf{Section} \texttt{scales}}} & \multicolumn{1}{c}{{\textbf{Description}}} \\ 
		\midrule
		\texttt{renscale} &
		This parameter may be used to adjust the value
		of the {\it renormalization} scale. This is the scale
		at which $\alpha_S$ is evaluated and will typically be set to
		a mass scale appropriate to the process ($M_W$, $M_Z$, $M_t$ for
		instance). \\
		\texttt{facscale} &
		This parameter may be used to adjust the value
		of the {\it factorization} scale and will typically be set to
		a mass scale appropriate to the process ($M_W$, $M_Z$, $M_t$ for
		instance). \\
		\texttt{dynamicscale} &
		This character string is used to specify whether
		the renormalization, factorization and fragmentation scales are dynamic, i.e. recalculated
		on an event-by-event basis. If this string is set to `{\tt none}' then the scales
		are fixed for all events at the values	specified by {\tt renscale}, {\tt facscale}
		as well as \texttt{fragmentation\_scale} as defined further below.
		
		The type of dynamic scale to be used is selected by using a particular string
		for the variable {\tt dynamicscale}, as indicated in \cref{tab:dynamicscales} on \cpageref{tab:dynamicscales}.
		Not all scales are defined for each process, with program execution halted if
		an invalid selection is made in the input file.
		The selection chooses a reference scale, $\mu_0$. The actual scales used in
		the code are then,
		\begin{equation}
		\mu_{\mathrm{ren}} = {\tt scale} \times \mu_0 \;, \qquad
		\mu_{\mathrm{fac}} = {\tt facscale} \times \mu_0
		\label{eq:dynscale}
		\end{equation}
		Note that, for simplicity, the fragmentation scale (relevant only for processes
		involving photons) is set equal to the renormalization scale.
		In some cases it is possible for the dynamic scale to become very large. This can cause problems 
		with the interpolation of data tables for the PDFs and fragmentation functions. As a result if a dynamic scale 
		exceeds a maximum of $60$ TeV (PDF) or $990$ GeV (fragmentation) this value is set by default to the maximum. 	
		\\
		\texttt{doscalevar} &
		
		This additional option can be set to \texttt{.true.} to enable scale variation.
		It performs a variation of the scales used in \cref{eq:dynscale} by a factor of 
		two so that it surveys the 
		additional possibilities,
		\begin{eqnarray}
		&&
		(2\mu_{\mathrm{ren}},2\mu_{\mathrm{fac}}),
		(\mu_{\mathrm{ren}}/2,\mu_{\mathrm{fac}}/2), \nonumber \\ &&
		(2\mu_{\mathrm{ren}},\mu_{\mathrm{fac}}),
		(\mu_{\mathrm{ren}}/2,\mu_{\mathrm{fac}}),
		(\mu_{\mathrm{ren}},2\mu_{\mathrm{fac}}),
		(\mu_{\mathrm{ren}},\mu_{\mathrm{fac}}/2) \,.
		\label{eq:scalevar}
		\end{eqnarray}
		The histograms corresponding to these different choices are included in the output file, from which an
		envelope of theoretical uncertainty may be constructed by the user. \\
		\texttt{maxscalevar} &
		Number of additional scale variation points to choose, can be set to two or six. For two
		it just samples the first two variations as in eq.~\ref{eq:scalevar}. \\
		\bottomrule
	\end{longtable}

\begin{table}
	\begin{center}
		\begin{longtable}{|l|l|l|}
			\hline
			{\tt dynamic scale} & $\mu_0^2$ & comments\\
			\hline 
			{\tt m(34)} & $(p_3+p_4)^2$ & \\
			{\tt m(345)} & $(p_3+p_4+p_5)^2$ & \\
			{\tt m(3456)} & $(p_3+p_4+p_5+p_6)^2$ & \\
			{\tt sqrt(M\pow 2+pt34\pow 2)} & $M^2 + (\vec{p_T}_3 + \vec{p_T}_4)^2$ & $M=$~mass of particle 3+4 \\
			{\tt sqrt(M\pow 2+pt345\pow 2)} & $M^2 + (\vec{p_T}_3 + \vec{p_T}_4 + \vec{p_T}_5)^2$ & $M=$~mass of 
			particle 3+4+5 \\
			{\tt sqrt(M\pow 2+pt5\pow 2)} & $M^2 + \vec{p_T}_5^2$ & $M=$~mass of particle 3+4 \\
			{\tt sqrt(M\pow 2+ptj1\pow 2)} & $M^2 + \vec{p_T}_{j_1}^2$ & $M=$~mass(3+4), $j_1=$ leading $p_T$ jet \\
			{\tt pt(photon)} & $\vec{p_T}_\gamma^2$ & \\
			{\tt pt(j1)} & $\vec{p_T}_{j_1}^2$ & \\
			{\tt HT} & $\sum_{i=1}^n {p_T}_i$ & $n$ particles (partons, not jets) \\
			\hline 
			\hline\end{longtable}
	\end{center}
	\caption{Choices of the input parameter {\tt dynamicscale} that result in an event-by-event
		calculation of all relevant scales using the given reference scale-squared $\mu_0^2$.
		\label{tab:dynamicscales}}
\end{table}
\bottomheading{Masses}
	\begin{longtable}{p{1.5cm}p{12cm}}
		\hline
		\multicolumn{1}{c}{{\textbf{Section} \texttt{masses}}} & \multicolumn{1}{c}{{\textbf{Description}}} \\ 
		\hline
		\texttt{hmass} & Higgs pole mass \\
		\texttt{mt} & Top-quark pole mass \\
		\texttt{mb} & Bottom-quark pole mass \\
		\texttt{mc} & Charm-quark pole mass \\
		\texttt{wmass} & W-boson pole mass \\
		\texttt{zmass} & Z-boson pole mass \\
		\hline
	\end{longtable}

\bottomheading{Basic jets}
\label{basicjets}
	\begin{longtable}{p{1.5cm}p{12cm}}
		\toprule
		\multicolumn{1}{c}{{\textbf{Section} \texttt{basicjets}}} & \multicolumn{1}{c}{{\textbf{Description}}} \\ 
		\midrule
		\texttt{inclusive} &
		This logical parameter chooses whether the
		calculated cross-section should be inclusive in the number of jets
		found at \NLO{}. An {\em exclusive}
		cross-section contains the same number of jets at next-to-leading
		order as at leading order. An {\em inclusive} cross-section may
		instead contain an extra jet at \NLO{}. \\
		\texttt{algorithm} &
		This specifies the jet-finding algorithm that
		is used, and can take the values
		{\tt ktal} (for the Run II $k_T$-algorithm), {\tt ankt} (for the
		``anti-$k_T$'' algorithm~\cite{Cacciari:2008gp}), {\tt cone} (for
		a midpoint cone algorithm), {\tt hqrk} (for a simplified cone
		algorithm designed for heavy quark processes) and {\tt none} (to
		specify no jet clustering at all). The latter option is only a
		sensible choice when the leading order cross-section is well-defined
		without any jet definition: e.g. the single top process,
		$q{\bar q^\prime} \to t{\bar b}$, which is finite as
		$p_T({\bar b}) \to 0$. \\
		\texttt{ptjetmin}, \texttt{etajetmax} &
		These specify the values
		of $p_{T,{\mathrm{min}}}$ and $|\eta|_{\mathrm{max}}$ for the
		jets that are found by the algorithm.  \\
		\texttt{etajetmin} &
		Optional parameter for setting a minimum jet rapidity $|\eta|_{\mathrm{min}}$. \\
		\texttt{ptjetmax} &
		Optional parameter for setting maximum jet $p_{T,{\mathrm{min}}}$\\
		\texttt{Rcutjet} &
		If the final state of the chosen process contains
		either quarks or gluons then for each event an attempt will be made
		to form them into jets. For this it is necessary to define the
		jet separation $\Delta R=\sqrt{{\Delta \eta}^2 + {\Delta \phi}^2}$
		so that after jet combination, all jet pairs are separated by
		$\Delta R >$~{\tt Rcutjet}.\\
		\bottomrule
	\end{longtable}

\bottomheading{Mass cuts}
\label{masscuts}
	\begin{longtable}{p{1.5cm}p{12cm}}
		\hline
		\multicolumn{1}{c}{{\textbf{Section} \texttt{masscuts}}} & \multicolumn{1}{c}{{\textbf{Description}}} \\ 
		\hline
		{\tt m34min}, {\tt m34max}, {\tt m56min}, {\tt m56max}, {\tt m3456min}, {\tt m3456max} &
		These parameters represent a basic set of mass cuts that are be applied
		to the calculated cross-section. The only events that contribute to
		the cross-section will have, for example,
		{\tt m34min} $<$ {\tt m34} $<$ {\tt m34max} where {\tt m34} is the
		invariant mass of particles 3 and 4 that are specified by {\tt nproc}.
		{\tt m34min}~$> 0$ is obligatory for processes which can involve a virtual
		photon, such as {\tt nproc=31}. By default, the maximum settings are set to $\sqrt{s}$.\\
		\hline
	\end{longtable}
\clearpage

\bottomheading{Cuts}
	\begin{longtable}{p{1.5cm}p{12cm}}
		\toprule
		\multicolumn{1}{c}{{\textbf{Section} \texttt{cuts}}} & \multicolumn{1}{c}{{\textbf{Description}}} \\ 
		\midrule
		\texttt{makecuts} &
		If this parameter is set to {\tt .false.} then
		no additional cuts are applied to the events and the remaining
		parameters in this section are ignored. Otherwise, events will
		be rejected according to a set of cuts that is specified below.
		Further options may be implemented by editing {\tt src/User/gencuts\_user.f90}. \\
		
		{\tt ptleptmin, etaleptmax} & These specify the values
		of $p_{T,{\mathrm{min}}}$ and $|\eta|_{\mathrm{max}}$ for one of the leptons produced
		in the process. One can also introduce optional settings \texttt{ptleptmax}
		and \texttt{etaleptmin}. \\
		
		{\tt etaleptveto} & This should be specified as a pair of double
		precision numbers that indicate a rapidity range that should be excluded
		for the lepton that passes the above cuts. \\
		
		{\tt ptminmiss} & Specifies the minimum missing transverse
		momentum (coming from neutrinos). \\
		
		{\tt ptlept2min}, \texttt{etalept2max} & These specify
		the values of $p_{T,{\mathrm{min}}}$ and $|\eta|_{\mathrm{max}}$ for the remaining
		leptons in the process. This allows for staggered cuts where, for
		instance, only one lepton is required to be hard and central.
		One can also introduce optional settings \texttt{ptlept2max} and
		\texttt{etalept2min}. \\
		
		{\tt etalept2veto} & This should be specified as a pair of double
		precision numbers that indicate a rapidity range that should be excluded
		for the remaining leptons. \\
		
		\bottomrule
	\end{longtable}
%\end{table}
\clearpage

\bottomheading{Cuts (continued)}
	\begin{longtable}{p{1.5cm}p{12cm}}
		\toprule
		\multicolumn{1}{c}{{\textbf{Section} \texttt{cuts}}} & \multicolumn{1}{c}{{\textbf{Description}}} \\ 
		\midrule
		{\tt m34transmin} & For general processes, this specifies the
		minimum transverse mass of particles 3 and 4,
		\begin{equation}
		\mbox{general}: \quad 2 p_T(3) p_T(4) \left( 1 - \frac{\vec{p_T}(3) \cdot \vec{p_T}(4)}{p_T(3) p_T(4)} \right) 
		> {\texttt{m34transmin}} 
		\end{equation}
		For the $W(\to \ell \nu)\gamma$ process the role of this cut changes, to become
		instead a cut on the transverse cluster mass of the $(\ell\gamma,\nu)$ system,
		\begin{eqnarray}
		W\gamma: && \left[ \sqrt{m_{\ell\gamma}^2 + |\vec{p_T}(\ell)+\vec{p_T}(\gamma)|^2} + p_T(\nu) \right]^2
		\nonumber \\ &&
		-|\vec{p_T}(\ell)+\vec{p_T}(\gamma)+\vec{p_T}(\nu)|^2 >  {\texttt{m34transmin}}^2
		\end{eqnarray}
		For the $Z\gamma$ process this parameter specifies a simple invariant mass cut,
		\begin{equation}
		Z\gamma: \quad m_{Z\gamma} > {\texttt{m34transmin}}
		\end{equation}
		A final mode of operation applies to the $W\gamma$ process and is triggered by a negative value
		of {\texttt{m34transmin}}. This allows simple access to the cut that was employed in v6.0 of the code:
		\begin{eqnarray}
		W\gamma, \mbox{obsolete}: &&
		\left[ p_T(\ell) +  p_T(\gamma) +  p_T(\nu) \right]^2 \nonumber \\ 
		&-&|\vec{p_T}(\ell)+\vec{p_T}(\gamma)+\vec{p_T}(\nu)|^2 > |{\texttt{m34transmin}}|
		\end{eqnarray}
		In each case the screen output indicates the cut that is applied. \\
		{\tt Rjlmin} & Using the definition of $\Delta R$ above,
		requires that all jet-lepton pairs are separated by
		$\Delta R >$~{\tt R(jet,lept)\_min}. \\
		
		{\tt Rllmin} & When non-zero, all lepton-lepton pairs
		must be separated by $\Delta R >$~{\tt R(lept,lept)\_min}. \\
		
		{\tt delyjjmin} & This enforces a pseudo-rapidity
		gap between the two hardest jets $j_1$ and $j_2$, so that:
		$|\eta^{j_1} - \eta^{j_2}| >$~{\tt Delta\_eta(jet,jet)\_min}. \\
		
		{\tt jetsopphem} & If this parameter is set to {\tt .true.},
		then the two hardest jets are required to lie in opposite hemispheres,
		$\eta^{j_1} \cdot \eta^{j_2} < 0$. \\
		
		{\tt lbjscheme} & This integer parameter provides no
		additional cuts when it takes the value {\tt 0}. When equal to
		{\tt 1} or {\tt 2}, leptons are required to lie between the two
		hardest jets. With the ordering $\eta^{j_-} < \eta^{j_+}$ for the
		pseudo-rapidities of jets $j_1$ and $j_2$:
		{\tt lbjscheme = 1} : 
		$\eta^{j_-} < \eta^{\mathrm{leptons}} < \eta^{j_+}$;
		{\tt lbjscheme = 2} :
		$\eta^{j_-}+{\tt Rcutjet} < \eta^{\mathrm{leptons}} < \eta^{j_+}-{\tt Rcutjet}$. \\
		
		{\tt ptbjetmin, etabjetmax} & If a process involving $b$-quarks is being calculated, then these can
		be used to specify {\em stricter} values of $p_T^{\mathrm{min}}$
		and $|\eta|^{\mathrm{max}}$ for $b$-jets. Similarly, values for \texttt{ptbjetmax} and \texttt{etabjetmin} can be 
		specified. \\
		\bottomrule
	\end{longtable}

\bottomheading{Photon}
		Note that all the photon cuts specified in this section of the input file, are applied even if {\tt makecuts} is set to {\tt .false.}.
	\begin{longtable}{p{3.5cm}p{12cm}}
		\hline
		\multicolumn{1}{c}{{\textbf{Section} \texttt{photon}}} & \multicolumn{1}{c}{{\textbf{Description}}} \\ 
		\hline
		{\tt fragmentation} &  This parameter is a logical variable that determines whether the production of photons 
		by a parton 
		fragmentation process is included. If {\tt fragmentation} is set to {\tt .true.}, the code uses a standard 
		cone isolation
		procedure (that includes LO fragmentation contributions in the NLO calculation).
		If {\tt fragmentation} is set to {\tt .false.}, the code implements
		a Frixione-style photon cut~\cite{Frixione:1998jh},
		\begin{equation}
		\sum_{i \in R_0} E_{T,i}^j  < \epsilon_h E_{T}^{\gamma} \bigg(\frac{1-\cos{R_{i\gamma}}}{1-\cos{R_0}}\bigg)^{n} 
		\;.
		\label{frixeq}
		\end{equation}
		In this equation, $R_0$, $\epsilon_h$ and $n$ are defined by {\tt cone\_ang}, {\tt epsilon\_h} 
		and {\tt n\_pow}  respectively (see below).
		$E_{T,i}^{j}$ is the transverse energy of a parton, $E_{T}^\gamma$ is the
		transverse energy of the photon and $R$ is the separation between the photon and the parton using the 
		usual definition
		\begin{equation}
		R=\sqrt{\Delta\phi_{i\gamma}^2+\Delta\eta_{i\gamma}^2} \,.
		\end{equation}
		$n$ is an integer parameter which by default is set to~1. \\
		
		{\tt fragmentation\_set} & A length eight character variable that is used to choose the particular photon 
		fragmentation set.
		Currently implemented fragmentation functions can be called with `{\tt BFGSet\_I}', `{\tt 
			BFGSetII}'~\cite{Bourhis:1997yu} or `{\tt GdRG\_\_LO}'~\cite{GehrmannDeRidder:1998ba}. \\
		
		{\tt fragmentation\_scale} & A double precision variable that will be used to choose the scale 
		at which the photon fragmentation is evaluated. \\
		
		{\tt gammptmin} & This specifies the value
		of $p_T^{\mathrm{min}}$ for the photon with the largest transverse momentum.
		Note that this cut, together with all the photon cuts specified in this section
		of the input file, are applied even if {\tt makecuts} is set to {\tt .false.}.
		One can also add an entry for \texttt{gammptmax} to cut on a range. \\
		
		{\tt gammrapmax} & This specifies the value
		of $|y|^{\mathrm{max}}$ for any photons produced in the process. One can also add an entry
		for \texttt{gammrapmin} to cut on a range. \\
		
		{\tt gammpt2}, {\tt gammpt3} & The values
		of $p_T^{\mathrm{min}}$ for the second and third photons, ordered by $p_T$. \\
		
		{\tt Rgalmin} & Using the usual definition of $R$ above,
		this requires that all photon-lepton pairs are separated by
		$R >$~{\tt Rgalmin}. This parameter must be non-zero
		for processes in which photon radiation from leptons is included. \\
		
		{\tt Rgagamin} & Using the usual definition of $R$ above,
		this requires that all photon pairs are separated by
		$R >$~{\tt Rgagamin}. \\
		
		{\tt Rgajetmin} & Using the usual definition of $R$ above,
		this requires that all photon-jet pairs are separated by
		$R >$~{\tt Rgajetmin}. \\
		
		{\tt cone\_ang} & Fixes the cone size ($R_0$) for photon isolation.
		This cone is used in both forms of isolation. \\
		
		{\tt epsilon\_h} & This cut controls the amount of radiation allowed in cone when  {\tt fragmentation} is set 
		to 
		{\tt .true.}. If  {\tt epsilon\_h} $ < 1$ then the photon is isolated using
		$\sum_{\in R_0} E_T{\mathrm{(had)}} < \epsilon_h \, p^{\gamma}_{T}.$ Otherwise {\tt epsilon\_h}  $ > 1$ sets 
		$E_T(max)$ in  $\sum_{\in R_0} E_T{\mathrm{(had)}} < E_T(max)$. \\
		
		{\tt n\_pow} & When using the Frixione isolation prescription, the exponent $n$ in Eq.~(\ref{frixeq}). \\
		
                {\tt fixed\_coneenergy} & This is only operational when using the Frixione isolation prescription.
		If {\tt fixed\_coneenergy} is .false. then $\epsilon_h$ controls the amount of hadronic energy allowed 
		inside the cone using the
		Frixione isolation prescription (see above, Eq.~(\ref{frixeq}))
		If {\tt fixed\_coneenergy} is .true. then this formula
		is replaced by one where $\epsilon_h E_T^\gamma \rightarrow \epsilon_h$. \\		

                {\tt hybrid}, {\tt  R\_inner} & If {\tt hybrid} is set to .true. use a hybrid isolation scheme
		with Frixione isolation on an inner cone of radius {\tt  R\_inner}. \\
    \hline
	\end{longtable}

\bottomheading{Histograms}
	\begin{longtable}{p{1.5cm}p{12cm}}
		\toprule
		\multicolumn{1}{c}{{\textbf{Section} \texttt{histogram}}} & \multicolumn{1}{c}{{\textbf{Description}}} \\ 
		\midrule
		\texttt{writetop} & Write output histograms suitable as input for top-drawer. \\
		\texttt{writetxt} & Write output histograms as whitespace-separated columns. \\
		\texttt{newstyle} & Use the new plotting infrastructure introduced in MCFM-10.0\\
		\bottomrule
	\end{longtable}

\bottomheading{Imtegration}
	\begin{longtable}{p{1.5cm}p{12cm}}
		\hline
		\multicolumn{1}{c}{{\textbf{Section} \texttt{integration}}} & \multicolumn{1}{c}{{\textbf{Description}}} \\ 
		\hline
		\texttt{usesobol} & When \texttt{.true.} and the number of MPI processes is a power of two, the Sobol 
		sequence is used, see ref.~\cite{MCFM9}, otherwise the MT19937 pseudo random number generator. \\
		\texttt{seed} & Initialization seed for MT19937 pseudo random number generator. \\
		\texttt{precisiongoal} & Relative precision goal for the integration. \\
		\texttt{readin} & When \texttt{.true.} the automatically written snapshot from a previous run will be read-in
		to resume the integration. \\
		\texttt{writeintermediate} & When \texttt{.true.} histograms are written after each Vegas iteration. \\
		\texttt{warmupprecisiongoal} & Sets the relative precision goal for the warmup run. Unless this precision
		is reached, the number of calls for the warmup is increased. \\
		\texttt{warmupchisqgoal} & Sets the $\chi^2$ per iteration goal for the warmup run. Unless the 
		$\chi^2/\text{it.}$ of the warmup is below this target, the number of calls for the warmup is increased. \\
		\hline
	\end{longtable}
\clearpage

\midheading{Process specific options}
\bottomheading{Single Top}
	\begin{longtable}{p{4.5cm}p{10.5cm}}
		\hline
		\multicolumn{1}{c}{{\textbf{Section} \texttt{singletop}}} & \multicolumn{1}{c}{{\textbf{Description}}} \\ 
		\hline
		\texttt{c\_phiq} & Sets real Wilson coefficient of $\Qone$ for processes 164 and 169. See \ref{subsec:offstop} 
		and ref.~\cite{Neumann:2019kvk}. \\
		\texttt{c\_phiphi} & Sets real and imaginary part of the $\Qtwo$ Wilson coefficient. \\
		\texttt{c\_tw} & Sets real and imaginary part of the $\Qthree$ Wilson coefficient. \\
		\texttt{c\_bw} & Sets real and imaginary part of the $\Qfour$ Wilson coefficient. \\
		\texttt{c\_tg} & Sets real and imaginary part of the $\Qsix$ Wilson coefficient. \\
		\texttt{c\_bg} & Sets real and imaginary part of the $\Qseven$ Wilson coefficient. \\
		\texttt{lambda} & Scale $\Lambda$, see \ref{subsec:offstop} and ref.~\cite{Neumann:2019kvk}. \\
		\texttt{enable\_lambda4} & Enable contributions of order $1/\Lambda^4$ when set to \texttt{.true.}. \\
		\texttt{disable\_sm} & When set to \texttt{.true.} the pure SM contributions are disabled, and just
		the SM-EFT interference and EFT contributions are calculated. \\
		\texttt{mode\_anomcoup} & When set to \texttt{.true.} at LO one can reproduce results obtained
			without power counting as in the anomalous couplings approach, see \ref{subsec:offstop} and 
			ref.~\cite{Neumann:2019kvk}. \\
               \texttt{nnlo\_enable\_light},
               \texttt{nnlo\_enable\_heavy\_prod},
               \texttt{nnlo\_enable\_heavy\_decay},
               \texttt{nnlo\_enable\_interf\_lxh},
               \texttt{nnlo\_enable\_interf\_lxd},
               \texttt{nnlo\_enable\_interf\_hxd},
               \texttt{nnlo\_fully\_inclusive}&
               At NNLO there are several different contributions from vertex
               corrections on the light-quark line, heavy-quark line in production, and
               heavy-quark line in the top-quark decay. Additionally there are one-loop
               times one-loop interference contributions between all three
               contributions. For a fully inclusive calculation without decay
	       \texttt{nnlo\_fully\_inclusive}  has to
               be set to `.true.` and the decay and decay interference parts have to be
               removed.	 Additionally jet requirements must be lifted.  For further information see
	       Section~\ref{single-top-quark-production-and-decay-at-nnlo}.      \\
    \hline
	\end{longtable}

 


\bottomheading{Anomalous $W/Z$ couplings}
	\begin{longtable}{p{1.5cm}p{12cm}}
		\hline
		\multicolumn{1}{c}{{\textbf{Section} \texttt{anom\_wz}}} & \multicolumn{1}{c}{{\textbf{Description}}} \\ 
		\hline
		{\tt enable} &  Boolean flag to enable anomalous W-boson and Z-boson coupling contributions for certain 
		processes. 	False has the same effect as setting all anomalous couplings to zero, but additionally skips 
		computation of anomalous coupling code parts. \\
		 & \\
		{\tt delg1\_z} & $\Delta g_1^Z$ {\it See section~\ref{subsec:diboson}.} \\
		{\tt delk\_z} & $\Delta\kappa^Z$ {\it See section~\ref{subsec:diboson}.} \\
		{\tt delk\_g} & $\Delta\kappa^\gamma$ {\it See sections~\ref{subsec:diboson} and~\ref{subsec:wgamma}.} \\
		{\tt lambda\_z} & $\Lambda^Z$ {\it See section~\ref{subsec:diboson}.} \\
		{\tt lambda\_g} & $\Lambda^\gamma$ {\it See sections~\ref{subsec:diboson} and~\ref{subsec:wgamma}.} \\
		 & \\
		{\tt h1Z} & $h_1^Z$ {\it Anomalous couplings for $Z\gamma$ process at NNLO. See 
		section~\ref{subsec:zgamma}.} \\
		{\tt h1gam} & $h_1^\gamma$ {\it See section~\ref{subsec:zgamma}.} \\
		{\tt h2Z} & $h_2^Z$ {\it See section~\ref{subsec:zgamma}.} \\
		{\tt h2gam} & $h_2^\gamma$ {\it See section~\ref{subsec:zgamma}.} \\
		{\tt h3Z} & $h_3^Z$ {\it See section~\ref{subsec:zgamma}.} \\
		{\tt h3gam} & $h_3^\gamma$ {\it See section~\ref{subsec:zgamma}.} \\
		{\tt h4Z} & $h_4^Z$ {\it See section~\ref{subsec:zgamma}.} \\
		{\tt h4gam} & $h_4^\gamma$ {\it See section~\ref{subsec:zgamma}.} \\
		 & \\
		{\tt tevscale} & Form-factor scale, in TeV {\it See section~\ref{subsec:diboson}.} 
		No form-factors are applied to the anomalous couplings if this value is negative. \\
		\hline
	\end{longtable}
%\end{table}

\clearpage

\bottomheading{$W/Z$+2 jets}
	\begin{longtable}{p{1.5cm}p{12cm}}
		\hline
		\multicolumn{1}{c}{{\textbf{Section} \texttt{wz2jet}}} & \multicolumn{1}{c}{{\textbf{Description}}} \\
		\hline
		\texttt{Qflag} &
		This only has an effect when running a
		$W+2$~jets or $Z+2$~jets process. When {\tt.true.}, it includes the effect of four-quark processes. Please see section~\ref{subsec:w2jets}
		below. \\
		\texttt{Gflag} &
		This only has an effect when running a
		$W+2$~jets or $Z+2$~jets process. When {\tt.true.}, it includes the effect of two-quark, two-gluon processes.
                Please see section~\ref{subsec:w2jets}
		below. \\
		\hline
\end{longtable}
        

\bottomheading{H jetmass}
	\begin{longtable}{p{1.5cm}p{12cm}}
		\toprule
		\multicolumn{1}{c}{{\textbf{Section} \texttt{hjetmass}}} & \multicolumn{1}{c}{{\textbf{Description}}} \\ 
		\midrule
		{\tt mtex} &
		Sets the order $k=0,2,4$ of the $1/m_t^k$ expansion for virtual corrections in the $H+$jet process 200. See 
		section \ref{subsec:hjetma}. \\
		\bottomrule
	\end{longtable}

\bottomheading{Anomalous $H$ couplings}
	\begin{longtable}{p{1.5cm}p{12cm}}
		\hline
		\multicolumn{1}{c}{{\textbf{Section} \texttt{anom\_higgs}}} & \multicolumn{1}{c}{{\textbf{Description}}} \\ 
		\hline
		{\tt hwidth\_ratio} & For processes {\tt 123}--{\tt 126}, {\tt 128}--{\tt 133} only,
		this variable provides a rescaling of the width of the Higgs boson.  Couplings are rescaled such that the
		corresponding cross section close to the Higgs boson peak is unchanged.  Further details of this procedure are 
		given in \href{https://arxiv.org/abs/1311.3589}{arXiv:1311.3589}. \\
		\texttt{cttH},\texttt{cWWH} & See \href{https://arxiv.org/abs/1311.3589}{arXiv:1311.3589}. \\
		\hline
	\end{longtable}

\bottomheading{Extra}
	\begin{longtable}{p{1.5cm}p{12cm}}
		\hline
		\multicolumn{1}{c}{{\textbf{Section} \texttt{extra}}} & \multicolumn{1}{c}{{\textbf{Description}}} \\ 
		\hline
		{\tt debug} &
		A logical variable which can be used during a 
		debugging phase to mandate special behaviours. 
		Passed by common block {\tt common/debug/debug}. \\
		
		{\tt verbose} &
		A logical variable which can be used during a debugging phase to write 
		special information. Passed in common block {\tt common/verbose/verbose}. \\
		
		{\tt new\_pspace} &
		A logical variable which can be used during a debugging phase to test alternative versions of the phase space.
		Passed in common block {\tt common/new\_pspace/new\_pspace}. \\
		
		{\tt spira} & 
		A logical variable. If {\tt spira} is \texttt{.true.}, we calculate the 
		width of the Higgs boson by interpolating from a table
		calculated using the NLO code of M. Spira. The default value is \texttt{.true.}.
	        Otherwise the LO value valid for low Higgs masses only is used. \\
		
		{\tt noglue} &
		A logical variable. 
		The default value is \texttt{.false.}. If set to \texttt{.true.}, no processes
		involving initial gluons are included. \\
		{\tt ggonly} &
		A logical variable. 
		The default value is \texttt{.false.}. If set to \texttt{.true.}, 
		only the processes
		involving initial gluons in both hadrons are included.\\
		{\tt gqonly} &
		A logical variable. The default value is \texttt{.false.}. If set to \texttt{.true.}, 
		only the processes
		involving an initial gluon in one hadron and an initial quark
		or antiquark in the other hadron (or vice versa) are included.\\
		{\tt omitgg} &
		A logical variable. 
		The default value is \texttt{.false.}. If set to \texttt{.true.}, the gluon-gluon
		initial state is not included.\\
		
		{\tt clustering} &
		This logical parameter determines whether clustering is performed to yield
		jets. Only during a debugging phase should this variable be set to \texttt{.false.}. \\
		
		{\tt colourchoice} &
		If colourchoice=0, all colour structures are included ($W,Z+2$~jets).
		If colourchoice=1, only the leading 
		colour structure is included ($W,Z+2$~jets). \\
		
		{\tt rtsmin} &
		A minimum value of $\sqrt{s_{12}}$, which ensures that the invariant mass
		of the incoming partons can never be less than {\tt rtsmin}. \\
		
		
%		{\tt cutoff} & When performing calculations, the code implements a small cutoff
%		on all invariant masses in order to ensure numerical stability.  This is
%		performed according to \texttt{src/Need/smallnew.f} \\

                \texttt{reweight} & Flag to set the use of
		the user-implemented reweighting procedure \texttt{reweight\_user}
		in the routine \texttt{src/User/gencuts\_user.f90}.\\
    		\hline
	\end{longtable}

\bottomheading{Dipoles}
	\begin{longtable}{p{1.5cm}p{12cm}}
		\hline
		\multicolumn{1}{c}{{\textbf{Section} \texttt{dipoles}}} & \multicolumn{1}{c}{{\textbf{Description}}} \\ 
		\hline
		{\tt aii} &
		A double precision variable which can be used to
		limit the kinematic range for the subtraction of initial-initial dipoles
		as suggested by Trocsanyi and Nagy~\cite{Nagy:2003tz}.   
		The value {\tt aii=1} corresponds 
		to standard Catani-Seymour subtraction.\\
		{\tt aif} &
		A double precision variable which can be used to
		limit the kinematic range for the subtraction of initial-final dipoles
		as suggested by Trocsanyi and Nagy~\cite{Nagy:2003tz}.   
		The value {\tt afi=1} corresponds 
		to standard Catani-Seymour subtraction.\\
		{\tt afi} &
		A double precision variable which can be used to
		limit the kinematic range for the subtraction of final-initial dipoles
		as suggested by Trocsanyi and Nagy~\cite{Nagy:2003tz}.   
		The value {\tt afi=1} corresponds 
		to standard Catani-Seymour subtraction.\\
		{\tt aff} &
		A double precision variable which can be used to
		limit the kinematic range for the subtraction of final-final dipoles
		as suggested by Trocsanyi and Nagy~\cite{Nagy:2003tz}.   
		The value {\tt aff=1} corresponds 
		to standard Catani-Seymour subtraction.\\
		{\tt bfi} &
		A double precision variable which can be used to
		limit the kinematic range for the subtraction of final-initial dipoles
		in the photon fragmentation case.\\
		{\tt bff} &
		A double precision variable which can be used to
		limit the kinematic range for the subtraction of final-final dipoles
		in the photon fragmentation case.\\
		\hline
	\end{longtable}

