\topheading{Configuration}
\label{Input_parameters}

\midheading{Compile-time settings}
MCFM allows the user to choose between a number of schemes
for defining the electroweak couplings. These choices are summarized
in Table~\ref{ewscheme}. The scheme is selected by modifying the
value of {\tt ewscheme} in {\tt src/User/mdata.f} prior to compilation, 
which also contains
the values of all input parameters (see also Table~\ref{default}).

\begin{table}
\begin{center}
	\caption{Different options for the scheme used to fix the electroweak
		parameters of the Standard Model and the corresponding default input
		values. $M_W$ and $M_Z$ are taken from ref.~\cite{Amsler:2008zzb}.}
	\label{ewscheme}
	\vspace{0.5em}
\begin{tabular}{|c|c|c|c|c|c|c|} \hline
 Parameter & Name & Input Value
 & \multicolumn{4}{c|}{Output Value determined by \tt ewscheme} \\
\cline{4-7}
& ({\tt \_inp}) & & {\tt -1} & {\tt 0} & {\tt 1} & {\tt 2} \\ \hline
$G_F$            & {\tt Gf}      & 1.16639$\times$10$^{-5}$ 
 & input & calculated & input & input \\
$\alpha(M_Z)$    & {\tt aemmz}   & 1/128.89                 
 & input & input & calculated & input \\
$\sin^2 \theta_w$& {\tt xw}      & 0.2223               
 & calculated & input & calculated & input \\
$M_W$            & {\tt wmass}   & 80.385 GeV                
 & input & calculated & input & calculated \\
$M_Z$            & {\tt zmass}   & 91.1876 GeV               
 & input & input & input & calculated \\
$m_t$            & {\tt mt}      & {\tt input.ini}                  
 & calculated & input & input & input \\
\hline
\end{tabular}
\end{center}
\end{table}

The default scheme corresponds to {\tt ewscheme=+1}. As described below, this corresponds to a scheme
in which the top quark mass is an input parameter so that it is
more suitable for many processes now included in the program.

The choice of ({\tt ewscheme=-1}) enforces the use of an effective field
theory approach, which is valid for scales below the top mass. In this
approach there are 4 independent parameters (which we choose to be
$G_F$, $\alpha(M_Z)$, $M_W$ and $M_Z$). For further details,
see Georgi~\cite{Georgi:1991ci}.

For all the other schemes ({\tt ewscheme=0,1,2}) the top mass is simply
an additional input parameter and there are 3 other independent
parameters from the remaining 5. The variable {\tt ewscheme} then performs
exactly the same role as {\tt idef} in MadEvent~\cite{Maltoni:2002qb}.
{\tt ewscheme=0} is the old MadEvent default and {\tt ewscheme=1} is the
new MadEvent default, which is also the same as that used in 
Alpgen~\cite{Alpgen} and LUSIFER~\cite{Lusifer}. 
For processes in which the top quark is directly produced  it is 
preferable to use  the schemes ({\tt ewscheme=0,1,2}), since in these schemes
one can adjust the top mass to its physical value (in the input file
{\tt input.ini}). Schemes where the $W$ and $Z$ masses are fixed to their measured
values are the most appropriate for $W$ and $Z$ production processes.

\begin{table}
\begin{center}
	\caption{Default values for the remaining parameters in MCFM.
		$\Gamma_W$ and $\Gamma_Z$ from ref.~\cite{Amsler:2008zzb}.}
	\label{default} 
	\vspace{0.5em}
\begin{tabular}{|c|c|c|} \hline
Parameter & Fortran name & Default value \\ 
\hline
$m_\tau$         & {\tt mtau}      & 1.777 GeV            \\
$m^2_\tau$& {\tt mtausq}  & 3.1577 GeV$^2$     \\
$\Gamma_\tau$    & {\tt tauwidth}& 2.269$\times$10$^{-12}$~GeV \\
$\Gamma_W$       & {\tt wwidth}  & 2.093 GeV               \\
$\Gamma_Z$       & {\tt zwidth}  & 2.4952 GeV               \\
$V_{ud}$         & {\tt Vud}     & 0.975                  \\
$V_{us}$         & {\tt Vus}     & 0.222             \\
$V_{ub}$         & {\tt Vub}     & 0.                     \\
$V_{cd}$         & {\tt Vcd}     & 0.222             \\
$V_{cs}$         & {\tt Vcs}     & 0.975                  \\
$V_{cb}$         & {\tt Vcb}     & 0.                     \\
\hline
\end{tabular}

\end{center}
\end{table}

% I consider this setting dangerous, where is it used?

%In the same file ({\tt mdata.f}) one can also choose the definition
%that the program uses for computing transverse quantities, namely
%transverse momentum or transverse energy. These are defined by,
%\begin{eqnarray}
%\mbox{transverse momentum:} & \sqrt{p_x^2+p_y^2} \; ,\nonumber \\
%\mbox{transverse energy:}   &
% \frac{E \sqrt{p_x^2+p_y^2}}{\sqrt{p_x^2+p_y^2+p_z^2}} \; .
%\end{eqnarray}
%The two definitions of course coincide for massless particles.
%The chosen definition is used for all cuts that are applied to the
%process and it is the one that is used in the default set of histograms.

\midheading{Parton distributions}
\label{subsec:pdfsets}
The value of $\alpha_s(M_Z)$ is not adjustable; it is hardwired with the
parton distribution. In addition, the parton distribution also specifies
the number of loops that should be used in the running of $\alpha_s$.
As default the code uses the LHAPDF library for PDF evaluation; a native
implementation of some (mostly older) PDF sets is also retained.

\midheading{Electroweak corrections}
\label{subsec:EW}

As of version 8.1, {\tt MCFM} allows the calculation of weak corrections to a
selection of processes: {\tt 31} (neutral-current DY),
{\tt 157} (top-pair production) and {\tt 190} (di-jet production).
This is controlled by the flag {\tt ewcorr} in the input file.  A complete description
of the calculations is provided in Ref.~\cite{Campbell:2016dks}.

By setting {\tt ewcorr} to {\tt sudakov}, the program performs a calculation of
the leading weak corrections to these processes using a Sudakov approximation that
is appropriate at high energies.   The calculation of the weak corrections using the
exact form of the one-loop amplitudes is obtained by using the flag {\tt exact}.
A comparison between the two approaches, together with discussions of the validity of
the Sudakov approximation, may be found in Ref.~\cite{Campbell:2016dks}.

For the case of top-pair and di-jet production, the weak one-loop corrections contain
infrared divergences that must be cancelled against corresponding real radiation
contributions (in much the same manner as a regular NLO QCD calculation).  For this
reason the screen output will contain two sets of iterations corresponding to the
virtual and real contributions.

For all processes, performing the calculation of weak
corrections enables a special mode of phase-space integration that is designed to
better-sample events produced at high-energies.  For this reason the VEGAS output that
appears on the screen does not correspond to a physical cross-section -- and a corresponding
warning message to this effect will be displayed.  In many cases the quantity of most interest
is the relative correction to the leading order result ($\delta_{\mathrm{wk}}$) given by,

\begin{equation}
\delta_{\mathrm{wk}} = \frac{d\sigma_{\mathrm{wk}}^{NLO} - d\sigma^{LO}}{d\sigma^{LO}} \;.
\end{equation}

It is straightforward to compute this quantity for a distribution by editing the appropriate
{\tt nplotter} routine.
This is achieved by filling a histogram with the weight corresponding
to the LO result, another with the weight for the NLO weak result and then an additional placeholder
histogram that contains the special string {\tt '+RELEW+'}.  Examples of the syntax and correct calling
sequence can be seen in the code. (The appropriate nplotter routine is displayed on the process web-page, reachable from 
the tables in Section~\ref{MCFMprocs}.)



% \midheading{Nuclear collisions}
% \label{sec:nucleus}
% 
% It is possible to specify nuclear collisions by choosing values
% of {\tt ih1} and/or {\tt ih2} above {\tt 1000d0}. In that case,
% the identity of the nucleus is specified by the atomic number
% and mass ($Z$ and $A$ respectively) as follows:
% \begin{equation}
% {\tt ih} = 1000Z+A.
% \end{equation}
% For example, to choose an incoming lead beam one would set
% {\tt ih1=+82207d0}, corresponding to $Z=82$ and $A=207$.
% When running the program, the value of {\tt sqrts} should also be
% changed. This must be done by hand and is not automatically taken
% care of by the
% program. The centre-of-mass energy is decreased by a factor of
% $\sqrt{Z/A}$ for each nuclear beam. 
% 
% The nucleon PDFs are calculated by applying the correction
% factors of EKS98~\cite{Eskola:1998df} on top of the PDF set that is selected.
% This construction simply corrects each parton distribution by
% a factor that depends on the value of $(x,\mu)$ in the event.
% This parametrization is limited to the region $\mu < 100$~GeV and
% any value above that threshold will instead default to $100$~GeV.
% 
% Note that the cross-section reported by the program at the end
% of the run is given per nucleon per beam. Therefore the
% appropriate factors of $A$ should be applied in order to obtain
% the total cross section.


