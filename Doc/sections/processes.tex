\topheading{Notes on specific processes}

\label{sec:specific}

The processes described in the file {\tt process.DAT} include appropriate boson decays when the parameter 
{\tt removebr} is set to {\tt .false.}. In many cases a more simple calculation can be performed by setting 
this parameter to {\tt .true.}, in which case these decays are not performed.
Technically the full calculation including the decays
is still performed but cuts are not performed on the decay products and the
branching ratio is divided out, thus yielding the cross section before decay.
In the notes below we indicate the simpler processes thus obtained. When running in
this mode, the parameter {\tt zerowidth} should be set to {\tt .true.} for consistency. However in certain 
circumstances, for the sake of comparison, it may be useful to run with it set to {\tt .false.}.

\midheading{$W$-boson production, processes 1,6}
\label{subsec:wboson}

These processes represent the production of a $W$ boson which subsequently
decays leptonically. The calculation may be performed at NLO.

When {\tt removebr} is true, the $W$ boson does not decay.

\midheading{EW corrections to $W$-boson production, processes 2,7}
\label{subsec:wbosonew}

These processes compute the electroweak corrections to the production of
a $W$ boson which subsequently decays leptonically.  If particle 5 is present
it should be interpreted as a photon.
The calculation must be performed at NLO.

\midheading{Photon-induced corrections to $W$-boson production, processes 3,8}
\label{subsec:wbosonphoton}

These processes compute the production of
a $W$ boson which subsequently decays leptonically through the reaction,
$q + \gamma \to e + \nu + q$.
The calculation must be performed at NLO.

\midheading{$W+$~jet production, processes 11,16}
\label{subsec:w1jet}

These processes represent the production of a $W$ boson which subsequently
decays leptonically, in association with a single jet.
The calculation may be performed at NLO.

When {\tt removebr} is true, the $W$ boson does not decay.

\midheading{$W+b$ production, processes 12,17}
\label{subsec:wb}

These processes represent the production of a $W$ boson which
subsequently decays leptonically, in association with a single bottom
quark, exploiting the weak transitions $c \to b$ and $u \to b$.
This is produced at leading order by an initial state which
contains a charm quark (or the CKM  suppressed $u$ quark) and a
gluon.  The effect of the bottom quark mass is included throughout the
calculation.  
For this case the CKM matrix elements $V_{cb}$ and $V_{ub}$,
(if they are equal to zero in the input data file, {\tt mdata.f})
are set equal to $0.041$ and $0.00347$ respectively. 
Otherwise the non-zero values specified in {\tt mdata.f} are used. 
The calculation of this process may
be performed at NLO.

When {\tt removebr} is true, the $W$ boson does not decay.

\midheading{$W+c$ production, processes 13,18}
\label{subsec:wc}

These processes represent the production of a $W$ boson which
subsequently decays leptonically, in association with a charm
quark. This is produced at leading order by an initial state which
contains a strange quark (or Cabibbo suppressed $d$ quark) and a
gluon.  The effect of the charm quark mass is included throughout the
calculation.  As of version 5.2, the calculation of this process may
be performed at NLO.

When {\tt removebr} is true, the $W$ boson does not decay.

\midheading{$W+c$ production ($m_c=0$), processes 14,19}
\label{subsec:wcmassless}

These processes are identical to {\tt 13} and {\tt 18} except for the fact
that the charm quark mass is neglected. The calculation can currently be
performed at LO only.

\midheading{$W+b{\bar b}$ production, processes 20,25}
\label{subsec:wbb}

These processes represent the production of a $W$ boson which subsequently
decays leptonically, in association with a $b{\bar b}$ pair. The effect of
the bottom quark mass is included throughout the calculation.  
Beginning with MCFM version 6.0 this calculation may be performed at NLO, thanks to
the incorporation of the virtual corrections from ref.~\cite{Badger:2010mg}.
When {\tt removebr} is true, the $W$ boson does not decay.

To select final states in which one of the $b$-quarks may be unobserved the
user can employ processes 401--408 instead (see section~\ref{subsec:wbbfilter}).
These processes use the same matrix
elements but make specific requirements on the kinematics of the $b$-quarks
and QCD radiation. 

\midheading{$W+b{\bar b}$ production ($m_b=0$), processes 21,26}
\label{subsec:wbbmassless}

These processes are identical to {\tt 20} and {\tt 25} except for the fact
that the bottom quark mass is neglected. This allows the calculation to be
performed up to NLO, with currently calculated virtual matrix elements. These 
processes run considerably faster than the corresponding processes with the mass
for the $b$ quark, (20,25). In circumstances where both $b$ quarks are at large 
transverse momentum, the inclusion of the mass for the $b$-quark is not mandatory
and a good estimate of the cross section may be obtained by using these processes.

When {\tt removebr} is true, the $W$ boson does not decay.

\midheading{$W+2$~jets production, processes 22,27}
\label{subsec:w2jets}

\begin{center}
[{\it For more details on this calculation, please see Refs.~\cite{Campbell:2002tg,Campbell:2003hd}}]
\end{center}

This process represents the production of a $W$ boson and $2$ jets,
where the $W$ boson decays leptonically. The calculation may be
performed up to NLO, as detailed below. Virtual amplitudes are
taken from ref.~\cite{Bern:1997sc}.

For these processes (and also for $Z+2$~jet production, {\tt nproc=44,46})
the next-to-leading order matrix elements are
particularly complex and so they have been divided into two groups.
The division is according to the lowest order diagrams from which they
originate:
\begin{enumerate}
\item Diagrams involving two external quark lines and two external gluons,
the ``{\tt Gflag}'' contribution. The real diagrams in this case thus
involve three external gluons.

\item Diagrams where all four external lines are quarks,
the ``{\tt Qflag}'' contribution. The real diagrams in this case 
involve only one gluon.
\end{enumerate}

By specifying {\tt Gflag} and {\tt Qflag} in the file {\tt input.ini} one may
select one of these options at a time. The full result may be obtained
by straightforward addition of the two individual pieces, with no
meaning attached to either piece separately. 
Both of these may be set to {\tt .true.} simultaneously, however this
may result in lengthy run-times for sufficient convergence of the integral.

When {\tt removebr} is true, the $W$ boson does not decay.

\midheading{$W+3$~jets production, processes 23,28}
\label{subsec:w3jets}

This process represents the production of a $W$ boson and $3$ jets,
where the $W$ boson decays leptonically. The calculation may be
performed at LO only.

When {\tt removebr} is true, the $W$ boson does not decay.

\midheading{$W+b{\bar b}+$~jet production ($m_b=0$), processes 24,29}
\label{subsec:wbbjetmassless}

These processes represent the production of a $W$ boson which subsequently
decays leptonically, in association with a $b{\bar b}$ pair and an
additional jet. The effect of the bottom quark mass is neglected throughout
and the calculation may be performed at LO only.

When {\tt removebr} is true, the $W$ boson does not decay.

\midheading{$Z$-boson production, processes 31--33}
\label{subsec:zboson}

These processes represent the production of a $Z$ boson which subsequently
decays either into electrons ({\tt nproc=31}), neutrinos ({\tt nproc=32})
or bottom quarks ({\tt nproc=33}). Where appropriate, the effect of a virtual
photon is also included. As noted above, in these latter cases {\tt m34min > 0}
is obligatory. The calculation may be performed at NLO,
although the NLO calculation of process {\tt 33} does not include radiation
from the bottom quarks (i.e.\ radiation occurs in the initial state only).

When {\tt removebr} is true in process {\tt 31}, the $Z$ boson does not decay.

\midheading{$Z$-boson production decaying to jets, processes 34--35}
Radiation from the final state quarks is not included in this process.

\midheading{$t \bar{t}$ production mediated by $Z/\gamma^*$-boson exchange, process 36}

These processes represent the production of a virtual $Z$ boson or photon 
which subsequently decays into $t \bar{t}$.
The leptonic decays of the top quarks are included.
Switching {\tt zerowidth} from {\tt .true.} to {\tt .false.} only affects
the $W$ bosons from the top quark decay.
Note that {\tt m34min > 0} is obligatory due to the inclusion of the
virtual photon diagrams. The calculation may be only be performed at LO.

\midheading{Lepton pair production through photonic initial states, process 310}
\label{subsec:gg2lep}

This process represents the production of a lepton pair through an electroweak
process involving two photons in the initial state, $\gamma\gamma \to e^- e^+$.


\midheading{$Z+$~jet production, processes 41--43}
\label{subsec:zjet}

These processes represent the production of a $Z$ boson and a single jet,
where the $Z$ subsequently
decays either into electrons ({\tt nproc=41}), neutrinos ({\tt nproc=42})
or bottom quarks ({\tt nproc=43}). Where appropriate, the effect of a virtual
photon is also included. The calculation may be performed at NLO,
although the NLO calculation of process {\tt 43} does not include radiation
from the bottom quarks.

When {\tt removebr} is true in process {\tt 41}, the $Z$ boson does not decay.

\midheading{$Z+2$~jets production, processes 44, 46}
\label{subsec:z2jets}

\begin{center}
[{\it For more details on this calculation, please see Refs.~\cite{Campbell:2002tg,Campbell:2003hd}}]
\end{center}

These processes represents the production of a $Z$ boson and $2$ jets,
including also the effect of a virtual photon ({\tt nproc=44} only). The $Z/\gamma^*$ decays
to an $e^+ e^-$ pair ({\tt nproc=44}) or into three species of neutrino ({\tt nproc=46}).
The calculation may be performed up to NLO --
please see the earlier Section~\ref{subsec:w2jets} for more details,
especially the discussion regarding {\tt Qflag} and {\tt Gflag}.
As of version 6.0, both of these may be set to {\tt .true.} simultaneously but this
may result in lengthy run-times for sufficient convergence of the integral.
Virtual amplitudes are taken from ref.~\cite{Bern:1997sc}.

When {\tt removebr} is true, the $Z$ boson does not decay.


\midheading{$Z+3$~jets production, processes 45, 47}
\label{subsec:z3jets}

These processes represent the production of a $Z$ boson and $3$ jets,
including also the effect of a virtual photon ({\tt nproc=45} only). The $Z/\gamma^*$ decays
to an $e^+ e^-$ pair ({\tt nproc=45}) or into three species of neutrino ({\tt nproc=47}).
The calculation may be performed at LO only.

When {\tt removebr} is true, the $Z$ boson does not decay.

\midheading{$Z+b{\bar b}$ production, process 50}
\label{subsec:zbb}

These processes represent the production of a $Z$ boson (or virtual photon)
which subsequently decays leptonically, in association
with a $b{\bar b}$ pair. The effect of
the bottom quark mass is included throughout the calculation.  
The calculation may be performed at LO only.

When {\tt removebr} is true, the $Z$ boson does not decay.

\midheading{$Z+b{\bar b}$ production ($m_b=0$), processes 51--53}
\label{subsec:zbbmassless}

Process {\tt 51} is identical to {\tt 50} except for the fact
that the bottom quark mass is neglected. This allows the calculation to be
performed up to NLO. The other processes account for the decays into
neutrinos ({\tt nproc=52}) and bottom quarks ({\tt nproc=53}). Note that
the NLO calculation of process {\tt 53} does not currently 
include radiation from the
bottom quarks produced in the decay.

When {\tt removebr} is true in process {\tt 51}, the $Z$ boson does not decay.

\midheading{$Z+b{\bar b}+$~jet production ($m_b=0$), process 54}
\label{subsec:zbbjetmassless}

This process represents the production of a $Z$ boson (and virtual photon)
which subsequently decays leptonically, in association
with a $b{\bar b}$ pair and an additional jet.
The effect of the bottom quark mass is neglected throughout
and the calculation may be performed at LO only.

When {\tt removebr} is true, the $Z$ boson does not decay.

\midheading{$Z+c{\bar c}$ production ($m_c=0$), process 56}
\label{subsec:zccmassless}

Process {\tt 56} is the equivalent of {\tt 51}, with the bottom quarks
replaced by charm. Although the charm mass is neglected, the calculation
contains diagrams with two gluons in the initial state and a
$Z$ coupling to the heavy quark line -- hence the dependence upon the quark
flavour.

When {\tt removebr} is true in process {\tt 56}, the $Z$ boson does not decay.

\midheading{Di-boson production, processes 61--89}
\label{subsec:diboson}

\begin{center}
[{\it For more details on this calculation, please see Refs.~\cite{Campbell:1999ah,Campbell:2011bn}}]
\end{center}

These processes represent the production of a diboson pair $V_1 V_2$,
where $V_1$ and $V_2$ may be either a $W$ or $Z/\gamma^*$. 
All the processes in this section may be calculated at NLO, with the exception
of {\tt nproc=66,69}. There are various
possibilities for the subsequent decay of the bosons, as specified in the
sections below. Amplitudes are taken from ref.~\cite{Dixon:1998py}.
Where appropriate, these processes include glue-glue initiated box diagrams
which first contribute at order $\alpha_s^2$ but are included here in the
NLO calculation. We also include singly resonant diagrams at NLO for all processes
in the case {\tt zerowidth = .false.}.

For processes {\tt 62}, {\tt 63}, {\tt 64}, {\tt 65}, {\tt 74}
and {\tt 75} the default behaviour is that the hadronic decay products
of the bosons are clustered into jets using the supplied jet
algorithm parameters, but no cut is applied on the number of jets.
This behaviour can be altered by changing the value of the
variable {\tt notag} in the file {\tt src/User/setnotag.f}.
 
\bottomheading{$WW$ production, processes 61-64, 69}

For $WW$ production, both $W$'s can decay leptonically ({\tt nproc=61}) or one
may decay hadronically ({\tt nproc=62} for $W^-$ and {\tt nproc=64} for $W^+$).
Corresponding to processes {\tt 62,64}, processes {\tt 63,65} implement radiation in 
decay from the hadronically decaying W's.
Process {\tt 69} implements the matrix elements for the leptonic decay of
both $W$'s but where no polarization information is retained. It is included
for the sake of comparison with other calculations.
Processes {\tt 62} and {\tt 64} may be run at NLO with the option {\tt todk},
including radiation in the decay of the hadronically decaying $W$.
Processes {\tt 63} and {\tt 65} give the effect of radiation in the decay alone
by making the choices {\tt virt},  {\tt real} or {\tt tota}.

Note that, in processes
{\tt 62} and {\tt 64}, the NLO corrections include radiation from the
hadronic decays of the $W$.

The NLO calculations include contributions from the process $gg \to WW$
that proceeds through quark loops. The calculation of loops containing the third quark generation
includes the effect of the top quark mass (but $m_b=0$), while the first two
generations are considered massless. For numerical stability, a small cut on the
transverse momentum of the $W$ bosons is applied: $p_T(W)>0.05$~GeV for loops
containing massless (first or second generation) quarks, $p_T(W)>2$~GeV for $(t,b)$
loops. This typically removes less than $0.1$\% of the total cross section. The
values of these cutoffs can be changed by editing ${\tt src/WW/gg\_WW.f}$ and recompiling.

When {\tt removebr} is true in processes {\tt 61} and {\tt 69},
the $W$ bosons do not decay.

\bottomheading{$WW$+jet production, process 66}

This process is only implemented for the leptonic decay modes of both $W$
bosons and is currently limited to LO accuracy only. When {\tt removebr} is true,
the $W$ bosons do not decay.

\bottomheading{$WZ$ production, processes 71--80}

For $WZ$ production, the $W$ is chosen to decay leptonically. The $Z$ (or
virtual photon, when appropriate) may decay into electrons
({\tt nproc=71},{\tt 76}), neutrinos ({\tt nproc=72},{\tt 77}), a
pair of bottom quarks ({\tt nproc=73},{\tt 78}), three generations of down-type
quarks ({\tt nproc=74},{\tt 79}) or two generations of up-type quarks ({\tt nproc=75},{\tt 80}).
In process {\tt 78} the mass of the $b$-quark is neglected.
These processes will be observed
in the final state as $W$-boson + two or three jets.
In processes {\tt 72} and {\tt 77}, a sum is performed over all three species of neutrinos.

When {\tt removebr} is true in processes {\tt 71} and {\tt 76},
neither the $W$ or the $Z$ boson decays.

\bottomheading{$ZZ$ production, processes 81--84, 90}

The $Z$'s can either both decay leptonically ({\tt nproc=81}),
one can decay leptonically while the other decays into neutrinos
({\tt nproc=82}) or bottom quarks ({\tt nproc=83}), or
one decays into neutrinos and the other into a bottom quark pair
({\tt nproc=84}).
In process {\tt 83} the mass of the $b$-quark is neglected. Note that, in processes
{\tt 83}--{\tt 84}, the NLO corrections do not include
radiation from the bottom quarks that are produced by the $Z$ decay.
In process {\tt 90} the two $Z$ bosons decay to identical charged leptons,
and interference effects between the decay products of the two $Z$ bosons
are included.
In all cases these processes also include the contribution from a virtual photon .

The NLO calculations include contributions from the process $gg \to ZZ$
that proceeds through quark loops. The calculation of loops containing the third quark generation
includes the effect of both the top and the bottom quark mass ($m_t,m_b \neq 0$), while the first two
generations are considered massless. For numerical stability, a small cut on the
transverse momentum of the $Z$ bosons is applied: $p_T(Z)>0.1$~GeV.
This typically removes less than $0.1$\% of the total cross section. The
values of these cutoffs can be changed by editing {\tt src/ZZ/getggZZamps.f} 
and recompiling.

When {\tt removebr} is true in process {\tt 81}, neither of the $Z$ bosons decays.


\bottomheading{$ZZ$+jet production, process 85}

This process is only implemented for the case when one $Z$ boson decays to
electrons and the other to neutrinos (i.e. the companion of {\tt nproc=82}).
It may only be calculated at LO. When {\tt removebr} is true, the $Z$ bosons
do not decay.

\input{sections/WWZanom.tex}

\midheading{$WH$ production, processes 91-94, 96-99, 900}
\label{subsec:wh}

These processes represent the production of a $W$ boson which subsequently
decays leptonically, in association with a Standard Model Higgs boson that
decays into a bottom quark pair ({\tt nproc=91, 96}), 
a pair of $W$ bosons ({\tt nproc=92, 97}), 
a pair of $Z$ bosons ({\tt nproc=93, 98}), or a pair of photons ({\tt nproc=94, 99}).  
Note that in the cases of Higgs decay to $W$,($Z$) pairs, 
below the $W$,($Z$) pair threshold
one of the $W$,($Z$) bosons is virtual 
and therefore one must set {\tt zerowidth=.false.}.
The calculation may be performed at NLO.
Note that the bottom quarks are considered massless and radiation from the
bottom quarks in the decay is not included.
\texttt{nproc=900} may be used to compute the sum over both W charges in
one run (with the decay products 3 and 4 representing lepton and antilepton
respectively).

When {\tt removebr} is true, neither the $W$ boson nor the Higgs decays.

\midheading{$ZH$ production, processes 101--109}
\label{subsec:zh}

These processes represent the production of a $Z$ boson (or virtual photon)
in association with a Standard Model Higgs boson that
decays into a bottom quark pair ({\tt nproc=101-103}),
or decays into a pair of photons ({\tt nproc=104-105}) 
or a pair of $W$ bosons ({\tt nproc=106-108}),
or a pair of $Z$ bosons ({\tt nproc=109}). 
The $Z$ subsequently decays into 
either an $e^+ e^-$ pair ({\tt nproc=101, 106, 109}), neutrinos ({\tt nproc=102, 107})
or a bottom quark pair ({\tt nproc=103, 108}).
The calculation may be performed at NLO, although radiation from the
bottom quarks in the decay of the Higgs (or the $Z$, for processes
{\tt 103, 108}) is not included.

When {\tt removebr} is true in processes {\tt 101, 106, 109}, neither the $Z$ boson
nor the Higgs decays.

\midheading{Higgs production, processes 111--121}
\label{subsec:h}

These processes represent the production of a Standard Model Higgs
boson that decays either into a bottom quark
pair ({\tt nproc=111}), a pair of tau's ({\tt nproc=112}), 
a $W^+W^-$ 
pair that further decays leptonically ({\tt nproc=113}) 
a $W^+W^-$ pair where the $W^-$ decays hadronically ({\tt nproc=114,115}) 
or a $ZZ$ pair ({\tt nproc=116-118}) . In addition, the loop-level decays of the Higgs 
into a pair of photons ({\tt nproc=119}) and the $Z\gamma$ decay are included
({\tt nproc=120,121}).

For the case of $W^+W^-$ process {\tt nproc=115} gives the contribution 
of radiation from the hadronically decaying $W^-$.
Process {\tt 114} may be run at NLO with the option {\tt todk},
including radiation in the decay of the hadronically decaying $W^-$.~\footnote{
We have not included the case of a hadronically decaying $W^+$; it can
be obtained from processes {\tt nproc=114,115} by performing the
substitutions $\nu \to e^-$ and $e^+ \to \bar{\nu}$.}
For the case of a $ZZ$ decay,
the subsequent decays can either be into a pair of muons and a pair of electrons
({\tt nproc=116)}, a pair of muons and neutrinos ({\tt nproc=117}) or
a pair of muons and a pair of bottom quarks ({\tt nproc=118}).

At LO the relevant diagram
is the coupling of two gluons to the Higgs via a top quark loop.
This calculation is performed in the limit of infinite top quark mass, so that 
the top quark loop is replaced by an effective operator. This corresponds
to the effective Lagrangian,
\begin{equation}
\mathcal{L} = \frac{1}{12\pi v} \, G^a_{\mu\nu} G^{\mu\nu}_a H \;,
\label{eq:HeffL}
\end{equation}
where $v$ is the Higgs vacuum expectation value and $G^a_{\mu\nu}$ the
gluon field strength tensor.
The calculation may be performed at NLO, although radiation from the
bottom quarks in the decay of processes {\tt 111} and {\tt 118} is not yet included.

%At the end of the output the program will also display the cross section rescaled
%by the constant factor,
%\begin{equation}
%\frac{\sigma_{\rm LO}(gg \to H, \mbox{finite}~m_t)}{\sigma_{\rm LO}(gg \to H, m_t \to \infty)} \;.
%\label{eqn:hrescale}
%\end{equation}
%For the LO calculation this gives the exact result when retaining a finite value for $m_t$,
%but this is only an approximation at NLO. The output histograms are not rescaled in this way.

When {\tt removebr} is true in processes {\tt 111,112,113,118},
the Higgs boson does not decay.

Process {\tt 119} implements the decay of the Higgs boson into two photons
via loops of top quarks and $W$-bosons.
The decay is implemented using the formula Eq.(11.12) from ref.~\cite{Ellis:1991qj}.
When {\tt removebr} is true in process {\tt 119} the Higgs boson does not decay.

Processes {\tt 120} and {\tt 121} implement the decay of the Higgs boson into an lepton-antilepton
pair and a photon. As usual the production of a charged lepton-antilepton pair is mediated by a 
$Z/\gamma^*$ (process {\tt 120}) and the production of three types of neutrinos 
$\sum  \nu \bar{\nu}$ by a $Z$-boson (process {\tt 121}). These processes are implemented 
using a generalization of the formula of \cite{Djouadi:1996yq}. (Generalization to take into
account off-shell $Z$-boson and adjustment of the sign of $C_2$ in their Eq.(4)).


\midheading{$H \to W^+W^-$ production, processes 123-126}
These processes represent the production of a Higgs boson that decays to $W^+ W^-$,
with subsequent decay into leptons. For process {\tt 123}, the exact form of the triangle
loop coupling a Higgs boson to two gluons is included, with both top and bottom quarks
circulating in the loop. This is to be contrasted with process {\tt 113} in which only the
top quark contribution is included in the effective coupling approach.

Process {\tt 124} includes only the effect of the interference of the
Higgs and $gg \to W^+W^-$ amplitudes, as described in ref.~\cite{Campbell:2011cu}.
The calculation is available at LO only. LO corresponds to $O(\alpha_s^2)$ in this case.
The calculation of loops containing the third quark generation
includes the effect of the top quark mass (but $m_b=0$), while the first two
generations are considered massless. For numerical stability, a small cut on the
transverse momentum of the $W$ bosons is applied: $p_T(W)>0.05$~GeV for loops
containing massless (first or second generation) quarks, $p_T(W)>2$~GeV for $(t,b)$
loops. This typically removes less than $0.1$\% of the cross section. The
values of these cutoffs can be changed by editing ${\tt src/HWW/gg\_WW\_int.f}$ and recompiling.

Process {\tt 125} includes all $gg$-intitiated diagrams that have a Higgs boson in the $s$-channel,
namely the square of the $s$-channel Higgs boson production and the interference with the diagrams
that do not contain a Higgs boson, (i.e. $gg \to W^+W^- \to \nu_e e^+ e^- \bar{\nu_e}$).

The result for the square of the box diagrams alone, i.e. the process
$gg \to W^+W^- \to \nu_e e^+ e^- \bar{\nu_e}$, may be obtained by running process
{\tt nproc=61} with {\tt part=virt} and {\tt ggonly=.true.} 

Process {\tt 126} calculates the full result for this process from  $gg$-intitiated diagrams.
This includes diagrams that have a Higgs boson in the $s$-channel, the continuum $W^+W^-$
diagrams described above and their interference. 


\midheading{$H \to ZZ \to e^- e^+ \mu^- \mu^+$ production, processes 128-133}
These processes represent the production of a Higgs boson that decays to $Z Z$,
with subsequent decay into charged leptons. For process {\tt 128}, the exact form of the triangle
loop coupling a Higgs boson to two gluons is included, with both top and bottom quarks
circulating in the loop. This is to be contrasted with process {\tt 116} in which only the
top quark contribution is included in the effective coupling approach.

Process {\tt 129} includes only the effect of the interference of the
Higgs and $gg \to ZZ$ amplitudes.
The calculation is available at LO only. LO corresponds to $O(\alpha_s^2)$ in this case.
The calculation of loops containing the third quark generation
includes the effect of both the top quark mass and the bottom quark, while the first two
generations are considered massless. For numerical stability, a small cut on the
transverse momentum of the $Z$ bosons is applied: $p_T(Z)>0.05$~GeV.
This typically removes less than $0.1$\% of the cross section. The
values of these cutoffs can be changed by editing ${\tt src/ZZ/getggZZamps.f}$ and recompiling.

Process {\tt 130} includes all $gg$-intitiated diagrams that have a Higgs boson in the $s$-channel,
namely the square of the $s$-channel Higgs boson production and the interference with the diagrams
that do not contain a Higgs boson, (i.e. $gg \to Z/\gamma^*+Z/\gamma^* \to e^- e^+ \mu^- \mu^+$).

Process {\tt 131} calculates the full result for this process from  $gg$-intitiated diagrams.
This includes diagrams that have a Higgs boson in the $s$-channel, the continuum $ Z/\gamma^*+Z/\gamma^*$
diagrams described above and their interference. 

Process {\tt 132}  gives the result for the square of the box diagrams alone, i.e. the process
$gg \to Z/\gamma^*+Z/\gamma^* \to e^- e^+ \mu^- \mu^+$.

Process {\tt 133} calculates the interference for the $qg$ initiated process.

For those processes that include contributions from the Higgs boson, the form
of the Higgs propagator may be changed by editing the file
{\tt src/Need/sethparams.f}.  If the logical variable {\tt CPscheme} is
changed from the default value {\tt .false.} to {\tt .true.} then the
Higgs propagator is computed using the ``bar-scheme'' that is
implemented in the HTO code of G. Passarino~\cite{Goria:2011wa,Passarino:2010qk}.
The value of the Higgs boson width has been computed with v1.1 of the
HTO code, for Higgs masses in the interval $50 < m_H< 1500$~GeV.  These
values are tabulated, in $0.5$~GeV increments, in the file
{\tt Bin/hto\_output.dat}.  The widths for other masses in this range
are obtained by linear interpolation.

\bottomheading{Specifying other final states}
\label{specifyingZdecays}
As described above, these processes refer to a final state 
$e^- e^+ \mu^- \mu^+$.  It is however possible to specify a final state
that corresponds to a different set of $Z$ boson decays.  This is achieved
by altering the value of {\tt NPROC} in the input file by appending a
period, followed by two 2-character strings that identify each of the decays.
Possible values for the strings, and the corresponding decays, are
shown in the table below.
\begin{center}
\begin{tabular}{ll}
string & $Z$ decay \\
\hline
{\tt el,EL} & $(e^-,e^+)$ \\
{\tt mu,MU,ml,ML} & $(\mu^-,\mu^+)$ \\
{\tt tl,TL} & $(\tau^-,\tau^+)$ \\
{\tt nu,NU,nl,NL} & $(\nu,\bar\nu) \times 3$ \\
{\tt bq,BQ} & $(b,\bar b)$ \\
\end{tabular}
\end{center}
Note that, for the case of neutrino decays, the sum over three flavours of
neutrino is performed.  The labelling of the particles in the output is best
understood by example.  Setting {\tt nproc=132.ELNU} corresponds to the
process $gg \to Z/\gamma^*+Z/\gamma^* \to e^-(p_3) e^+(p_4) \nu(p_5) \bar\nu(p_6)$.
Note that the default process corresponds to the string {\tt ELMU} so that,
for instance {\tt nproc=132.ELMU} is entirely equivalent to
{\tt nproc=132}.
The effect of changing the lepton flavour is only seen in the output
of LHE events, where the correct mass is then used when producing the
event record.

\midheading{$e^- e^+ \nu_e \bar \nu_e$ production, processes 1281, 1311, 1321}
These processes compute cross sections relevant for the final state
$e^- e^+ \nu_e \bar \nu_e$, i.e. with charged leptons and neutrinos from the
same (electron) doublet.  As a result they receive contributions from diagrams
with resonant $ZZ$ propagators and resonant $WW$ propagators.  Process
{\tt 1281} only includes amplitudes containing a Higgs boson (c.f. processes {\tt 123} and {\tt 128}).
Process {\tt 1321} only includes continuum (box-diagram) amplitudes (c.f. processes {\tt 127} and {\tt 132}).
Process {\tt 1311} includes both amplitudes and the effects of the interference between them (c.f. processes {\tt 126} and {\tt 131}).
The effect of the interference between the $WW$ and $ZZ$ diagrams can be assessed by, for instance, comparing process {\tt 1281} with the sum of processes
{\tt 123} and one-third of {\tt 128.ELNU}, where the weighting is to divide out the natural sum over three neutrino flavours in process {\tt 128.ELNU}.

Event generation is not available for these processes at present.



\midheading{$e^- e^+ \nu \bar \nu$ production, processes 1282, 1312, 1322}
These processes compute cross sections relevant for the final state
$e^- e^+ \nu \bar \nu$, i.e. an electron pair and a sum over all three
flavours of neutrino.  For muon and tau neutrinos, only $ZZ$ diagrams
contribute.  For electron neutrinos there are contributions from diagrams
with resonant $ZZ$ propagators and resonant $WW$ propagators.  Process
{\tt 1282} only includes amplitudes containing a Higgs boson (c.f. processes
{\tt 123} and {\tt 128}). Process {\tt 1322} only includes continuum (box-diagram) amplitudes (c.f. processes {\tt 127} and {\tt 132}).
Process {\tt 1312} includes both amplitudes and the effects of the interference between them (c.f. processes {\tt 126} and {\tt 131})s.
The effect of the interference between the $WW$ and $ZZ$ diagrams can be assessed by, for instance, comparing process {\tt 1282} with the sum of processes {\tt 123} and {\tt 128.ELNU}.

Event generation is not available for these processes at present.


\midheading{$H+b$ production, processes 136--138}
\label{subsec:Hb}

\begin{center}
[{\it For more details on this calculation, please see Ref.~\cite{Campbell:2002zm}}]
\end{center}

These processes represent the production of a Standard Model Higgs
boson that decays into a pair of bottom quarks,
in association with a further bottom quark. The initial state at lowest order
is a bottom quark and a gluon.
The calculation may be performed at NLO, although radiation from the
bottom quarks in the Higgs decay is not included.

For this process, the matrix elements are divided up into a number of
different sub-processes, so the user must sum over these after performing
more runs than usual. At lowest order one can proceed as normal, using
{\tt nproc=136}. For a NLO calculation, the sequence of runs is as follows:
\begin{itemize}
\item Run {\tt nproc=136} with {\tt part=virt} and {\tt part=real} (or, both
at the same time using {\tt part=tota});
\item Run {\tt nproc=137} with {\tt part=real}.
\end{itemize}
The sum of these yields the cross-section with one identified $b$-quark in
the final state. To calculate the contribution with two $b$-quarks in the
final state, one should use {\tt nproc=138} with {\tt part=real}.

When {\tt removebr} is true, the Higgs boson does not decay.


\midheading{$t\bar{t}$ production with 2 semi-leptonic decays, processes 141--145}
\label{subsec:ttbar}

These processes describe $t \bar{t}$ production including semi-leptonic
decays for both the top and the anti-top. 
In version 6.2 we have updated this to use the one-loop amplitudes of
ref.~\cite{Badger:2011yu}. The code for the virtual amplitudes now runs
about three times faster than earlier versions where the virtual
amplitudes of ref.~\cite{Korner:2002hy} were used.  
Switching {\tt zerowidth} from {\tt .true.} to {\tt .false.} only affects 
the $W$ bosons from the top quark decay, because our method of including spin
correlations requires the top quark to be on shell.

Process {\tt 141} includes all corrections, i.e.\ both radiative corrections
to the decay and to the production. This process is therefore the
basic process for the description of top production where both quarks
decay semi-leptonically.  When {\tt removebr} is true in process {\tt 141},
the top quarks do not decay.
When one wishes to calculate observables related to the decay of the top
quark, {\tt removebr} should be false in process {\tt 141}.
The LO calculation proceeds as normal. 
At NLO, there are two options:
\begin{itemize}
\item {\tt part=virt, real} or {\tt tota} : final state radiation is included
in the production stage only
\item {\tt part = todk} : radiation is included in the decay of the top
quark also and the final result corresponds to the sum of real and virtual
diagrams. 
Note that these runs automatically perform an extra integration, so
will take a little longer.
\end{itemize}

Process {\tt 142} includes only the corrections in
the semileptonic decay of the top quark. Thus it is of primary
interest for theoretical studies rather than for physics applications.  
Because of the method that we have used to include the radiation in the decay,
the inclusion of the corrections in the decay does not change the
total cross section. This feature is explained in section 6 of ref.~\cite{Campbell:2012uf}.

In the case of process {\tt 145}, there are no spin correlations in
the decay of the top quarks. The calculation is performed by
multiplying the spin summed top production cross section, by the decay
matrix element for the decay of the $t$ and the $\bar{t}$. These
processes may be used as a diagnostic test for the importance of the
spin correlation.



\midheading{$t\bar{t}$ production with decay and a gluon, process 143}
This process describes lowest order $t \bar{t}+g$ production 
including two leptonic decays $t \to b l \nu$. 
When {\tt removebr} is true, the top quarks do not decay.
This LO process only includes radiation only includes radiation in production.

\midheading{$t\bar{t}$ production with one hadronic decay, processes 146--151}

These processes describe the hadronic production of a pair of top
quarks, with one quark decaying hadronically and one quark decaying
semileptonics.  For processes {\tt 146--148}, the top quark decays
semileptonically whereas the anti-top quark decays hadronically.  For
processes {\tt 149--151}, the top quark decays hadronically whereas the
anti-top quark decays semi-leptonically.  The base processes for
physics are process {\tt 146} and {\tt 149} which include
radiative corrections in both production and decay.  Switching {\tt zerowidth} from 
{\tt .true.} to {\tt .false.} only affects the $W$ bosons from the top
quark decay, because our method of including spin correlations
requires the top quark to be on shell.
When one wishes to calculate observables related to the decay of the top
quark, {\tt removebr} should be false in processes {\tt 146} and {\tt 149}.
The LO calculation proceeds as normal. At NLO, there are two options:
\begin{itemize}
\item {\tt part=virt, real} or {\tt tota} : final state radiation is included
in the production stage only
\item {\tt part = todk} : radiation is included in the decay of the top
quark also and the final result corresponds to the sum of real and virtual
diagrams. 
Note that these runs automatically perform an extra integration, so
will take a little longer.
\end{itemize}


Processes {\tt 147} and {\tt 150} include only the radiative 
corrections in the decay of the top quark without including 
the radiative corrections in the hadronic decay of the $W$-boson.
Because of the method that we have used to include the radiation in the decays,
the inclusion of the corrections in this stage of the decay does not change the
total cross section.
Process {\tt 148} ({\tt 151}) includes only the radiative corrections
in the hadronic decay of the $W$-boson coming from the anti-top (top).
The inclusion of the corrections in this stage of the decay increases the
partial width by the normal $\alpha_s/\pi$ factor.


\midheading{$Q\overline{Q}$ production, processes 157--159}
These processes calculate the production of heavy quarks
({\tt 157} for top, {\tt 158} for bottom and {\tt 159} for charm) up to NLO 
using the matrix elements of ref.~\cite{Nason:1987xz}. No decays
are included.

\midheading{$t{\bar t}+$~jet production, process 160}
This process calculates the production of top quarks and a single jet
at LO, without any decay of the top quarks.


\midheading{Single-top-quark production and decay at NNLO, process 1610}
\label{single-top-quark-production-and-decay-at-nnlo}

This calculation is based on ref.~\cite{Campbell:2020fhf}. See also
ref.~\cite{Campbell:2021qgd} for the role of double-DIS scales and the
relevancy for PDFs.

This process can be run by using process number 1610. The resulting
histograms and cross-sections are printed for a strict fixed-order
expansion as well as for a naive addition of all contributions. The
fixed-order expansion assembles pieces according to the following
formula. Please see ref.~\cite{Campbell:2020fhf} for more details.

\input{sections/equation_twidth.tex}

At each order a corresponding top-decay width is used throughout all
parts. The NNLO width is obtained from ref.~\cite{Blokland:2005vq} and at
LO and NLO from ref.~\cite{Czarnecki:1990kv}. These widths agree with
numerical results obtained from our calculation of course.

This process can be run with a fixed scale or with dynamic DIS (DDIS)
scales by setting \texttt{dynamicscale\ =\ DDIS},
\texttt{renscale\ =\ 1.0} and \texttt{facscale\ =\ 1.0}.

At NNLO there are several different contributions from vertex
corrections on the light-quark line, heavy-quark line in production, and
heavy-quark line in the top-quark decay. Additionally there are one-loop
times one-loop interference contributions between all three
contributions. These contributions can be separately enabled in the
\texttt{singletop} block:

\begin{verbatim}
[singletop]
    nnlo_enable_light = .true.
    nnlo_enable_heavy_prod = .true.
    nnlo_enable_heavy_decay = .true.
    nnlo_enable_interf_lxh = .true.
    nnlo_enable_interf_lxd = .true.
    nnlo_enable_interf_hxd = .true.
    nnlo_fully_inclusive = .false.
\end{verbatim}

For a fully inclusive calculation without decay the last setting has to
be set to \texttt{.true.} and the decay and decay interference parts
have to be removed. Additionally jet requirements must be lifted, see
below.

When scale variation is enabled with DDIS scales then automatically also
a variation around the fixed scale \(\mu=m_t\) is calculated for
comparison.

This process uses a fixed diagonal CKM matrix with
\(V_{ud}=V_{cs}=V_{tb}=1\). The setting \texttt{removebr=.true.} removes
the \(W\to \nu e\) branching ratio.

This process involves complicated phase-space integrals and we have
pre-set the initial integration calls for precise differential
cross-sections with fiducial cuts. The number of calls can be tuned
overall with the multiplier setting
\texttt{integration\%globalcallmult}. For total fully inclusive
cross-sections the number of calls can be reduced by a factor of ten by
setting \texttt{integration\%globalcallmult\ =\ 0.1}, for example.

For scale variation uncertainties and PDF uncertainties we recommend to
start with the default number of calls and a larger number of warmup
iterations \texttt{integration\%iterbatchwarmup=10}, for example. For
the warmup grid no scale variation or PDF uncertainties are calculated
and this ensures a good Vegas integration grid that can be calculated
fast. The setting \texttt{integration\%callboost} modifies the number of
calls for subsequent integration iterations after the warmup. For
example setting it to \texttt{0.1} reduces the calls by a factor of ten.
This is typically enough to compute the correlated uncertainties for a
previously precisely determined central value.

At NNLO the default value for \(\tau_{\text{cut}}\) is \(10^{-3}\), which
is the value used for all the plots in our publication. We find that
cutoff effects are negligible at the sub-permille level for this choice.
We strongly recommend to not change this value.

\paragraph{Using the plotting routine with b-quark
tagging}\label{using-the-plotting-routine-with-b-quark-tagging}

The calculation has been set up with b-quark tagging capabilities that
can be accessed in both the \texttt{gencuts\_user.f90} routine and the
plotting routine \texttt{nplotter\_singletop\_new.f90}. The plotting
routine is prepared to generate all histograms shown in our publication
in ref.~\cite{Campbell:2020fhf}. By default the top-quark is
reconstructed using the leading b-quark jet and the exact W-boson
momentum, but any reconstruction algorithm can easily be implemented.

We have added the \texttt{gencuts\_user.f90} file as used for the plots
in our paper~\cite{Campbell:2020fhf} in \\
\texttt{src/User/docs/gencuts\_user\_singletop\_nnlo.f90} that can be
used as a guide on how to access the b-quark tagging in the
\texttt{gencuts\_user} routine.

See also \texttt{nplotter\_ktopanom.f} (used for the NLO off-shell
calculation in ref.~\cite{Neumann:2019kvk} for a reconstruction of the
W-boson. It is based on requiring an on-shell W-boson and selecting the
solution for the neutrino z-component that gives the closest on-shell
top-quark mass by adding the leading b-quark jet.

\paragraph{Calculating fully inclusive
cross-sections}\label{calculating-fully-inclusive-cross-sections}

When calculating a fully inclusive cross-section without top-quark decay
please set \texttt{zerowidth\ =\ .true.},
\texttt{removebr\ =\ .true.} in the general section of the input file;
\texttt{inclusive\ =\ .true}, \texttt{ptjetmin\ =\ 0.0},
\texttt{etajetmax\ =\ 99.0} in the basicjets section;
\texttt{makecuts\ =\ .false.} in the cuts section; also set
\texttt{nnlo\_enable\_heavy\_decay\ =\ .false.} and
\texttt{nnlo\_enable\_interf\_lxd\ =\ .false.},
\texttt{nnlo\_enable\_interf\_hxd\ =\ .false.} and
\texttt{nnlo\_fully\_inclusive\ =\ .true.} in the singletop section.

These settings ensure that neither the decay nor any production times
decay interference contributions are included. The last setting makes
sure that only the right pieces in the fixed-order expansion of the
cross-section are included. It also ensures that the b-quark from the
top-quark decay is not jet-tagged and just integrated over.

\paragraph{Notes on runtimes and demo files}\label{notes-on-runtimes-and-demo-files}

Running the provided input file \\
\texttt{input\_singletop\_nnlo\_Tevatron\_total.ini} with
-integration\%globalcallmult=0.1 and without histograms takes about 4-5
CPU days. So depending on the number of cores, this can be run on a
single desktop within a few hours.

Running the input file \\
\texttt{input\_singletop\_nnlo\_LHC\_fiducial.ini} with the default set
of calls and histograms takes about 3 CPU months (about 3 wall-time
hours on our cluster with 45 nodes). For the fiducial cross-section
(without precise histograms) a setting of
\texttt{-integration\%globallcallmult=0.2} can also be used.

Note that \texttt{-extra\%nohistograms\ =\ .true.} has been set in these
demonstration files, so no further histograms from
\texttt{nplotter\_singletop\_new.f90} are generated.

The input file \texttt{input\_singletop\_nnlo\_LHC\_fiducial.ini}
together with the file \\
\texttt{src/User/docs/gencuts\_user\_singletop\_nnlo.f90} replacing
\texttt{src/User/gencuts\_user.f90} reproduces the fiducial
cross-sections in ref.~\cite{Campbell:2020fhf} table 6.


\midheading{Single top production, processes 161--177}
\label{subsec:stop}

\begin{center}
[{\it For more details on this calculation, please see Ref.~\cite{Campbell:2004ch}}]
\end{center}

These processes represent single top production and may be calculated up to
NLO as described below.

Single top production is divided as usual into $s$-channel 
(processes {\tt 171-177}) and $t$-channel ({\tt 161-167})
diagrams. Each channel includes separately the production of a top
and anti-top quark, which is necessary when calculating rates at the LHC.
Below we illustrate the different use of these processes by considering
$t$-channel top production ({\tt 161,162}), although the procedure is the same
for anti-top production ({\tt 166,167}) and the corresponding $s$-channel
processes ({\tt 171,172}) and ({\tt 176,177}).


To calculate cross-sections that do not include any decay of the (anti-)top
quark, one should use process {\tt 161}
(or, correspondingly, {\tt 166}, {\tt 171} and {\tt 176}) with {\tt removebr}
true. The procedure is exactly the same
as for any other process.
Switching {\tt zerowidth} from {\tt .true.} to {\tt .false.} only affects
the $W$ boson from the top quark decay.

For processes {\tt 161}, {\tt 162}, {\tt 163}, {\tt 166}, {\tt 167}
and {\tt 168} the default behaviour when {\tt removebr} is true is that
partons are clustered into jets using the supplied jet
algorithm parameters, but no cut is applied on the number of jets.
This behaviour can be altered by changing the value of the
variable {\tt notag} in the file {\tt src/User/setnotag.f}.
 
When one wishes to calculate observables related to the decay of the top
quark, {\tt removebr} should be false.
The LO calculation proceeds as normal. At NLO, there are two options:
\begin{itemize}
\item {\tt part=virt, real} or {\tt tota} : final state radiation is included
in the production stage only
\item {\tt part = todk} : radiation is included in the decay of the top
quark also and the final result corresponds to the sum of real and virtual
diagrams. This process can only be performed at NLO with 
{\tt zerowidth = .true}. This should be set automatically.
Note that these runs automatically perform an extra integration, so
will take a little longer.
\end{itemize}

The contribution from radiation in the decay may be calculated separately using
process {\tt 162}. This process number can be used with {\tt part=virt,real}
only. To ensure consistency, it is far simpler to use {\tt 161}
and this is the recommended approach.

A further option is provided for the $t-$channel single top process (when no
top quark decay is considered), relating to NLO real radiation diagrams that
contain a bottom quark. In the processes above the bottom quark is taken to
be massless. To include the effect of $m_b > 0$, one can run process
{\tt 163} ({\tt 168}) in place of {\tt 161} ({\tt 166}) and additionally include
process $\tt 231$ ({\tt 236}) at leading order.
The non-zero bottom quark mass has little effect on
the total cross section, but enables a (LO) study of the bottom quark kinematics.
Higher order corrections to the bottom quark kinematics can only be studied by running
process {\tt 231} ({\tt 236}) at NLO.

\midheading{Off-shell single top production in SM and SMEFT, processes 164,169}
\label{subsec:offstop}
\begin{center}
	[{\it For more details on this calculation, please refer to ref.~\cite{Neumann:2019kvk}}]
\end{center}

The processes 164 and 169 represent off-shell single-top-quark and anti-top-quark production
in the complex-mass scheme, respectively. Both the SM and contributions from the SMEFT can be calculated.

Dynamical double deep inelastic scattering scales can be
consistently used at NLO by setting \texttt{dynamicscale} to `DDIS'
and \texttt{scale}$=$\texttt{facscale} to 1d0. In this way the
momentum transfer along the $W$-boson $Q^2$ is used as the scale for
the light-quark-line corrections $\mu^2=Q^2$, and $\mu^2=Q^2+m_t^2$ for
the heavy-quark-line corrections. These scales are also consistently
used for the non-resonant contributions, with QCD corrections on the
$ud$-quark line, and separate QCD corrections on the bottom-quark
line.

The new block `Single top SMEFT, nproc=164,169' in the input
file governs the inclusion of SMEFT operators and corresponding
orders.  The scale of new physics $\Lambda$ can be separately set, and
has a default value of $1000$~GeV.  The flag \texttt{enable
	1/lambda4} enables the $1/\Lambda^4$ contributions, where operators
$\Qtwo, \Qfour, \Qseven$ and $\Qnine$ can contribute for the first
time.  For the non-Hermitian operators we allow complex Wilson
coefficients.  We also have a flag to disable the pure SM
contribution, leaving only contributions from SMEFT operators
either interfered with the SM amplitudes or as squared
contributions at $1/\Lambda^4$.  This can be used to directly and
quickly extract kinematical distributions and the magnitudes of
pure SMEFT contributions.

To allow for easier comparisons with previous anomalous couplings
results, and possibly estimate further higher order effects, we allow
for an anomalous couplings mode at LO by enabling the corresponding
flag.  The relations between our operators and the anomalous couplings
are

\begin{align*}
	 \delta V_L &= \Cone \frac{m_t^2}{\Lambda^2} ,\,\text{where } V_L = 1 + \delta V_L\,,\\
	 V_R &= \Ctwo{}^* \frac{m_t^2}{\Lambda^2}\,, \\
	 g_L &= -4\frac{m_W m_t}{\Lambda^2} \cdot \Cfour\,, \\
	 g_R &= -4 \frac{m_W m_t}{\Lambda^2} \cdot \Cthree{}^*\,,
\end{align*}

where $m_W$ is the $W$-boson mass, and $m_W = \frac{1}{2} g_W v$ has
been used to derive this equivalence.  Note that the minus sign for
$g_L$ and $g_R$ is different from the literature. See also the publication for more information.

For comparisons with on-shell results one needs to add up the contributions
from processes 161 at NLO and from the virt and real contributions from 162, see above.

The analysis/plotting routine is contained in the file
`\texttt{src/User/nplotter\_ktopanom.f}', where all observables
presented in this study are implemented, and the $W$-boson/neutrino
reconstruction is implemented and can be switched on or off.

\midheading{$Wt$ production, processes 180--187}
\label{subsec:wt}

\begin{center}
[{\it For more details on this calculation, please see Ref.~\cite{Campbell:2005bb}}]
\end{center}

These processes represent the production of a $W$ boson that decays leptonically
in association with a top quark. The lowest order diagram involves a gluon and
a bottom quark from the PDF, with the $b$-quark radiating a $W$ boson and
becoming a top quark. The calculation can be performed up to NLO.

Processes {\tt 180} and {\tt 185} produce a top quark that does not decay,
whilst in processes {\tt 181} and {\tt 186} the top quark decays leptonically.
Consistency with
the simpler processes ({\tt 180,185}) can be demonstrated by running process
{\tt 181,186} with {\tt removebr} set to true.

At next-to-leading order, the calculation includes contributions from diagrams
with two gluons in the initial state, $gg \rightarrow Wtb$. The $p_T$ of the
additional $b$ quark is vetoed according to the value of the parameter
{\tt ptmin\_bjet} which is specified in the input file. The contribution from
these diagrams when the $p_T$ of the $b$ quark is above {\tt ptmin\_bjet}
is zero. The values of this parameter and the factorization scale ({\tt facscale})
set in the input file should be chosen carefully. Appropriate values for both
(in the range $30$-$100$~GeV) are discussed in the associated paper.

When one wishes to calculate observables related to the decay of the top
quark, {\tt removebr} should be false.
The LO calculation proceeds as normal. At NLO, there are two options:
\begin{itemize}
\item {\tt part=virt, real} or {\tt tota} : final state radiation is included
in the production stage only
\item {\tt part = todk} : radiation is included in the decay of the top
quark also and the final result corresponds to the sum of real and virtual
diagrams. This process can only be performed at NLO with 
{\tt zerowidth = .true}. This should be set automatically.
Note that these runs automatically perform an extra integration, so
will take a little longer.
\end{itemize}

The contribution from radiation in the decay may be calculated separately using
processes {\tt 182,187}. These process numbers can be used with {\tt part=virt,real}
only. To ensure consistency, it is far simpler to use {\tt 181,186}
and this is the recommended approach.

\midheading{Di-jet production, processes 190--191}
\label{subsec:dijet}

Process {\tt 190} represents di-jet production through strong interactions.
It may be calculated to LO only.

Process {\tt 191} is an ancillary process that is used in the calculation of
weak one-loop corrections to di-jet production.  When computed at LO it gives the
contribution of weak (${\cal O}(\alpha^2)$) and mixed weak-strong (${\cal O}(\alpha\alpha_s)$)
mediated processes to di-jet production.  Please refer to Ref.~\cite{Campbell:2016dks} for details.

\midheading{$H+$~jet production with finite top-mass effects, process 200}
\label{subsec:hjetma}
This process represents the production of a Higgs boson in association with a single jet based on refs.~ 
\cite{Neumann:2016dny,Neumann:2018bsx,Budge:2020oyl}. Decay modes are currently unsupported/untested. The top-quark mass is taken 
into account 
exactly for the born and real-emission parts, as well as for the singular part of the virtual corrections.

The finite part of the two-loop virtual corrections can be computed in different ways.
\begin{itemize}
    \item In a low energy asymptotic expansion in $1/m_t^k$ up to order $k=2,4$ by setting mtex to $2$ or $4$ in the 
    input file. This is recommended for transverse momenta up to $\simeq 225$~GeV.
    \item In a high energy expansion by setting mtex=100 in the input file. This is recommended for transverse momenta 
    beyond $450$~GeV.
    \item In a rescaling approach where the finite part of the two-loop virtual amplitude in the effective field theory 
    ($m_t=\infty$) is rescaled pointwise by the ratio of the one-loop amplitude computed with full $m_t$ dependence to 
    the one-loop amplitude for $m_t=\infty$. This mode is the default mode and enabled with mtex=0 in the input file. 
    This is the recommended approach for the intermediate energy region and for estimating top-mass uncertainties in 
    the transition regions between these approaches.
\end{itemize}

\midheading{$H+$~jet production, processes 201--210}
\label{subsec:hjet}

These processes represent the production of a Higgs boson in association
with a single jet, with the subsequent decay of the Higgs to either
a pair of bottom quarks (processes {\tt 201,203,206}) 
or to a pair of tau's ({\tt 202,204,207}),
or to a pair of $W$'s which decay leptonically ({\tt 208}),
or to a pair of $Z$'s which decay leptonically ({\tt 209}),
or to a pair of photons ({\tt 210}).

The Higgs boson couples to a pair of gluons via a loop of heavy fermions
which, in the Standard Model, is accounted for almost entirely by including
the effect of the top quark alone. For processes {\tt 201,202,206,207}, the
matrix elements include the full dependence on the top quark mass.
The calculation can only be performed at LO. 
However, the Higgs boson can either be the Standard Model one
(processes {\tt 201,202}) or a pseudoscalar ({\tt 206,207}).
Note that the pseudoscalar case corresponds, in the heavy top limit, to the effective Lagrangian,
\begin{equation}
\mathcal{L} = \frac{1}{8\pi v} \, G^a_{\mu\nu} \widetilde G^{\mu\nu}_a A \;,
\end{equation}
where $\widetilde G^{\mu\nu}_a = i\epsilon^{\mu\nu\alpha\beta}
 G_{\alpha\beta}^a$.
The interaction differs from the scalar case in Eq.~{\ref{eq:HeffL}} by a factor of $3/2$
and hence the rate is increased by a factor of $(3/2)^2$.


For processes {\tt 203,204,208,209,210}, the calculation is performed in the
limit of infinite top quark mass, so that NLO results can be obtained.
The virtual matrix elements have been implemented from
refs~\cite{Ravindran:2002dc} and~\cite{Schmidt:1997wr}.
Phenomenological results have previously been 
given in refs.~\cite{deFlorian:1999zd},\cite{Ravindran:2002dc} 
and \cite{Glosser:2002gm}.
Note that the effect of radiation from the bottom quarks in process {\tt 203}
is not included.

When {\tt removebr} is true in processes {\tt 201}, {\tt 203}, {\tt 206}, {\tt 208}, {\tt 209}
and {\tt 210}, the Higgs boson does not decay.

\midheading{Higgs production via WBF, processes 211--217}
\label{subsec:wbf}

\begin{center}
[{\it For more details on this calculation, please see Ref.~\cite{Berger:2004pca}}]
\end{center}

These processes provide predictions for the production of a Higgs boson in
association with two jets via weak-boson fusion (WBF). The Higgs boson
subsequently decays to either a pair of bottom quarks
(processes {\tt 211, 216}), to a pair of tau's ({\tt 212, 217}), 
to a pair of $W$ bosons ({\tt 213}),
to a pair of $Z$ bosons ({\tt 214}),
or to a pair of photons ({\tt 215}).

Calculations can be performed up to NLO for processes {\tt 211}--{\tt 215}.
In addition to this, processes {\tt 216} and {\tt 217} provide the lowest
order calculation of the WBF reaction which radiates an additional jet. 

When {\tt removebr} is true, the Higgs boson does not decay.

\midheading{$\tau^+\tau^-$ production, process 221}
\label{subsec:tautau}

This process provides predictions for the production of a tau lepton
pair mediated by $\gamma^*/Z$, with subsequent leptonic decays. The calculation is available at LO
only. The relevant matrix elements are adapted from the ones in
ref.~\cite{Kleiss:1988xr}.

When {\tt removebr} is true, the tau leptons do not decay.


%
\midheading{$e^-e^+ \nu_{\mu} \bar\nu_{\mu} $-pair production via WBF, processes 222}
The {\it weak} processes occur in $O(\alpha^6)$, whereas the {\it strong} processes occur in $O(\alpha^4 \alpha_S^2)$.
These processes can currently only be calculated at lowest order.
\begin{eqnarray}
&222 &  f(p_1)+f(p_2) \to Z(e^-(p_3),e^+(p_4))Z(\nu_\mu(p_5),\bar{\nu}_\mu(p_6)))+f(p_7)+f(p_8) [WBF]  \nonumber \\
&2201 & f(p_1)+f(p_2) \to Z(e^-(p_3),e^+(p_4))Z(\mu^-(p_5),\mu^+(p_6)))+f(p_7)+f(p_8) [strong] \nonumber \\
&2221 & f(p_1)+f(p_2) \to Z(e^-(p_3),e^+(p_4))Z(\nu_\mu(p_5),\bar{\nu}_\mu(p_6)))+f(p_7)+f(p_8) [strong]  \nonumber
\end{eqnarray}
%
\midheading{$\nu_e  e^+ \mu^- \mu^+$-pair production via WBF, processes 223,2231}
The {\it weak} processes occur in $O(\alpha^6)$, whereas the {\it strong} processes occur in $O(\alpha^4 \alpha_S^2)$.
These processes can currently only be calculated at lowest order.
\begin{eqnarray}
& 223  & f(p_1)+f(p_2) \to W^+(\nu_e(p_3),e^+(p_4))Z(\mu^-(p_5),\mu^+(p_6)))+f(p_7)+f(p_8) [weak]  \nonumber \\
& 2231 & f(p_1)+f(p_2) \to W^+(\nu_e(p_3),e^+(p_4))Z(\mu^-(p_5),\mu^+(p_6)))+f(p_7)+f(p_8) [strong]  \nonumber
\end{eqnarray}

\midheading{$e^- \bar\nu_{e} \nu_{\mu} \mu^+$-pair production via WBF, processes 224,2241}
The {\it weak} processes occur in $O(\alpha^6)$, whereas the {\it strong} processes occur in $O(\alpha^4 \alpha_S^2)$.
These processes can currently only be calculated at lowest order.
\begin{eqnarray}
&224   & f(p_1)+f(p_2) \to W^-(e^-(p_3),\bar{\nu}_e(p_6)))W^+(\nu_\mu(p_5),\mu^+(p_4))+f(p_7)+f(p_8) [WBF]    \nonumber 
\\
&2241  & f(p_1)+f(p_2) \to W^-(e^-(p_3),\bar{\nu}_e(p_6)))W^+(\nu_\mu(p_5),\mu^+(p_4))+f(p_7)+f(p_8) [strong]  
\nonumber   
\end{eqnarray}
%
\midheading{$e^- \bar\nu_{e} \mu^- \mu^+$-pair production via WBF, processes 225,2251}
The {\it weak} processes occur in $O(\alpha^6)$, whereas the {\it strong} processes occur in $O(\alpha^4 \alpha_S^2)$.
These processes can currently only be calculated at lowest order.
\begin{eqnarray}
&225   & f(p_1)+f(p_2) \to W^-(e^-(p_3),\bar{\nu}_e(p_4))Z(\mu^-(p_5),\mu^+(p_6)))+f(p_7)+f(p_8) [weak]    \nonumber \\
&2251  &  f(p_1)+f(p_2) \to W^-(e^-(p_3),\bar{\nu}_e(p_4))Z(\mu^-(p_5),\mu^+(p_6)))+f(p_7)+f(p_8) [strong]  \nonumber  
\end{eqnarray}


\midheading{$e^- e^+ \bar\nu_{e} \nu_{e}$-pair production via WBF, processes 226}
The {\it weak} processes occur in $O(\alpha^6)$, whereas the {\it strong} processes occur in $O(\alpha^4 \alpha_S^2)$.
This process can currently only be calculated at lowest order.
\begin{eqnarray}
&226  &  f(p_1)+f(p_2) \to e-(p_3)+e^+(p_4)+\nu_e(p_5)+\bar{\nu}_e(p_6)+f(p_7)+f(p_8) [WBF]    \nonumber
\end{eqnarray}

\midheading{$\nu_{e} e^+ \nu_{\mu} \mu^+ $-pair production via WBF, processes 228,2281}
This is pure electroweak process, occuring in $O(\alpha^6)$.
These processes can currently only be calculated at lowest order.
\begin{eqnarray}
&228  &  f(p_1)+f(p_2) \to W^+(\nu_e(p_3),e^+(p_4)))W^+(\nu_\mu(p_5),\mu^+(p_6))+f(p_7)+f(p_8) [WBF]    \nonumber \\
&2281 &  f(p_1)+f(p_2) \to W^+(\nu_e(p_3),e^+(p_4)))W^+(\nu_\mu(p_5),\mu^+(p_6))+f(p_7)+f(p_8) [strong]     \nonumber
\end{eqnarray}
\midheading{$e^- \bar{\nu}_{e} \mu^- \bar{\nu}_{\mu} $-pair production via WBF, processes 229,2291}
This is pure electroweak process, occuring in $O(\alpha^6)$.
These processes can currently only be calculated at lowest order.
\begin{eqnarray}
& 229 &   f(p_1)+f(p_2) \to W^-(e^-(p_3),\bar{\nu}_e(p_4)))W^+(\mu^-(p_5),\bar{\nu}_\mu(p_6))+f(p_7)+f(p_8) [WBF]   
\nonumber \\
&2291 &  f(p_1)+f(p_2) \to W^-(e^-(p_3),\bar{\nu}_e(p_4)))W^-(\mu^-(p_5),\bar{\nu}_\mu(p_6))+f(p_7)+f(p_8) [strong]  
\nonumber  
\end{eqnarray}

\midheading{$t$-channel single top with an explicit $b$-quark, processes 231--240}
\label{subsec:stopb}

\begin{center}
[{\it For more details on this calculation, please see Ref.~\cite{Campbell:2009ss}}]
\end{center}

These represent calculations of the $t$-channel single top~({\tt 231}) and anti-top~({\tt 231})
processes in a scheme with four flavours of quark in the proton, so that $b$-quarks are not present in the proton.
The $b$-quark is instead explicitly included in the final state.

Processes {\tt 232} and {\tt 236} represent $t$-channel single top production in association
with a further jet and may be calculated at LO only.

Processes {\tt 233} and {\tt 238} are the complete four-flavour scheme $t$-channel single top production processes.
These are therefore the processes that should be used for most physics applications.
When one wishes to calculate observables related to the decay of the top
quark, {\tt removebr} should be false in processes {\tt 233} and {\tt 236}.
The LO calculation proceeds as normal. At NLO, there are two options:
\begin{itemize}
\item {\tt part=virt, real} or {\tt tota} : final state radiation is included
in the production stage only
\item {\tt part = todk} : radiation is included in the decay of the top
quark also and the final result corresponds to the sum of real and virtual
diagrams. 
Note that these runs automatically perform an extra integration, so
will take a little longer.
\end{itemize}


Processes {\tt 234} and {\tt 239} give the extra contribution due to radiation 
in top decay. These processes are mainly of theoretical interest. 

Processes {\tt 235} and {\tt 240} are the leading order single top processes with an 
extra radiated parton. These processes do not includes jets produced in the decay process.

\midheading{$W^+W^++$jets production, processes 251,252}
These processes represent the production of two $W^+$ 
bosons in association with two (process {\tt 251}) or three (process {\tt 252})
jets.  The lowest order at which two positively charged $W$ bosons 
can be produced is with two jets. 
This process is only implemented for leptonic decays of the 
$W$ particles. The calculation is available at LO only.
The calculation and code are from ref.~\cite{Melia:2010bm}.
{\tt removebr} is not implemented and has no effect.

\midheading{$Z+Q$ production, processes 261--267}
\label{subsec:ZQ}

\begin{center}
[{\it For more details on this calculation, please see Ref.~\cite{Campbell:2003dd}}]
\end{center}

These processes represent the production of a $Z$
boson that decays into a pair of electrons,
in association with a heavy quark, $Q$.

For processes {\tt 261}, {\tt 262}, {\tt 266} and {\tt 267} the initial
state at lowest order is the heavy quark and a gluon and 
the calculation may be performed at NLO.
As for $H+b$ production, the matrix elements are divided into two
sub-processes at NLO. Thus the user must sum over these after performing
more runs than usual. At lowest order one can proceed as normal, using
{\tt nproc=261} (for $Z+b$) or {\tt nproc=262} (for $Z+c$).
For a NLO calculation, the sequence of runs is as follows:
\begin{itemize}
\item Run {\tt nproc=261} (or {\tt 262}) with {\tt part=virt} and
{\tt part=real} (or, both at the same time using {\tt part=tota});
\item Run {\tt nproc=266} (or {\tt 267}) with {\tt part=real}.
\end{itemize}
The sum of these yields the cross-section with one identified heavy quark in
the final state when {\tt inclusive} is set to {\tt .false.} . To calculate the
rate for at least one heavy quark, {\tt inclusive} should be {\tt .true.}.

For processes {\tt 263} and
{\tt 264}, the calculation uses the matrix elements for the production
of a $Z$ and a heavy quark pair and demands that one of the heavy quarks
is not observed. It may either lie outside the range of $p_T$ and $\eta$
required for a jet, or both quarks may be contained in the same jet.
Due to the extra complexity (the calculation must retain the full
dependence on the heavy quark mass), this can only be computed at LO.

When {\tt removebr} is true, the $Z$ boson does not decay.

\midheading{$H + 2$~jet production, processes 270--274}

These processes represent the production of a Standard Model Higgs boson
in association with two jets. The Higgs boson
subsequently decays to either a pair of photons ({\tt nproc=270}), a bottom quark pair ({\tt nproc=271}), 
a pair of tau's ({\tt nproc=272}), a pair of leptonically decaying $W$'s ({\tt nproc=273}) 
or a pair of leptonically decaying $Z$'s ({\tt nproc=274}).

The matrix elements are included in the infinite top mass limit
using the effective Lagrangian approach. 

When {\tt removebr} is true, the Higgs boson does not decay.

\midheading{$H + 3$~jet production, processes 275-278}

These processes represent the production of a Standard Model Higgs boson
in association with three jets. The Higgs boson
subsequently decays to either a bottom quark pair ({\tt nproc=275}), 
or a pair of tau's ({\tt nproc=276})
or a pair of $W$'s that decay leptonically into a single generation of leptons ({\tt nproc=278})
or a pair of $Z$'s that decay leptonically into a single generation of leptons ({\tt nproc=279}).
The matrix elements are included in the infinite top mass limit
using an effective Lagrangian approach. These calculations can be
performed at LO only.

When {\tt removebr} is true, the Higgs boson does not decay.

\midheading{Direct $\gamma$ production, processes 280--282}
\label{subsec:dirphot}

These processes represent the production a real photon.
Since this process includes a real photon, the cross section diverges
when the photon is very soft or in the direction of the beam.
Thus in order to produce sensible results, the input file must supply values for both
{\tt ptmin\_photon} and {\tt etamax\_photon}. This will ensure that
the cross section is well-defined.

The calculation of process {\tt 282} is only available at leading order.

\midheading{Direct $\gamma$ + heavy flavour production, processes 283--284}
\label{subsec:heavyfl}

These processes represent the production a real photon with a $b$ quark
or a charm quark
Since this process includes a real photon, the cross section diverges
when the photon is very soft or in the direction of the beam.
Thus in order to produce sensible results, the input file must supply values for both
{\tt ptmin\_photon} and {\tt etamax\_photon}. This will ensure that
the cross section is well-defined.

The calculation of process {\tt 283}--{\tt 284} is only available at leading order.

\midheading{$\gamma\gamma$ production, processes 285-286}
\label{subsec:gamgam}

Process {\tt 285} represents the production of a pair of real photons.
Since this process includes two real photons, the cross section diverges
when one of the photons is very soft or in the direction of the beam.
Thus in order to produce sensible results, the input file must supply values for both
{\tt ptmin\_photon} and {\tt etamax\_photon}. This will ensure that
the cross section is well-defined.

The calculation of process {\tt 285} may be performed at NLO using either the
Frixione algorithm~\cite{Frixione:1998jh} or standard cone isolation.  This process also includes
the one-loop gluon-gluon contribution as given in
ref.~\cite{Bern:2002jx}.  The production of a photon via parton fragmentation is included at NLO and 
can be run separately by using the {\tt frag} option in {\tt part}. This option includes the contributions from the 
integrated 
photon dipole subtraction terms and the LO QCD matrix element multiplied by the fragmentation function.  

%Process {\tt 285} can be run using different cuts for each photon. Setting the first 9 characters of the runstring to 
%{\tt Stag\_phot} will apply the following default cuts:
%\begin{eqnarray*}
%p_T^{\gamma_1} > 40~\mbox{GeV}, \; p_T^{\gamma_2} > {\tt ptmin\_photon}~\mbox{GeV}, \; |\eta^{\gamma_i}| < {\tt 
%etamax\_photon}
%\end{eqnarray*} 
%These values can be changed by editing the file photon\_cuts.f in src/User. 

The phase space cuts for the final state photons are defined in {\tt{input.ini}}, for multiple photon processes such 
as {\tt 285 - 287} the $p_T$'s of the individual photons (hardest, second hardest and third hardest or softer) can be 
controlled independently. 
The remaining cuts on $R_{\gamma j}$, $\eta_{\gamma}$ etc. are applied universally to all photons. Users wishing to 
alter
this feature should edit the file {\tt{photon\_cuts.f}} in the directory {\tt{src/User}}. 


Process {\tt 286}, corresponding to $\gamma\gamma+$jet production, can be computed at leading order only.

\midheading{$\gamma\gamma\gamma$ production, process 287}
\label{subsec:trigam}

Process {\tt 287} represents the production of three real photons.
The cross section diverges
when one of the photons is very soft or in the direction of the beam.
Thus in order to produce sensible results, the input file must supply values for both
{\tt ptmin\_photon} and {\tt etamax\_photon}. This will ensure that
the cross section is well-defined.

The calculation of process {\tt 285} may be performed at NLO using either the
Frixione algorithm~\cite{Frixione:1998jh} or standard cone isolation.  The production of a photon via parton 
fragmentation is included at NLO and 
can be run separately by using the {\tt frag} option in {\tt part}. This option includes the contributions from the 
integrated 
photon dipole subtraction terms and the LO QCD matrix element multiplied by the fragmentation function.  
The phase space cuts for the final state photons are defined in {\tt{input.ini}}, for multiple photon processes such 
as {\tt 285 - 287} the $p_T$'s of the individual photons (hardest, next-to hardest and softest) can be controlled 
independently. 
The remaining cut on $R_{\gamma j}$, $\eta_{\gamma}$ etc. are applied universally to all photons. Users wishing to alter
this feature should edit the file {\tt{photon\_cuts.f}} in the directory {\tt{src/User}}. 


\midheading{$\gamma\gamma\gamma\gamma$ production, process 289}
\label{subsec:fourgam}

Process {\tt 289} represents the production of four real photons.
The cross section diverges
when one of the photons is very soft or in the direction of the beam.
Thus in order to produce sensible results, the input file must supply values for both
{\tt ptmin\_photon} and {\tt etamax\_photon}. This will ensure that
the cross section is well-defined.

The calculation of process {\tt 289} may be performed at NLO using either the
Frixione algorithm~\cite{Frixione:1998jh} or standard cone isolation.  The production of a photon via parton 
fragmentation is included at NLO and
can be run separately by using the {\tt frag} option in {\tt part}. This option includes the contributions from the 
integrated
photon dipole subtraction terms and the LO QCD matrix element multiplied by the fragmentation function.
The phase space cuts for the final state photons are defined in {\tt{input.ini}}, for multiple photon processes such
as {\tt 285 - 289} the $p_T$'s of the individual photons (hardest, next-to hardest and softest) can be controlled 
independently.
The remaining cut on $R_{\gamma j}$, $\eta_{\gamma}$ etc. are applied universally to all photons. Users wishing to alter
this feature should edit the file {\tt{photon\_cuts.f}} in the directory {\tt{src/User}}.

Note that for this process the second softest and softest photons are forced to have equal minimum $p_T$, defined
by the {\tt{[ptmin\_photon(3rd)]}} variable in the input file.


\midheading{$W\gamma$ production, processes 290-299, 2941, 2991}
\label{subsec:wgamma}

These processes represent the production of a $W$ boson which subsequently
decays leptonically, in association with a real photon.
Since this process includes a real photon, the cross section diverges
when the photon is very soft or in the direction of the beam.
Thus in order to produce sensible results, the input file must supply values for both
{\tt ptmin\_photon} and {\tt etamax\_photon}. Moreover, when the parameters {\tt zerowidth}
and {\tt removebr} are set to {\tt .false.} the decay $W \to \ell \nu$ will include
photon radiation from the lepton, so that a non-zero {\tt R(photon,lept)\_min} should
also be supplied. This will ensure that the cross section is well-defined.
Virtual amplitudes are taken from ref.~\cite{Dixon:1998py}.

The calculation of processes {\tt 290} and {\tt 295} may be performed
at NLO using the Frixione algorithm~\cite{Frixione:1998jh} or standard isolation. 

For processes {\tt 290} and {\tt 295} the role of {\tt mtrans34cut} changes to become a cut 
on the transverse mass on the $M_{345}$ system, i.e. the photon is included with the leptons in the cut. 

Processes {\tt 292} and {\tt 297} represent the $W\gamma$+jet production
processes.  They may be computed to NLO. 

Processes {\tt 294} and {\tt 299} represent the photon-induced reactions,
$p + \gamma \to e \nu \gamma j$ and should be computed at NLO. 
Processes {\tt 2941} and {\tt 2991} represent the photon-induced reactions,
$p + \gamma \to e \nu \gamma j j$ and should be computed at NLO. 

\bottomheading{Anomalous $WW\gamma$ couplings}
Processes {\tt 290}-{\tt 297} may also be computed including the effect of anomalous $WW\gamma$ couplings, making
use of the amplitudes calculated in Ref.~\cite{DeFlorian:2000sg}. Including only dimension 6 operators
or less and demanding gauge, $C$ and $CP$ invariance gives the general form of the anomalous 
vertex~\cite{DeFlorian:2000sg},
\begin{eqnarray}
 \Gamma^{\alpha \beta \mu}_{W W \gamma}(q, \bar q, p) &=& 
  {\bar q}^\alpha g^{\beta \mu} 
    \biggl( 2 + \Delta\kappa^\gamma + \lambda^\gamma {q^2\over M_W^2} \biggr) 
 - q^\beta g^{\alpha \mu}
    \biggl( 2 + \Delta\kappa^\gamma + \lambda^\gamma {{\bar q}^2\over M_W^2}
\biggr) \nonumber \\  
&& \hskip 1 cm
 + \bigl( {\bar q}^\mu - q^\mu \bigr) 
 \Biggl[ - g^{\alpha \beta} \biggl( 
   1 + {1\over2} p^2 \frac{\lambda^\gamma}{M_W^2} \biggr) 
 +\frac{\lambda^\gamma}{M_W^2} p^\alpha p^\beta \Biggr] \,,
\end{eqnarray}
where the overall coupling has been chosen to be $-|e|$. The parameters that
specify the anomalous couplings, $\Delta\kappa^\gamma$ and $\lambda^\gamma$, are
specified in the input file as already discussed in Section~\ref{subsec:diboson}.
If the input file contains a negative value for the form-factor scale $\Lambda$
then no suppression factors are applied to the anomalous couplings.
Otherwise, the couplings are included
in MCFM only after suppression by dipole form factors,
\begin{equation}
\Delta \kappa^{\gamma} \rightarrow
 \frac{\Delta \kappa_1^{\gamma}}{(1+\hat{s}/\Lambda^2)^2}, \qquad
\lambda^{\gamma} \rightarrow
 \frac{\Delta \lambda^{\gamma}}{(1+\hat{s}/\Lambda^2)^2} \;,
\end{equation}
where $\hat{s}$ is the $W\gamma$ pair invariant mass.

The Standard Model cross section is obtained by setting $\Delta\kappa^\gamma = \lambda^\gamma = 0$.

\midheading{$Z\gamma$, production, processes 300, 305}
\label{subsec:zgamma}


Processes {\tt 300} and {\tt 305} represent the production of a $Z$ boson (or virtual photon for process {\tt 300})
in association with a real photon based on ref.~\cite{Campbell:2017aul}. The $Z/\gamma^*$ subsequently decays into 
either an $e^+ e^-$ pair ({\tt nproc=300}) or neutrinos ({\tt nproc=305}).
Since these processes include a real photon, the cross section diverges
when the photon is very soft or in the direction of the beam.
Thus in order to produce sensible results, the input file must supply values for both
{\tt ptmin\_photon} and {\tt etamax\_photon}. Moreover, when the parameters {\tt zerowidth}
and {\tt removebr} are set to {\tt .false.} the decay $Z \to e^- e^+$ ({\tt nproc=300})
will include photon radiation from both leptons, so that a non-zero {\tt R(photon,lept)\_min}
should also be supplied. This will ensure that the cross section is well-defined.
The calculation of processes {\tt 300} may be performed
at NNLO using the Frixione algorithm~\cite{Frixione:1998jh} or standard isolation. 
%Processes {\tt 302} and {\tt 307}  represents the production of a $Z$ boson (or virtual photon)
%in association with a real photon and an additional jet. These processes are also available at NLO including 
%the full fragmentation processes. Anomalous couplings are not available for these processes. 
%Processes {\tt 304} and {\tt 309}  represents the production of a $Z$ boson (or virtual photon)
%in association with a real photon and and two additional jets. These processes are available at leading order only. 
%When {\tt removebr} is true in process {\tt 300} or {\tt 302} the $Z$ boson does not decay.

For the process {\tt 300}  the role of {\tt mtrans34cut} changes to become a cut 
on the invariant mass on the $M_{345}$ system, i.e. the photon is included with the leptons in the cut. 

\input{sections/ZZanom.tex}

\midheading{$Z\gamma\gamma$ production processes, 301, 306} 

Processes {\tt{301}} and {\tt{306}} represent the production of a $Z$ boson 
(or virtual photon for process {\tt 301}) in association with two photons.   The $Z/\gamma^*$ subsequently decays into 
either an $e^+ e^-$ pair ({\tt nproc=301}) or neutrinos ({\tt nproc=306}).
Since these processes include real photons, the cross section diverges
when either of the photons is very soft or in the direction of the beam.
Thus in order to produce sensible results, the input file must supply values for both
{\tt ptmin\_photon} and {\tt etamax\_photon}. Moreover, when the parameters {\tt zerowidth}
and {\tt removebr} are set to {\tt .false.} the decay $Z \to e^- e^+$ ({\tt nproc=301})
will include photon radiation from both leptons, so that a non-zero {\tt R(photon,lept)\_min}
should also be supplied. This will ensure that the cross section is well-defined.
Anomalous couplings are not currently implemented for these processes. 

\midheading{$Z\gamma j$, production, processes 302, 307}
\label{subsec:zgammajet}
Processes {\tt 302} and {\tt 307} represent the production of a $Z$ boson (or virtual photon)
in association with a real photon and at least one jet. 
The $Z/\gamma^*$ subsequently decays into 
either an $e^+ e^-$ pair ({\tt nproc=302}) or neutrinos ({\tt nproc=307}).
Since these processes include a real photon and a jet, the cross section diverges
when the photon or jet is very soft or in the direction of the beam.
Thus in order to produce sensible results, the input file must supply values for both
{\tt ptmin\_photon} and {\tt etamax\_photon}, and {\tt ptjet\_min} and {\tt etajet\_max}.
 Moreover, when the parameters {\tt zerowidth}
and {\tt removebr} are set to {\tt .false.} the decay $Z \to e^- e^+$ ({\tt nproc=302})
will include photon radiation from both leptons, so that a non-zero {\tt R(photon,lept)\_min}
should also be supplied. This will ensure that the cross section is well-defined.
The calculation of processes {\tt 302} and {\tt 307} may be performed
at NLO using the Frixione algorithm~\cite{Frixione:1998jh} or standard isolation. 
Anomalous couplings are not currently implemented for these processes. 


\midheading{$Z\gamma\gamma j$ and $Z\gamma j j $, 303, 304, 308 and 309}

These processes are available at LO only. The $Z/\gamma^*$ subsequently decays into 
either an $e^+ e^-$ pair ({\tt nproc=303,304}) or neutrinos ({\tt nproc=308,309}). 
Since these processes include a real photon and a jet, the cross section diverges
when a photon or a jet is very soft or in the direction of the beam.
Thus in order to produce sensible results, the input file must supply values for both
{\tt ptmin\_photon} and {\tt etamax\_photon}, and {\tt ptjet\_min} and {\tt etajet\_max}.
 Moreover, when the parameters {\tt zerowidth}
and {\tt removebr} are set to {\tt .false.} the decay $Z \to e^- e^+$ ({\tt nproc=303, 304})
will include photon radiation from both leptons, so that a non-zero {\tt R(photon,lept)\_min}
should also be supplied. This will ensure that the cross section is well-defined.
Anomalous couplings are not currently implemented for these processes. 



%Processes {\tt 303} and {\tt 308}  represents the production of a $Z$ boson (or virtual photon)
%in association with a two photons and and an additional jet. These processes are available at leading order only. 
%These processes do not currently have anomalous couplings implemented. 

\midheading{$W+Q+$~jet production processes 311--326}
\label{subsec:wQj}

These processes represent the production of a $W$
boson that decays leptonically,
in association with a heavy quark, $Q$ and an additional light jet. In
processes {\tt 311} and {\tt 316} $Q$ is a bottom quark, whilst
processes {\tt 321} and {\tt 326} involve a charm quark.
In these processes the quark $Q$ occurs as parton PDF in the initial state. 
The initial state in these processes consists of a light quark and a heavy 
quark, with the light quark radiating the $W$ boson. These calculations may
be performed at LO only.

When {\tt removebr} is true, the $W$ boson does not decay.

\midheading{$W+c+$~jet production, processes 331, 336}
\label{subsec:wcj}

These processes represent the production of a $W$
boson that decays leptonically,
in association with a charm quark and an additional light jet. 

In contrast to processes {\tt 321} and {\tt 326} described above, the initial
state in this case consists of two light quarks, one of which is a
strange quark which radiates the $W$ boson. The calculation may
be performed at LO only.

When {\tt removebr} is true, the $W$ boson does not decay.

\midheading{$Z+Q+$jet production, processes 341--357}
\label{subsec:ZQj}

\begin{center}
[{\it For more details on this calculation, please see Ref.~\cite{Campbell:2005zv}}]
\end{center}

These processes represent the production of a $Z$
boson that decays into a pair of electrons,
in association with a heavy quark, $Q$ and an untagged jet.

For processes {\tt 341} and {\tt 351} the initial state at lowest
order is the heavy quark and a gluon and the calculation may be
performed at NLO.  Thus in these processes the quark $Q$ occurs as
parton PDF in the initial state.  As for $H+b$ and $Z+Q$ production,
the matrix elements are divided into two sub-processes at NLO. Thus
the user must sum over these after performing more runs than usual. At
lowest order one can proceed as normal, using {\tt nproc=341} (for
$Zbj$) or {\tt nproc=351} (for $Zcj$).  For a NLO calculation, the
sequence of runs is as follows:
\begin{itemize}
\item Run {\tt nproc=341} (or {\tt 351}) with {\tt part=virt} and
{\tt part=real} (or, both at the same time using {\tt part=tota});
\item Run {\tt nproc=342} (or {\tt 352}) with {\tt part=real}.
\end{itemize}
The sum of these yields the cross-section with one identified heavy
quark and one untagged jet in the final state when {\tt inclusive} is
set to {\tt .false.} . To calculate the rate for at least one heavy
quark and one jet (the remaining jet may be a heavy quark, or
untagged), {\tt inclusive} should be {\tt .true.}.

Processes {\tt 346,347} and {\tt 356,357} are the lowest order processes that enter
the above calculation in the real contribution. They can be computed only at LO.

When {\tt removebr} is true, the $Z$ boson does not decay.

\midheading{$c \overline s \to W^+$, processes 361--363}
\label{subsec:csbar}
These processes represent the production of a $W^+$ from a charm and anti-strange
quark at LO. The $W^+$ boson decays into a neutrino and a positron.

The NLO corrections to this LO process include a contribution of the form,
$g\overline s \to W^+ \overline c$. For process {\tt 361} this contribution is
calculated in the approximation $m_c=0$ at NLO. In order to perform the NLO calculation 
for a non-zero value of $m_c$, one must instead sum the results of processes {\tt 362}
and {\tt 363} for {\tt part=tota}.

\midheading{$W\gamma\gamma$ production, processes 370-371}
\label{subsec:wgamgam}

These processes represent the production of a $W$ boson which subsequently
decays leptonically, in association with two real photons.
Since this process includes real photons, the cross section diverges
when the photon is very soft or in the direction of the beam.
Thus in order to produce sensible results, the input file must supply values for both
{\tt ptmin\_photon} and {\tt etamax\_photon}. Moreover, when the parameters {\tt zerowidth}
and {\tt removebr} are set to {\tt .false.} the decay $W \to \ell \nu$ will include
photon radiation from the lepton, so that a non-zero {\tt R(photon,lept)\_min} should
also be supplied. This will ensure that the cross section is well-defined.

These processes may be computed at leading order only.

\midheading{$W+Q$ production in the 4FNS, processes 401--408}
\label{subsec:wbbfilter}
These processes represent the production of a $W$ boson and one or more jets,
at least one of which is a $b$-quark, calculated in the 4-flavour number scheme (4FNS). 
This implies that contributions that explicitly contain a $b$-quark in the initial state
are not included.
These processes all use the same matrix
elements as processes 20 and 25 (see section~\ref{subsec:wbb}), but make different
cuts on the final state. The final state is specified by the process number and
the value of the flag {\tt inclusive}, as shown in Table~\ref{table:wbbfilter}.
An additional flag is hard-coded into the file {\tt src/User/filterWbbmas.f} to control
the inclusion of the 3-jet configuration, $(b,\overline b,j)$ when {\tt inclusive} is set to {\tt .true.}.
By default this is included, {\tt veto3jets = .false.}. If this flag is set to {\tt .true.} 
then the $(b,\overline b,j)$ contribution
is not included in processes 401, 402, 406, 407.

\begin{table}
\begin{center}
\begin{tabular}{|c|c|c|} \hline
 Process ($W^+$/$W^-$) & {\tt inclusive=.false.} & {\tt inclusive=.true.} \\
\hline
{\tt 401}/{\tt 406} & $(b)$ or $(\overline b)$ & + ($b,\overline b$) or ($b,j$) or ($\overline b,j$) \\
{\tt 402}/{\tt 407} & $(B)$ & + ($B,j$) \\
{\tt 403}/{\tt 408} & $(b,\overline b)$ & \mbox{(no extra configurations)} \\
\hline
\end{tabular}
\caption{The different final states allowed in the calculation of processes 401--408. A jet containing
both $b$ and $\overline b$ quarks is denoted by $B$ and a light (quark or gluon) jet by $j$. The inclusive
(right-hand) column also allows the final states in the exclusive (middle) column.}
\label{table:wbbfilter}
\end{center}
\end{table}

As usual, jets may be unobserved as a result of them falling outside the $p_T$
and rapidity ranges specified in the input file. In addition, the number of jets
may be different from the number of partons in the matrix element calculation as
a result of merging in the jet algorithm.

%\end{document}
\midheading{$W+Q$ production in the 5FNS, processes 411, 416}
\label{subsec:wb5FNS}

These processes represent production of a $W$ boson in association with a
$b$-jet, computed in the 5-flavour number scheme, i.e. a $b$-quark is present in
the initial state. The lowest order processes are the same as in processes {\tt 311}, {\tt 316}.
The results at NLO are not physical cross sections since part of the corrections
are not included in order to avoid double-counting with the 4FNS process (processes
{\tt 401} and {\tt 406}). To obtain combined 4FNS+5FNS predictions, the user
should select process {\tt 421} ($W^+$) or {\tt 426} ($W^-$).

\midheading{$W+Q$ production in the combined 4FNS/5FNS, processes 421, 426}
\label{subsec:wbcombined}
These processes represent the production of a $W$ boson and one or more jets,
at least one of which is a $b$-quark, calculated by combining the 4- and 5-flavour results
of processes {\tt 401}, {\tt 411} (for {\tt 421}) and {\tt 406}, {\tt 416} (for {\tt 426}).
The selection of the final state is the same as for processes {\tt 401} and {\tt 406}, as
described in Section~\ref{subsec:wbbfilter}. The procedure for combining the two
calculations is described in refs.~\cite{Campbell:2008hh,Caola:2011pz}.

\midheading{$W+b{\bar b}+$~jet production, processes 431,436}
\label{subsec:wbbjetmassive}

These processes represent the production of a $W$ boson which subsequently
decays leptonically, in association with a $b{\bar b}$ pair and an
additional jet. The effect of the bottom quark mass is included (c.f. the massless approximation
used in processes {\tt 24}, {\tt 29})
and the calculation may be performed at LO only.

When {\tt removebr} is true, the $W$ boson does not decay.

\midheading{Diboson+jet production, processes 461--487}
\label{subsec:dibosonjet}

These processes represent the production of a vector boson pair in association
with a jet.  Theuy are the counterparts of the corresponding diboson process
(\texttt{nproc-400}) described above, but also including a jet in the final
state.  They may be computed to NLO.

\midheading{$W+t{\bar t}$ processes 500--516}
\label{subsec:wttdecay}

These processes represent the production of a $W^\pm$ boson which subsequently
decays leptonically, in association with a $t{\bar t}$ pair. In all except processes 
{\tt 500} and {\tt 510} the decays of the top and anti-top quark are included.
Processes {\tt 501,502} and {\tt 511,512} refer to the semileptonic decay of the top and antitop quarks,
with the latter process in each pair giving the radiation in the decay of the top and antitop.
Process {\tt 503} ({\tt 513}) refers to the semileptonic decay of the top (antitop)
and the hadronic decay of the antitop (top). Processes {\tt 506}({\tt 516}) gives the semileptonic
decay of the antitop(top) and the hadronic decay of the top(antitop).  Processes {\tt 506}({\tt 516}) 
do not give same sign lepton events, so they may be of less phenomenological importance. For this reason 
we have not yet included radiation in the decay for these processes. 

For processes {\tt 503}, {\tt 506}, {\tt 513}
and {\tt 516} the default behaviour is that the hadronic decay products
are clustered into jets using the supplied jet
algorithm parameters, but no cut is applied on the number of jets.
This behaviour can be altered by changing the value of the
variable {\tt notag} in the file {\tt src/User/setnotag.f}.

The top quarks are always
produced on-shell, which is a necessity for a gauge invariant result
from this limited set of diagrams, but all spin correlations are included.
Switching {\tt zerowidth} from {\tt .true.} to {\tt .false.} only affects
the $W$ bosons (both the directly produced one and from the top quark decay).
Processes {\tt 501} and {\tt 511} may be run at NLO with the option {\tt todk},
including radiation in the decay of the top quark, see section \ref{subsec:ttbar}.






\midheading{$Zt{\bar t}$ production, processes 529-533}
\label{subsec:ztt}

These processes represent the production of a $Z$ boson in association
with a pair of top quarks.
For process {\tt 529}, neither the top quarks nor the $Z$-boson
decays. 
In processes {\tt 530-533}, the top quarks are always
produced on-shell, which is a necessity for a gauge invariant result
from this limited set of diagrams.
Switching {\tt zerowidth} from {\tt .true.} to {\tt .false.} only affects
the $Z$ and the $W$ bosons from the top quark decay.
In process {\tt 530} the $Z$ boson decays into an electron pair, whilst
in {\tt 531} the decay is into a massless bottom quark pair.
In process {\tt 532--533} the $Z$ boson decays into an electron pair, whilst
on or other of the top quark or top anti-quark decays hadronically.
The calculations can be performed at LO only.

For processes {\tt 532} and {\tt 533} the default behaviour is that the hadronic decay products
are clustered into jets using the supplied jet
algorithm parameters, but no cut is applied on the number of jets.
This behaviour can be altered by changing the value of the
variable {\tt notag} in the file {\tt src/User/setnotag.f}.

When {\tt removebr} is true in process {\tt 530}, the top quarks and the $Z$ boson do not decay.

\midheading{$Ht$ and $H\bar{t}$ production, processes 540--557}

\begin{center}
[{\it For more details on this calculation, please see Ref.~\cite{Campbell:2013yla}}]
\end{center}

\label{subsec:Ht}
These processes describe the production of a single top quark ({\tt 540}, {\tt 544}, {\tt 550},
{\tt 554}) or antiquark ({\tt 541}, {\tt 547}, {\tt 551}, {\tt 557}) by $W$ exchange in the
$t$-channel, in association with a Higgs boson. These processes can be performed at NLO.
For processes {\tt 540}, {\tt 541}, {\tt 550},
{\tt 551}, the top quark does not decay, but the
Higgs boson decays to $b\bar{b}$, ({\tt 540}, {\tt 541}), or to $\gamma \gamma$, ({\tt 550}, {\tt 551}).
Processes {\tt 544}, {\tt 547} and {\tt 554}, {\tt 557} include the decay of the top quark and antiquark
in the approximation in which the top quark is taken to be on shell allowing a clean separation
between production and decay. 

It is possible to study the effects of anomalous couplings of the Higgs boson to the top quark and $W$ bosons. These 
are parametrized by $c_{t\bar{t}H} = g_{t\bar{t}H}/g_{t\bar{t}H}^{SM}$ and $c_{WWH} = g_{WWH}/g_{WWH}^{SM}$ 
respectively, so that $c_{t\bar{t}H}=c_{WWH}=1$ in the SM. Different couplings may be chosen by modifying the variables 
{\tt cttH} and {\tt cWWH} in {\tt src/Need/reader$\_$input.f} and recompiling.

\midheading{$Zt$ and $Z\bar{t}$ production, processes 560--569}\

\begin{center}
[{\it For more details on this calculation, please see Ref.~\cite{Campbell:2013yla}}]
\end{center}

\label{subsec:Zt}
These processes describe the production of a single top quark (or antiquark) by $W$ exchange in the
$t$-channel, in association with a $Z$ boson. Processes {\tt 560}, {\tt 561},
{\tt 564}, {\tt 567} can be performed at NLO.
Processes {\tt  560}-{\tt 563} are for stable top quarks, whereas processes {\tt 564}-{\tt 569}
include the decay of the top quark and antiquark
in the approximation inwhich the top quark is taken to
be on shell allowing a clean separation
between production and decay.

For processes {\tt 564} and {\tt 567} the default behaviour is that the hadronic decay products
are clustered into jets using the supplied jet
algorithm parameters, but no cut is applied on the number of jets.
This behaviour can be altered by changing the value of the
variable {\tt notag} in the file {\tt src/User/setnotag.f}.

\midheading{$HH$ production, processes 601--602}
These processes represent the production of a pair of Higgs bosons.
The production proceeds through gluon-fusion one-loop diagrams involving loops 
of top quarks. The formulae implemented in the code are taken from ref.~\cite{Glover:1987nx},
where the two Higgs bosons are treated as being on-shell. To enforce this 
condition, the code sets zerowidth to true, overriding the value set in the input file.
The calculation can be performed at LO only, (i.e.\ one-loop order only).
Two decays of the Higgs bosons are currently foreseen, although other decays can easily be implemented. 
In process {\tt 601}, one Higgs boson decays to
a pair of $b$-quarks, and the other decays to a pair of $\tau$'s.
In process {\tt 602}, one Higgs boson decays to
a pair of $b$-quarks, and the other decays to a pair of photons.

\midheading{$Ht{\bar t}$ production, processes 640--660}
\label{subsec:htt}

These processes represent the production of a Higgs boson in association
with a pair of top quarks. The calculation can be performed at LO only.

For process {\tt 640}, neither the top quarks nor the Higgs boson
decays. 
In processes {\tt 641-647}, the top quarks are always
produced on-shell, which is a necessity for a gauge invariant result
from this limited set of diagrams.
Switching {\tt zerowidth} from {\tt .true.} to {\tt .false.} only affects
the Higgs and the $W$ bosons from the top quark decay.
In process {\tt 641} both the top quarks decay leptonically
and the Higgs boson decays into a pair of bottom quarks. 
Consistency with
the simpler process ({\tt 640}) can be demonstrated by running process
{\tt 641} with {\tt removebr} set to true.
In process {\tt 644} the top quark decays leptonically
and the anti-top quark decays hadronically and the Higgs boson decays into a pair of bottom quarks. 
In process {\tt 647} the anti-top quark decays leptonically
and the top quark decays hadronically and the Higgs boson decays into a pair of bottom quarks. 

Processes {\tt 651}--{\tt 657} correspond to processes {\tt 641}--{\tt 647} but with the Higgs decaying
to two photons.
Processes {\tt 661}--{\tt 667} correspond to processes {\tt 641}--{\tt 647} but with the Higgs decaying
to two $W$-bosons which subsequently decay leptonically.

\midheading{Dark Matter Processes  Mono-jet and Mono-photon 800-848} 

\begin{center}
[{\it For more details on this calculation, please see Ref.~\cite{Fox:2012ru}}]
\end{center}

\textbf{This process is currently only officially supported with version 8.0 and earlier, use at your own risk!}

This section provides an overview of the Dark Matter (DM) processes
available in MCFM. Since these processes are quite different in the
range of possible input parameters (reflecting the range of potential
BSM operators) the majority of the new features are controlled by the
file {\tt dm\_parameters.DAT} located in the {\tt Bin} directory.  We
begin this section by describing the inputs in this file.  Note that
these processes are still controlled, as usual by {\tt input.ini}
which is responsible for selecting the process, order in perturbation
theory, PDFs and phase space cuts etc. The new file controls only the
new BSM parameters in the code.

\begin{itemize} 
\item 
{\tt [dm mass]} This parameter sets the mass of the dark matter particle $m_{\chi}$. 
\item 
{\tt [Lambda]} Controls the mass scale associated with the suppression of the higher dimensional operator in the 
effective theory approach. Note that each 
operator has a well defined scaling with respect to Lambda, so cross sections and distributions obtained with one 
particular value can be readily extended to 
determine those with different $\Lambda$. 
\item
{\tt [effective theory] } Is a logical variable which controls whether or not the effective field theory is used in the 
calculation of the DM process. If this value is set to 
{\tt .false.} then one must specify the mass of the light mediator and its width (see below for more details).
\item
{\tt [Yukawa Scalar couplings]} Is a logical variable which determines if the scalar and pseudo scalar operators scale 
with the factor $m_{q}/\Lambda$ ( {\tt. .true.}) 
or 1  ({\tt .false.}).  
\item
{ \tt [Left handed DM couplings] } and { \tt [Right handed DM couplings] } 
These variables determine the coupling of the
various flavours of quarks to the DM operators.  The default value is 1. 
Note that the code uses the fact that vector operators scale as
$(L+R)$ and axial operators scale as $(L-R)$ in constructing cross
sections. Therefore you should be careful if modifying these
parameters. For the axial and pseudo scalar operators the code will
set the right-handed couplings to be the negative of the left handed
input couplings (if this is not already the case from the setup) and
inform the user it has done so. The most likely reason to want to
change these values is to inspect individual flavour operators
separately, i.e.\ to investigate an operator which only couples to up
quarks one would set all couplings to 0d0 apart from {\tt [up type]}
which would be left as 1d0.
\item 
{\tt [mediator mass]} If {\tt [effective theory]} is set to {\tt .false.} this variable controls the mass of the 
mediating particle.
\item 
{\tt [mediator width]} If {\tt [effective theory]} is set to {\tt .false.} this variable controls the width of the 
mediating particle 
\item 
{\tt [g\_x]} If {\tt [effective theory]} is set to {\tt .false.} this variable controls the coupling of the mediating 
particle to the DM.
\item 
{\tt [g\_q]} If {\tt [effective theory]} is set to {\tt .false.} this variable controls the coupling of the mediating 
particle to the quarks.
\end{itemize}

We now discuss some details of the specific DM process.

\begin{itemize}
\item 
Processes 800 and 820 produce the 
mono-jet or mono-photon signature through the following vector operator, 
\begin{eqnarray}
\mathcal{O}_V&=&\frac{(\overline{\chi}\gamma_{\mu}\chi)(\overline{q}\gamma^{\mu}q)}{\Lambda^2}~,\label{eq:OV}  
\end{eqnarray}
These processes are available at NLO and include the usual treatment of photons. See for instance the $V\gamma$ 
processes ($\sim$ 300) in this 
manual for more details on photon setup in MCFM. As discussed above the code will calculate left and right-handed 
helicity amplitudes and build the 
vector operators from $(L+R)$. Therefore you should ensure that the Left and right-handed couplings are equal in  {\tt 
dm\_parameters.DAT}. 
Processes 840 and 845 represent the production of DM plus two jets or DM plus one jet and one photon and are available 
at LO. 
\item 
Processes 801 and 821 produce the 
mono-jet or mono-photon signature through the following axial-vector operator, 
\begin{eqnarray}
\mathcal{O}_A&=&\frac{(\overline{\chi}\gamma_{\mu}\gamma_5\chi)(\overline{q}\gamma^{\mu}\gamma_5q)}{\Lambda^2}~,\label{eq:OA}
\end{eqnarray}
These processes are available at NLO and include the usual treatment
of photons. See for instance the $V\gamma$ processes ($\sim$ 300) in
this manual for more details on photon setup in MCFM. As discussed
above the code will calculate left and right-handed helicity
amplitudes and build the axial vector operators from $(L-R)$. By
default the code will enforce the right handed couplings to equal to
the negative of the left handed couplings, if this is not
already the case in {\tt dm\_parameters.DAT}. Therefore the user does
not have to change this file when switching between vector and axial
vector operators.  Processes 841 and 846 represent the production of
DM plus two jets or DM plus one jet and one photon and are available
at LO.
\item 
Processes 802 and 822 produce the 
mono-jet or mono-photon signature through the following scalar operator, 
\begin{eqnarray}
\mathcal{O}_S&=&\frac{\Delta(\overline{\chi}\chi)(\overline{q}q)}{\Lambda^2}~,
\end{eqnarray}
These processes are available at NLO and include the usual treatment
of photons. See for instance the $V\gamma$ processes ($\sim$ 300) in
this manual for more details on photon setup in MCFM. As discussed
above the code will calculate left and right-handed helicity
amplitudes and build the vector operators from $(L+R)$. Therefore you
should ensure that the Left and right-handed couplings are equal in
{\tt dm\_parameters.DAT}. For these processes $\Delta$ is fixed from
the value of {\tt [Yukawa Scalar Couplings] } if this is {\tt .true.}
then $\Delta=m_q/\Lambda$ else $\Delta=1$.

Processes 842 and 847 represent the production of DM plus two jets or DM plus one jet and one photon and are available 
at LO. 
\item 
Processes 803 and 823 produce the 
mono-jet or mono-photon signature through the following pseudo-scalar operator, 
\begin{eqnarray}
\mathcal{O}_{PS}&=&\frac{m_q(\overline{\chi}\gamma_5\chi)(\overline{q}\gamma_5q)}{\Lambda^3}\label{eq:OPS}~.
\end{eqnarray}
These processes are available at NLO and include the usual treatment
of photons. See for instance the $V\gamma$ processes ($\sim$ 300) in
this manual for more details on photon setup in MCFM. As discussed
above the code will calculate left and right-handed helicity
amplitudes and build the pseudo scalar operators from $(L-R)$. By
default the code will enforce the right handed couplings to equal to
the negative of the left handed couplings, if this is not
already the case in {\tt dm\_parameters.DAT}. Therefore the user does
not have to change this file when switching between scalar and pseudo
scalar operators.  Processes 841 and 846 represent the production of
DM plus two jets or DM plus one jet and one photon and are available
at LO.  For these processes $\Delta$ is fixed from the value of {\tt
  [Yukawa Scalar Couplings] } if this is {\tt .true.} then
$\Delta=m_q/\Lambda$ else $\Delta=1$.

Processes 843 and 848 represent the production of DM plus two jets or DM plus one jet and one photon and are available 
at LO. 

\item 
Process 804 produces the 
mono-jet signature through the following gluon induced operator, 
\begin{eqnarray}
\mathcal{O}_g&=&\alpha_s\frac{(\chi\overline{\chi})(G^{\mu\nu}_aG_{a,\mu\nu})}{\Lambda^3}~,
\end{eqnarray}
This process is available at NLO. Process 844 represents the
production of DM plus two jets and is available at LO. Since this
operator is higher dimensional, extensions to a theory in which there
is a light mediator requires the definition of two new scales (one for
the EFT in the loop defining the operator). In this version we
therefore do not consider in a theory with a light mediator.
\item 
Process 805 is a separate case of the scalar operator for top quarks
\begin{eqnarray}
\mathcal{O}^{m_t}_S&=&\frac{m_t(\overline{\chi}\chi)(\overline{q}q)}{\Lambda^3}~,
\end{eqnarray}
This process is available at LO and proceeds through a gluon loop. 
\end{itemize}
