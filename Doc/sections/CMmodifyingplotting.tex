\midheading{Modifying the plotting routines and transition function.}
The plotting infrastructure has been completely rewritten for CuTe-MCFM, and we recommended to only 
use the new infrastructure
from this point on by setting \texttt{histogram\%newstyle = .true.} in
the input file. This is the default for the CuTe-MCFM example input files.

For the processes $W^\pm,Z,H$, $\gamma\gamma$, $Z\gamma$, $ZH$
and $W^\pm H$ we include predefined plotting routines that can be
adjusted. For example for $Z$ production the plotting routine is in the file
\texttt{src/User/nplotter\_Z\_new.f90}, and similarly for the other processes.
The routine \texttt{setup} defines all histograms with custom or uniform
binning and names. The
number of used histograms needs to be allocated in this routine. The
routine \texttt{book} is called for each phase space point. Through the
boolean variable \texttt{abovecut} it is known whether the routine is
called for ``boosted $q_T=0$'' (resummed part and fixed-order expansion of
resummed part) or for $q_T>0$ (fixed-order). All provided example input files
use the transition function as defined above, see also \href{https://arxiv.org/abs/2009.11437}{arXiv:2009.11437}.

The plotting routine returns the calculated observables in the
\texttt{vals} array, and Vegas weights in \texttt{wts}. The transition
function is implemented by reweighting the original Vegas weights with
the output of the transition function. To disable the transition
function, one sets \texttt{trans} to $1$ before filling the \texttt{wts}
array.

Apart from modifying a default set of kinematical cuts in the input
file, cuts can also be set in the file
\texttt{src/User/gencuts\_user.f90} in a fully flexible way based on the
event's four momenta. Some commented out examples are included there.
