\newpage

\topheading{Z production at N$^3$LO and N$^4$LL}
\label{n3losec}

Based on \href{https://arxiv.org/abs/2207.07056}{arXiv:2207.07056} (Neumann, Campbell '22).

This page describes how to obtain Z-boson predictions at the level of up
to N$^4$LL+N$^3$LO and at a fixed order of up to N$^3$LO. The
highest order predictions are then are at the level of $\alpha_s^3$ up
to missing N$^3$LO PDFs, which both affect the logarithmic accuracy
and the fixed-order accuracy.

\textbf{Warning}: Please note that predictions at the level of
$\alpha_s^3$ are computationally very expensive due to the Z+jet NNLO
matching corrections calculated with a small (5 GeV) cutoff. Our
production plots typically run on 128 NERSC Perlmutter nodes for 12
hours, about 100k CPU hours. If you do not have these resources and are
mostly interested in the region of small $q_T$ (less than about 40 GeV), the
matching to fixed order can be performed at the level of $\alpha_s^2$.
This changes results by about 10\% above 40 GeV (missing
$\alpha_s^3$/Z+jet NNLO corrections at large $q_T$), but typically
just at the level of 2\% below 30 GeV, depending on cuts.

For $Z$ production one can start with the input file
\texttt{Bin/input\_Z.ini} that has a set of default cuts for $Z$
production, i.e.~a mass window of the lepton pair around $m_Z$
(\texttt{m34min} and \texttt{m34max} are set), and lepton minimum
transverse momenta (\texttt{ptleptmin} and \texttt{ptlept2min}, both the
same, i.e.~symmetric cuts).

After choosing a set of PDFs (\texttt{lhapdf\%lhapdfset}), beamfunctions
grids should be pre-generated by running MCFM with
\texttt{resummation\%makegrid=.true.}.


\midheading{N$^4$LL + matching at $\alpha_s^2$ fixed-order (NLO~$Z$+jet)}

The fully matched result consists of the purely resummed part, the
fixed-order Z+jet calculation and the fixed-order expansion of the
resummation to remove overlap. At N$^3$LL$^\prime$+NNLO these three
parts can be computed together automatically with
\texttt{general\%part=resNNLOp}, or with \texttt{general\%part=resNNLO}
at N$^3$LL+NNLO (\texttt{general\%part=resNLO} at NNLL+NLO). At the
level of N$^4$LL+N$^3$LO the matching is with NNLO Z+jet predictions
and, due to the computational requirements, these three parts are kept
separate and have to be assembled manually.


\bottomheading{Purely resummed N$^4$LL}

The purely resummed N$^4$LL part can be obtained by running with
\texttt{part\ =\ resonlyN3LO}. Similarly the N$^3$LL resummation is
obtained with \texttt{part\ =\ resonlyNNLO} and N$^3$LL$^\prime$
with \texttt{part\ =\ resonlyNNLOp} (see overview of configuration
options). Scale variation of hard, low and rapidity scale can be enabled
with \texttt{scales\%doscalevar\ =\ .true.}.

The resummation part will be cut off at large transverse momenta through
a transition function defined in the plotting routine. We recommend to
use the default transition function with a parameter $(q_T^2/Q^2)=0.4$
or $0.6$. The default plotting routine generates histograms with both
choices that allows for estimating a matching uncertainty.

Since the resummation becomes also invalid and numerically unstable for
$q_T>m_Z$, we select the resummation integration range between $0$
and $80$ GeV with \texttt{resummation\%res\_range=0\ 80}.



\bottomheading{Fixed-order expansion of the resummed result}

The fixed-order expansion of the resummed result (removing overlap with
fixed-order Z+jet at NLO) (in the following called resexp) can be
obtained by running \texttt{part\ =\ resexpNNLO}. We recommend a lower
cutoff of 1 GeV, setting \texttt{resexp\_range\ =\ 1.0\ 80.0} in the
\texttt{resummation} section.

This part makes use of the transition function to ensure that this part
is turned off at large $q_T$. Therefore the range is also limited to
80 GeV.



\bottomheading{Fixed-order Z+jet at NLO}

The fixed-order $\alpha_s^2$ corrections (in the following called
resabove) can be obtained by running \texttt{part\ =\ resaboveNNLO}. We
recommend a cutoff of 1 GeV, setting \texttt{fo\_cutoff\ =\ 1.0} in the
\texttt{resummation} section. This cutoff disables matching corrections
below 1 GeV and must agree with the lower value of
\texttt{resexp\_range}.


\bottomheading{Combination and scale uncertainties}

After running all three parts separately, the generated histograms can
be added manually in a plotting program. The matching corrections
consist of fixed-order result + fixed-order expansion of the resummed
result. At $\alpha_s^2$ a manual combination should agree with an
automatic combination through \texttt{part\ =\ resNNLO}, for example.

To obtain uncertainties from scale variation the following procedure
should be followed. The scales in the matching corrections must match,
i.e.~resexp\_scalevar\_01 should be added to resabove\_scalevar\_01, and
resexp\_scalevar\_02 should be added to resexp\_scalevar\_02. Note that
the scale variation histograms only give the difference to the central
value. So the minimum of the scale varied matching corrections consist
of:

\begin{verbatim}
	min(resabove + resabove_scalevar_01 + resexp + resexp_scalevar_01,
	resabove + resabove_scalevar_02 + resexp + resexp_scalevar_02)
\end{verbatim}

Similarly the maximum can be taken, both giving an envelope of
uncertainties. Note that in the resummation and its fixed-order
expansion we have not decoupled the scale in the PDFs from other scales.
Therefore when combining resexp with resabove, only the simultaneous
variation of factorization scale and renormalization scale upwards and
downwards can be used for the scale variation, corresponding to "\_01"
and "\_02".

Finally the scalevar\_maximum and scalevar\_minimum histograms of the
purely resummed result should be considered as an additional envelope.
For this part the envelope of all scale variations is taken. The
variation of the rapidity scale plays an important role and can be
enabled by setting \texttt{scalevar\_rapidity\ =\ .true.} in the
\texttt{{[}resummation{]}} section. It gives two important additional
variations to the 2, 6, or 8-point variation of hard and resummation
scale in the resummed part.

\midheading{Adding $\alpha_s^3$ matching corrections (Z+jet NNLO coefficient)}

To obtain the matching corrections at $\alpha_s^3$ we compute just the
$\alpha_s^3$ \emph{coefficient} and add it to the previously obtained
lower order results.


\bottomheading{Fixed-order Z+jet NNLO coefficient}

To obtain the fixed-order $\alpha_s^3$ corrections please run with
\texttt{part\ =\ resaboveN3LO}. We recommend a matching cutoff of 5 GeV,
setting \texttt{fo\_cutoff\ =\ 5.0} in the \texttt{resummation} section
and consequently a jettiness cutoff of \texttt{taucut=0.08} in the
\texttt{nnlo} section. It is possible to run with a larger
\texttt{fo\_cutoff} keeping the same \texttt{taucut} value, but either a
smaller \texttt{fo\_cutoff} or a larger \texttt{taucut} value will
require a new validation of results.



\bottomheading{Fixed-order Z+jet NNLO coefficient}

To obtain the fixed-order $\alpha_s^3$ corrections please run the
$Z$+jet process (\texttt{nproc=41}) with \texttt{part=nnlocoeff} in
the \texttt{{[}general{]}} section with a fixed $q_T$ cutoff, i.e.~by
setting \texttt{pt34min\ =\ 5.0} in the \texttt{{[}masscuts{]}} section.
The Z+jet calculation is based on jettiness slicing, which requires a
jettiness cutoff. For a $q_T$ cutoff of 5 GeV (for resummation this is
the matching-corrections cutoff) we recommend a jettiness cutoff of
\texttt{taucut=0.08} in the \texttt{{[}nnlo{]}} section. It is possible
to run with a larger $q_T$ cutoff, keeping the same \texttt{taucut}
value, but either a smaller $q_T$ cutoff or a larger \texttt{taucut}
value will require a new validation of results. See
\href{https://arxiv.org/abs/2207.07056}{arXiv:2207.07056} for technical
details.


\bottomheading{$\alpha_s^3$ fixed-order expansion coefficient of the resummed result}

The $\alpha_s^3$ fixed-order expansion coefficient of the resummed
result (removing overlap with fixed-order Z+jet at NNLO) can be obtained
by running \texttt{part\ =\ resexpN3LO}. \textbf{\emph{NOTE}} that this
only returns the N$^3$LO expansion \textbf{\emph{coefficient}}, to
match with the fixed-order \texttt{nnlocoeff} part. Similarly, to match
with the fixed-order part, we recommend a cutoff of 5 GeV, setting
\texttt{resexp\_range\ =\ 5.0\ 80.0} in the \texttt{resummation}
section.

\bottomheading{Combination}

Similary to the lower order, the matching corrections $\alpha_s^3$
coefficient can be added to lower order $\alpha_s^2$ results.

\midheading{Fixed order	N$^3$LO}

To compute fixed-order N$^3$LO cross-sections with $q_T$
subtractions one needs to calculate the fixed-order Z+jet NNLO
coefficient with a $q_T$ cutoff, as outlined above. The below-cut
contribution can be obtained via \texttt{part=n3locoeff} in the
\texttt{{[}general{]}} section for $Z$ production,
i.e.~\texttt{nproc=31}, where the \texttt{qtcut} value in the
\texttt{{[}nnlo{]}} section has to match the \texttt{pt34min} value
chosen for the Z+jet NNLO calculation.

We recommend to calculate the fixed-order NNLO coefficient first, as it
is instructional to understand the procedure at N$^3$LO. This proceeds
by combining NLO Z+jet result with a \texttt{pt34min} value with the
\texttt{part=nnloVVcoeff} part (below-cut at NNLO), where \texttt{qtcut}
has to be set to match the \texttt{pt34min} value. The result of this
manual procedure must agree with the automatic calculation,
i.e.~calculating Z with \texttt{part=nnlo} or \texttt{part=nnlocoeff}.
Please pay particular attention to the difference of calculating the
NNLO ($\alpha_s^2$) and N$^3$LO ($\alpha_s^3$) coefficients and
the full result.

