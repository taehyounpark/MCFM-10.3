\topheading{NNLO using non-local subtraction}
We have performed NLO and NNLO calculations using different methods and published the results in ref.~\ref{Campbell:2022gdq}.
NLO calculations can performed by three methods, (subtraction, zero-jettiness slicing and $q_T$-slicing.
NNLO calculations can be performed by zero-jettiness slicing and $q_T$-slicing.

\midheading{Benchmark results for NNLO}
\label{sec:NNLO}
\label{sec:benchmark}
We perform benchmark calculations with the default set of EW parameters, (see ref.~\ref{Campbell:2022gdq})
and for the LHC operating at $\sqrt s = 14$~TeV.  We allow all
vector bosons to be off-shell ({\tt zerowidth} is {\tt .false.})
and include their decays ({\tt removebr} is {\tt .false.}).
For each Higgs boson process we consider the decay $H \to \tau^- \tau^+$.
For parameters that are set in the input file we use,
\begin{eqnarray}
m_H = 125~\mbox{GeV} \,, \quad
m_t = 173.3~\mbox{GeV} \,, \quad 
m_b = 4.66~\mbox{GeV} \,, 
\end{eqnarray}
and we use the NNLO CT14 pdf set (i.e. {\tt pdlabel} is {\tt CT14.NN}) with
$\mu_F = \mu_R = Q^2$ (i.e. we set {\tt dynamicscale} equal to
either {\tt m(34)} or {\tt m(345)} or {\tt m(3456)}, as appropriate).
Our generic set of cuts is,
\begin{eqnarray}
&& p_T(\mbox{lepton}) > 20~\mbox{GeV} \,, \quad
|\eta(\mbox{lepton})| < 2.4 \,, \quad \nonumber \\
&& p_T(\mbox{photon 1}) > 40~\mbox{GeV} \,, \quad
   p_T(\mbox{photon 2}) > 25~\mbox{GeV} \,, \quad \nonumber \\
&&|\eta(\mbox{photon})| < 2.5 \,, \quad 
\Delta R(\mbox{photon 1, photon 2}) > 0.4 \,, \quad \nonumber \\
&& E_T^{\mbox{miss}} > 30~\mbox{GeV} \,, \quad \Delta R(\text{photon}, \text{lepton}) > 0.3 \quad
\end{eqnarray}
For $Z$ production we also impose a minimum $Z^*$ virtuality ({\tt m34min})
of $40$~GeV.

Our benchmark results are shown in Table~\ref{NNLObenchmarks} and were performed on an Intel Xeon X5650 @ 2.67GHz
system with 16 nodes of 12 cores each. MCFM was compiled with the Intel compiler and default optimizations as well
as the default MCFM setup including all pre-defined histograms.  The NNLO CPU time includes the time necessary 
for the NLO calculation. The numbers for the NLO and NNLO results were obtained independently. By tweaking the 
initial number of calls or the number of iterations per batch it is 
certainly possible to optimize the runtimes. While the numerical precision is not yet good sufficient for most of the 
fitted corrections to significantly improve the results, the fits are highly reliable and correctly estimate the 
residual $\taucut$ dependence.

\begin{table}[]
	\caption{Benchmark cross-sections at NLO and NNLO, using the parameters
		and settings described in the text. All numbers are obtained for a numerical 0.2\% precision goal.
	All NLO numbers are obtained within minutes on a desktop system, except for $Z\gamma$, which requires at the 
	order of 20-30 minutes. The NNLO CPU time includes the time for the NLO calculation.}
	\label{NNLObenchmarks} 
	\vspace{0.5em}
	\begin{tabular}{lllllll}
		\hline
		Process & \texttt{nproc} & $\taucut$ [GeV] & $\sigma^{\text{NLO}}$ & $\sigma^{\text{NNLO}}$ & fitted corr. & CPU time [h]\\
                \hline 
	       $W^+$ &  1 &   $6\cdot10^{-3}\, m_W$  &  $4.221$~nb & $4.209\pm 0.005$~nb & $-27\pm 15$~pb  & 7.6 \\	      
	       $W^-$ &  6 &   $6\cdot10^{-3}\, m_W$ & $3.315$~nb & $3.275\pm 0.004$~nb & $-25\pm 10$~pb & 7.8 \\
	       $Z$   &  31 &   $6\cdot10^{-3}\, m_Z$ & $885.3$~pb & $875.8\pm 0.9$~nb & $-3.5\pm  2$~fb & 13.0 \\
	       $H$   &  112 &	$4\cdot10^{-3}\, m_H$ & $1.396$~pb & $1.872\pm 0.002$~pb & $7\pm 6$~fb & 9.7 \\
	       $\gamma\gamma$	&  285 &   $1\cdot10^{-4}\, m_{\gamma\gamma}$ & $27.91$~pb & $43.54\pm 0.08$~pb & $0.36\pm 0.10$~pb & 83.2 \\
	       $W^+ H$   &  91 &   $3\cdot10^{-3}\, m_{W^+ H}$ & $2.204$~fb & $2.262\pm 0.004$~fb & $0.002\pm 0.008$~fb & 16.0 \\
	       $W^- H$   &  96 &   $3\cdot10^{-3}\, m_{W^- H}$ & $1.491$~fb & $1.526\pm 0.003$~fb & $-0.005\pm 0.007$~fb & 13.0 \\
	       $Z H$   &  110 &   $3\cdot10^{-3}\, m_{Z H}$ & $0.753$~fb & $0.842\pm 0.001$~fb & $-0.005\pm 0.003$~fb & 12.5 \\
	       $Z \gamma$   &  300 &   $3\cdot10^{-4}\, m_{Z \gamma}$ & $434$~fb & $525.5\pm 1.0$~fb & $4.5\pm 1.7$~fb & 202.5 \\
	 \hline
	\end{tabular}
\end{table}

