\midheading{References}
The program has been
developed over a number of years and results have been presented in
a number of published papers.  The papers describing the original
code and the most significant developments in the NLO implementation are:
\begin{itemize}
\item J.~M.~Campbell and R.~K.~Ellis, \\
  {\it ``An update on vector boson pair production at hadron colliders,''} \\
  Phys.\ Rev.\ D {\bf 60}, 113006 (1999)
  \href{https://arxiv.org/abs/hep-ph/9905386}{arXiv:hep-ph/9905386}.
\item J.~M.~Campbell, R.~K.~Ellis and C.~Williams, \\
  {\it ``Vector boson pair production at the LHC,''} \\
  JHEP {\bf 1107}, 018 (2011)
  \href{https://arxiv.org/abs/1105.0020}{arXiv:1105.0020 [hep-ph]}. 
\item J.~M.~Campbell, R.~K.~Ellis and W.~Giele, \\
  {\it ``A Multi-Threaded Version of MCFM''}, \\
    EPJ {\bf C75}, 246 (2015)
    \href{https://arxiv.org/abs/1503.06182}{arXiv:1503.06182 [hep-ph]}.
\end{itemize}
 
As of \textbf{v8.0} MCFM can also compute selected color-singlet processes through NNLO in 
QCD perturbation theory.  The processes available at this precision, as well as
benchmark numbers, are detailed in Section~\ref{sec:scetqt}.  When using MCFM 8.0 or newer
for NNLO calculations please refer to:
\begin{itemize}
\item 
  R.~Boughezal, J.~M.~Campbell, R.~K.~Ellis, \\
   C.~Focke, W.~Giele, X.~Liu,~F. Petriello and  C.~Williams, \\
  {\it ``Color singlet production at NNLO in MCFM''},
  \href{https://arxiv.org/abs/1605.08011}{arXiv:1605.08011}.
\end{itemize}

A significant overhaul of MCFM was undertaken in \textbf{v9.0} of MCFM. If you use
this version or newer please cite
\begin{itemize}
\item    John M. Campbell and Tobias Neumann,\\
  {\it ``Precision Phenomenology with MCFM''},
  \href{https://arxiv.org/abs/1909.09117}{arXiv:1909.09117}
\end{itemize}

Version \textbf{10.0} includes SCET-based $q_T$ resummation and is called CuTe-MCFM when this
feature is used. When you use the $q_T$ resummation please cite
\begin{itemize}
\item Thomas Becher and Tobias Neumann,\\
{\it ``Fiducial $q_T^2$ resummation of color-singlet processes at N$^3$LL+NNLO''},
\href{https://arxiv.org/abs/2009.11437}{arXiv:2009.11437}
\end{itemize}
The manual for the resummation functionality can be found in \texttt{cute-mcfm.pdf}.

Version \textbf{10.1} includes a C++ interface to many of the 1-loop matrix elements
included in the code. When you use this functionality please cite
\begin{itemize}
\item
  John M. Campbell, Stefan H{\"o}che and Christian T. Preuss,\\
  {\it Accelerating LHC phenomenology with analytic one-loop amplitudes: A C++ interface to MCFM},
  \href{https://arxiv.org/abs/2107.04472}{arXiv:2107.04472}
\end{itemize}
A guide to the structure of the interface is included in this paper.

Version \textbf{10.2} includes a more complete treatment of color singlet production
processes at NNLO, expecially colour singlet processes involving pairs of bosons.
\begin{itemize}
\item
    John M. Campbell, R. Keith Ellis and Satyajit Seth,\\
    {\it ``Non-local slicing approaches for NNLO QCD in MCFM''},
    \href{https://arxiv.org/abs/2202.07738}{arXiv:2202.07738 [hep-ph]}
\end{itemize}
%Other relevant references, corresponding to publications associated with the
%implementation of specific processes at NLO and NNLO, are listed
%in \ref{MCFMrefs}.
