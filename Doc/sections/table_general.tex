\begin{longtable}{p{1.5cm}p{12cm}}
		\hline
		\multicolumn{1}{c}{{\textbf{Section} \texttt{general}}} & \multicolumn{1}{c}{{\textbf{Description}}} \\ 
		\hline
		\texttt{nproc} & 
		The process to be studied is given by
		choosing a process number, according to Tables
		in Section~\ref{MCFMprocs}.
		$f(p_i)$ denotes a generic partonic jet. Processes denoted as
		``LO'' may only be calculated in the Born approximation. For photon
		processes, ``NLO+F'' signifies that the calculation may be performed
		both at NLO and also including the effects of photon fragmentation
		and experimental isolation. In contrast, ``NLO'' for a process involving
		photons means that no fragmentation contributions are included and isolation
		is performed according to the procedure of Frixione~\cite{Frixione:1998jh}.	\\
 		\texttt{part} &
		The type of calculation to be performed.  Possible values are given in 
		Tables~\ref{tab:partchoicesfo} and~\ref{tab:partchoicesresum}. \\
		\texttt{runstring} &
		When MCFM is run, it will write output to several files. The
		label {\tt runstring} will be included in the names of these files.
		\\
		\texttt{rundir} &
		Directory for output and snapshot files
		\\
		\texttt{sqrts} & Center of mass energy in GeV. \\
		\texttt{ih1}, \texttt{ih2} &
		The identities of the incoming hadrons
		may be set with these parameters, allowing simulations for both
		$p{\bar p}$ (such as the Tevatron) and $pp$ (such as the LHC). 
		Setting {\tt ih1} equal to ${\tt +1}$ corresponds to
		a proton, whilst ${\tt -1}$ corresponds to an anti-proton. \\
%		Values greater than {\tt 1000d0} represent a nuclear collision,
%		as described in Section~\ref{sec:nucleus}. \\
		\texttt{zerowidth} &
		When set to {\tt .true.} then all 
		bosons are produced on-shell. This is appropriate for calculations
		of {\it total} cross-sections (such as when using {\tt removebr} equal
		to {\tt .true.}, below). When interested in decay products of the
		bosons this should be set to {\tt .false.}. \\
		\texttt{removebr} &
		When set to {\tt .true.} the branching ratios are 
		removed for unstable particles such as vector bosons or top quarks. See the
		process notes in Section~\ref{sec:specific}, or the process
                web-pages accessed via the \href{\mcfmprocs/proclist.html}{list of processes}
                for further details. \\
		\texttt{ewcorr} & 
		Specifies whether or not to compute EW corrections
		for the process.  Default is {\tt none}.  May be set to {\tt exact}
		or {\tt sudakov} for processes {\tt 31} (neutral-current DY),
		{\tt 157} (top-pair production) and {\tt 190} (di-jet production).
		For more details see section~\ref{subsec:EW}.		\\
%                {\texttt{vdecayid}}, {\texttt{v34id}}, {\texttt{v56id}} &
%		Flags to manually set the decays of vector bosons (34) and
%		(56) (experimental, not for general use). \\
		\hline
	\end{longtable}
