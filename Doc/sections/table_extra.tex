	\begin{longtable}{p{1.5cm}p{12cm}}
		\hline
		\multicolumn{1}{c}{{\textbf{Section} \texttt{extra}}} & \multicolumn{1}{c}{{\textbf{Description}}} \\ 
		\hline
		{\tt debug} &
		A logical variable which can be used during a 
		debugging phase to mandate special behaviours. 
		Passed by common block {\tt common/debug/debug}. \\
		
		{\tt verbose} &
		A logical variable which can be used during a debugging phase to write 
		special information. Passed in common block {\tt common/verbose/verbose}. \\
		
		{\tt new\_pspace} &
		A logical variable which can be used during a debugging phase to test alternative versions of the phase space.
		Passed in common block {\tt common/new\_pspace/new\_pspace}. \\
		
		{\tt spira} & 
		A logical variable. If {\tt spira} is \texttt{.true.}, we calculate the 
		width of the Higgs boson by interpolating from a table
		calculated using the NLO code of M. Spira. The default value is \texttt{.true.}.
	        Otherwise the LO value valid for low Higgs masses only is used. \\
		
		{\tt noglue} &
		A logical variable. 
		The default value is \texttt{.false.}. If set to \texttt{.true.}, no processes
		involving initial gluons are included. \\
		{\tt ggonly} &
		A logical variable. 
		The default value is \texttt{.false.}. If set to \texttt{.true.}, 
		only the processes
		involving initial gluons in both hadrons are included.\\
		{\tt gqonly} &
		A logical variable. The default value is \texttt{.false.}. If set to \texttt{.true.}, 
		only the processes
		involving an initial gluon in one hadron and an initial quark
		or antiquark in the other hadron (or vice versa) are included.\\
		{\tt omitgg} &
		A logical variable. 
		The default value is \texttt{.false.}. If set to \texttt{.true.}, the gluon-gluon
		initial state is not included.\\
		
		{\tt clustering} &
		This logical parameter determines whether clustering is performed to yield
		jets. Only during a debugging phase should this variable be set to \texttt{.false.}. \\
		
		{\tt colourchoice} &
		If colourchoice=0, all colour structures are included ($W,Z+2$~jets).
		If colourchoice=1, only the leading 
		colour structure is included ($W,Z+2$~jets). \\
		
		{\tt rtsmin} &
		A minimum value of $\sqrt{s_{12}}$, which ensures that the invariant mass
		of the incoming partons can never be less than {\tt rtsmin}. \\
		
		
%		{\tt cutoff} & When performing calculations, the code implements a small cutoff
%		on all invariant masses in order to ensure numerical stability.  This is
%		performed according to \texttt{src/Need/smallnew.f} \\

                \texttt{reweight} & Flag to set the use of
		the user-implemented reweighting procedure \texttt{reweight\_user}
		in the routine \texttt{src/User/gencuts\_user.f90}.\\
    		\hline
	\end{longtable}
